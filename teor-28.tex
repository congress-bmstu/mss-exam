\que{Энергетические пары тензоров напряжений и деформаций. }
В законе изменения кинетической энергии использовалась величина $W_{(i)}$ -- мощность внутренних
поверхностных сил, которую мы определяли как
$W_{(i)} = - \int_V \mathbf{T} \cdot\cdot (\nabla \otimes \mathbf{v})^T \, dV$.
Обозначим $w_{(i)} = \mathbf{T} \cdot\cdot (\nabla \otimes \mathbf{v})^T$ --
\emph{мощность напряжений}, тогда $W_{(i)} = - \int_V w_{(i)} \, dV$.

\begin{definition}[\footnote{Димитриенко -- Нелинейная МСС, стр 150}]
  \emph{Энергетическими тензорами напряжений} $\stackrel{(n)}{\mathbf{T}}$,
  и \emph{энергетическими тензорами деформаций} $\stackrel{(n)}{\mathbf{C}}$
  называются такие пары тензоров, которыми наиболее удачно можно представить мощность напряжений
  в виде:
  \[
    w_{(i)} = \stackrel{(n)}{\mathbf{T}} \cdot\cdot \dfrac{d}{dt} \stackrel{(n)}{\mathbf{C}} + \mathbf{T}^K \cdot\cdot \mathbf{W},
  \]
  где $T^K = \dfrac{1}{2} (T - T^T)$ -- кососиметричная часть тензора напряжений Коши $T$, а 
  $\mathbf{W}$ -- тензор вихря.

  В случаях, когда $T = T^T$, второе слагаемое отсутствует.
\end{definition}

Выделяют 5 пар таких тензоров.

\begin{center}
  \begin{tabular}{|c|c|c|}
    \hline
    $n$ & $\stackrel{(n)}{\mathbf{T}}$ & $\stackrel{(n)}{\mathbf{C}}$ \\
    \hline
    I & $\mathbf{F}^T \cdot \mathbf{T}^S \cdot \mathbf{F}$ & $\boldsymbol{\Lambda} = \dfrac{1}{2} (\mathbf{E} - \mathbf{U}^{-2})$ \\
    II & $1/2 (\mathbf{F}^T \cdot \mathbf{T}^S \cdot \mathbf{O} + \mathbf{O}^T \cdot \mathbf{T}^S \cdot \mathbf{F})$ & $\mathbf{E} - \mathbf{U}^{-1}$ \\
    III & $\mathbf{O}^T \cdot \mathbf{T}^s \cdot \mathbf{O}$ & $\mathbf{B}$ \\
    IV & $1/2 (\mathbf{F}^{-1} \cdot \mathbf{T}^s \cdot \mathbf{O} + \mathbf{O}^T \cdot \mathbf{T}^S \cdot \mathbf{F}^{-1T})$ & $\mathbf{U} - \mathbf{E}$ \\
    V & $\mathbf{F}^{-1} \cdot \mathbf{T}^S \cdot \mathbf{F}^{-1T}$ & $\mathbf{C} = \dfrac{1}{2}(\mathbf{U}^2 - \mathbf{E})$ \\
    \hline
  \end{tabular}
\end{center}
где $T^S = \dfrac{1}{2} (T + T^T)$ - симметрическая часть тензора напряжений Коши;
$\mathbf{F}$ -- тензор градиента деформаций;
$\mathbf{O}, \mathbf{U}$ -- полярное разложение тензора градиента деформаций $\mathbf{F} = \mathbf{O} \cdot \mathbf{U}$.

% В случае третьей пары, энергетический тензор деформаций $\stackrel{(III)}{\mathbf{C}} = \mathbf{B}$
% неизвестен, но является решением следующего ДУ

% TODO проверить, что ещё можно написать.

