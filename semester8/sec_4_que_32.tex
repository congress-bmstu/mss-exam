\que{Соотношения на поверхностях сильных разрывов в идеальных газах. Соотношения Гюгонио.}

На $\Sigma$ имеет место соотношение для скачков функции:
\begin{equation*}
	\begin{cases}
		\rho_1 \left(v_{n_1} - D\right) = \rho_2 \left(v_{n_2} - D\right) = -M, \\
		\rho_1 \mathbf{v} \left(v_{n_1} - D\right) + p_1 \mathbf{n} = \rho_2 \mathbf{v} \left(v_{n_2} - D\right) + p_2 \mathbf{n}, \\
		\mathbf{T} \cdot \mathbf{n} = - p \mathbf{E} \cdot \mathbf{n} = - \mathbf{p}_n \\ 
		\rho_1 \left(v_{n_1} - D\right) \left(e_1 + \frac{v^2_n}{2}\right) + \rho_1 v_{n_1} q_{n_1} = \rho_2 \left(v_{n_2} - D\right)\left(e_2 + \frac{v_n^2}{2}\right) + \rho_2 v_{n_2} q_{n_2}.
	\end{cases}
\end{equation*}
где $\mathbf{n} \mathbf{T} \mathbf{v} = - \mathbf{n} \mathbf{E} \mathbf{v} p = - p v_n$; и введены обозначения $v_n = \mathbf{v} \cdot \mathbf{n}; \, v_{n_\alpha} = \mathbf{v}_{\alpha} \cdot \mathbf{n}; q_n = \mathbf{q} \cdot \mathbf{n}; q_{n_\alpha} = \mathbf{q}_{\alpha} \cdot \mathbf{n}$.

Преобразуем для случая:
\begin{equation*}
	M \equiv \rho_1 \left(v_{n_1} - D\right) \not \equiv 0,
\end{equation*}
т.е есть переход материальной точки через поверхность сдвига. 

Умножим скалярно на $\mathbf{n}$:
\begin{equation*}
	\rho_1 v_{n_1} \left(v_{n_1} - D\right) + p_1 = \rho_2 v_{n_2} \left(v_{n_2} - D\right) + p_2.
\end{equation*}

Введем векторы $\mathbf{r}_{\alpha}$ в касательной плоскости к $\Sigma$:
\begin{align*}
	\mathbf{n} \cdot \mathbf{r}_{\alpha} = 0, \quad \abs{\mathbf{n}}, \abs{\mathbf{r}_{\alpha}} = 1, \\
	\mathbf{n} \mathbf{r}_{\alpha} \text{ --- образуют ортогональный базис.}
\end{align*}

Обозначим $\mathbf{v}_{r_{\alpha}} = \mathbf{v}_{\alpha} \cdot \mathbf{r}_{\alpha}$ --- касательные состовляющие вектора скорости.

Имеет место соотношение:
\begin{equation*}
	\mathbf{v}_{\alpha} = \mathbf{v}_{n_\alpha} \cdot \mathbf{n} + v_{r_1} \cdot \mathbf{r}_1 + v_{r_2} \mathbf{r}_2. 
\end{equation*}

Умножаем второе соотношение из системы скалярно на $\mathbf{r}_{\alpha}$, тогда:
\begin{equation*}
	\rho_1 v_{r_\alpha 1} \left(v_{n_1} - D\right) = \rho_2 v_{r_{\alpha} 1} \left(v_{n_2} - D\right)
\end{equation*}

В силу первого соотношения системы:
\begin{equation*}
	\rho_1 \left(v_{n_1} - D\right)\left(v_{r_\alpha 1} - v_{r_\alpha 2}\right) = 0, \quad M[v_{r_\alpha}] = 0.
\end{equation*}
Таким образом, приходим к теореме.

\begin{theorem}
	В идеальном газе при наличии перехода материальных точек через поверхность разрыва $S(t)$ (ударная волна или фазовое превращение, когда $M \not = 0$), касательные составляющие скорости $v_{\tau_\alpha}$ при переходе через $S(t)$ остаются непрерывными:
	\begin{equation*}
		M \not = 0, \quad [v_{\tau_\alpha}] = 0,
	\end{equation*}
	при отсутствии переходов материальных точек через касательные составляющие скорости могут терпеть разрыв:
	\begin{equation*}
		M = 0, \quad [v_{\tau_\alpha}] \not = 0.
	\end{equation*}
\end{theorem}

Таким образом, соотношения можно записать в виде следующих скалярных соотношений:
\begin{equation}
	\begin{cases}
		\rho_1 \left(v_{n_1} - D\right) = \rho_2 \left(v_{n_2} - D\right) = -M, \\
		\rho_1 v_{n_1} \left(v_{n_1} - D\right) + p_1 = \rho_2 v_{n_2} \left(v_{n_2} - D\right) + p_2 + C_{n\Sigma}, \\
		\rho_1 \left(e_1 + \frac{v_{n_1}^2}{2}\right) \left(v_{n_1} - D\right) + p_1 v_{n_1} = \\
		\quad = \rho_2 \left(e_2 + \frac{v_{n_2}^2}{2}\right) \left(v_{n_2} - D\right) + p_2 v_{n_2} + C'_{3\Sigma}.
	\end{cases} \label{3}
\end{equation}

здесь учтено, что скачок кинетической энергии в идеальном газе:
\begin{equation*}
	[\rho\abs{\mathbf{v}} / 2] = [\rho \left(v^2_n + v^2_{\tau_1} + v^2_{\tau_2}\right) / 2] = [\rho v^2_n / 2]
\end{equation*}
--- есть скачок кинетической энергии только по нормали. 

Для соотношения скачка энтропии также:
\begin{equation*}
	\rho_1 \eta_1 \left(v_{n_1} - D\right) = \rho_2 \eta_2 \left(v_{n_2} - D\right) + C'_{4\Sigma},
\end{equation*}
где
\begin{equation*}
	C'_{4\Sigma} = C_{4\Sigma} - [q_{n}/\theta].
\end{equation*}

\paragraph{Соотношения Гюгонио}

Умножим первое уравнение системы соотношений на поверхностях разрывов из начала этого пункта на $D$ и вычтем полученное соотношение из третьей строки той же системы, тогда (речь о \ref{3}):
\begin{equation*}
	\rho_1 v_{n_1} \left(v_{n_1} - D\right) - \rho_1 D \left(v_{n_1} - D\right) + p_1 = \rho_2 v_{n_2} (v_{n_2} - D) - \rho_2 D \left(v_{n_2} - D\right) + p_2 + C_{n\Sigma}.
\end{equation*} 

или

\begin{equation*}
	\rho_1 \left(v_{n_1} - D\right)^2 + p_1 = \rho_2 \left(v_{n_2} - D\right)^2 + p_2 + C_{n\Sigma}.
\end{equation*}

Умножим это соотношение на $D$ и опять вычтем из третьей строки:
\begin{align*}
	\rho_1 e_1 \left(v_{n_1} - D\right) &+ \rho \frac{v_{n_1}^2}{2} \left(v_{n_1} - D\right) + p_1(v_{n_1} - D) - \rho_1 D \left(v_{n_1} - D\right)^2 = \rho_2 e_2 (v_{n_2} - D) + \\
	&+ \rho \frac{v^2_{n_2}}{2} \left(v_{n_2} - D\right) + p_2 \left(v_{n_2} - D\right) - \rho_2 D \left(v_{n_2} - D\right)^2 + \left(C_{3\Sigma}' - D C_{n\Sigma}\right).
\end{align*}

В этом выражении можно убрать слева и справа множители $\rho_1 \left(v_{n_1} - D\right)$ и $\rho_2 \left(v_{n_2} - D\right)$, тогда имеем:
\begin{equation*}
	e_1 + \frac{v^2_{n_1}}{2} + \frac{p_1}{\rho_1} - D \left(v_{n_1} - D\right) = e_2 + \frac{v^2_{n_2}}{2} + \frac{p_2}{\rho_2} - D \left(v_{n_2} - D\right) + C''_{3\Sigma},
\end{equation*}
где 
\begin{equation*}
	C''_{3\Sigma} = - \frac{1}{M} \left(C'_{3\Sigma} - D C_{n\Sigma}\right).
\end{equation*}
Здесь рассмотрен случай, когда $M \not = 0$.

Добавляя в левую и правую части полученного соотношения слагаемое $-D/2$ имеем:
\begin{equation*}
	e_1 + \frac{p_1}{\rho_1} + \frac{1}{2} \left(v^2_{n_1} - 2Dv_{n_1} + D^2\right) = e_2 + \frac{p_2}{\rho_2} + \frac{1}{2} \left(v_{n_2}^2 - 2 D v_{n_2} + D^2\right) + C''_{3\Sigma}.
\end{equation*}

Отсюда окончательно получаем:
\begin{equation*}
	e_1 + \frac{p_1}{\rho_1} + \frac{(v_{n_1} - D)^2}{2} = e_2 + \frac{p_2}{\rho_2} + \frac{(v_{n_2} - D)^2}{2} + C''_{3\Sigma}
\end{equation*}

Введем обозначения для относительной скорости:
\begin{equation*}
	u_1 = v_{n_1} - D, \quad u_2 = v_{n_2} - D.
\end{equation*}

Это скорости в системе координат, движущейся вместе с поверхностью разрыва $S$. Тогда соотношения запишутся в виде:
\begin{equation*}
	\begin{cases}
		\rho_1 u_1 = \rho_2 u_2, \\
		\rho_1 u^2_1 + p_1 = \rho_2 u^2_2 + p_2 + C_{n\Sigma}, \\
		e_1 + (p_1 / \rho_1) + (u_1^2 / 2) = e_2 + (p_2 / \rho_2) + (u^2_2 / 2) + C''_{3\Sigma}.
	\end{cases}
\end{equation*}

Эти соотношения называют \textit{соотношениями Гюгонио.}