\que{Классификация поверхностей раздела. Аксиома о классе функций при переходе через поверхность разрыва. Правило дифференцирования объемного интеграла при наличии поверхности разрыва.}

\paragraph{Классификация поверхностей раздела}
\begin{center}
	\begin{tikzpicture}[node distance=4cm]
		
		\node (start) [block] {Разрывы\\ (поверхность разрыва $S(t)$)};
		
		\node (slab) [block, below left of=start] {Слабые\\ $\mathring{\rho}, \mathring{A}_\alpha, \mathring{B}_\alpha$ - непрерывны по $\mathring{S}$, а их производные - терпят разрыв};
		
		\node (siln) [block, below right of=start] {Сильные\\ Хотя бы одна $\mathring{\rho}, \mathring{A}_\alpha, \mathring{B}_\alpha$ - терпят разрыв};
		
		\node (neper) [block, below left of=siln] {Материальные точки не переходят через $S(t)$};
		\node (per) [block, below right of=siln] {Материальные точки  переходят через $S(t)$};
		\node (per1) [block1, below left of=neper] {Контактный разрыв\\(ОС не меняются)};
		\node (per2) [block1, right of=per1] {Поверхность контакта\\(ОС меняются)};
		
		\node (per3) [block1,right of=per2] {Ударная волна\\(ОС не меняются)};
		\node (per4) [block1, right of=per3] {Фазовое превращение\\(ОС меняются)};
		
		\draw [arrow] (start) -- (slab);
		\draw [arrow] (start) -- (siln);
		\draw [arrow] (siln) -- (neper);
		\draw [arrow] (siln) -- (per);
		
		\draw [arrow] (neper) -- (per1);
		\draw [arrow] (neper) -- (per2);
		
		\draw [arrow] (per) -- (per3);
		\draw [arrow] (per) -- (per4);
	\end{tikzpicture}
\end{center}
	\begin{definition}
		Поверхность разрыва $S(t)$, при переходе через которую терпят разрыв сами неизвестные функции $\rho, \mathbf{u},\mathbf{v},\mathbf{T},\mathbf{F},$ и др. называют \textit{ поверхностью сильного разрыва}, если ж е эти функции остаются непрерывными, а терпят разрыв только их первые производные по t и координатам ($\mathbf{\nabla}\rho,\mathbf{\nabla}\otimes \mathbf{u}, \mathbf{\nabla}\otimes \mathbf{v}$ и т.д.), то $S(t)$ называют \textit{поверхностью слабого разрыва}.
	\end{definition}
	\begin{definition}
	Поверхность сильного разрыва $S(t)$, через которую для $\forall t\in [t_1,t_2]$ не переходят материальные точки, называют
	\textit{поверхностью контактного разрыва}, если определяющие соотношения в областях $V_+(t)$ и $V_-(t)$ одинаковы для всех рассматриваемых $t$, в противном ж е случае ее называют \textit{поверхностью контакта}.
	\end{definition}
	\begin{definition}
		Поверхность сильного разрыва $S(t)$, через которую при
	$\forall t\in [t_1,t_2]$ переходят материальные точки, называют \textit{поверхностью ударной волны}, если определяющие соотношения одинаковы для всех рассматриваемых $t$, в противном случае $S(t)$ называют \textit{поверхностью фазового перехода}.
	\end{definition}
	
	\begin{definition}
		Поверхность разрыва $S(t)$, при переходе через которую
		терпят разрывы радиусы-векторы $\mathbf{x}$ материальных точек $M \in S(t)$, называют \textit{некогерентной}, в противном случае поверхность называют \textit{когерентной}.
	\end{definition}
	
\paragraph{Аксиома о классе функций при переходе через поверхность разрыва}
Рассмотрим далее поверхность сильного разрыва $\mathring{S}(t)$, разделяющую область $\mathring{V}$ в $\mathring{\mathcal{K}}$ на две подобласти $\mathring{V}_+$ и $\mathring{V}_-$.

\begin{definition}
	Пусть имеется функция $A(\mathring{\mathbf{x}},t)$, определенная в области $\mathring{V}$. \textit{Скачком} функции \textit{через поверхность разрыва $\mathring{S}$} называют следующую величину:
	\[
	[A]=\left.A_+\right|_{\mathring{S}}-\left.A_-\right|_{\mathring{S}}
	\]
	где
	\[
	\left.A_{\pm}\right|_{\mathring{S}}=\lim_{
		\begin{aligned}
			&\mathring{\mathbf{x}}\to\mathring{\mathbf{x}}_\Sigma\\
			\mathring{\mathbf{x}}\in&\mathring{V}_{\pm},\mathring{\mathbf{x}}_\Sigma\in\mathring{S}
		\end{aligned}}A_{\pm}(\mathring{\mathbf{x}},t).
	\]
\end{definition}

\begin{axiom*}
	efwef
\end{axiom*}

\paragraph{Правило дифференцирования объемного интеграла при наличии поверхности разрыва}
