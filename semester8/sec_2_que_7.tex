\que{Классификация поверхностей раздела. Аксиома о классе функций при переходе через поверхность разрыва. Правило дифференцирования объемного интеграла при наличии поверхности разрыва.}

\paragraph{Классификация поверхностей раздела}
\begin{center}
	\begin{tikzpicture}[node distance=4cm]
		
		\node (start) [block] {Разрывы\\ (поверхность разрыва $S(t)$)};
		
		\node (slab) [block, below left of=start] {Слабые\\ $\mathring{\rho}, \mathring{A}_\alpha, \mathring{B}_\alpha$ - непрерывны по $\mathring{S}$, а их производные - терпят разрыв};
		
		\node (siln) [block, below right of=start] {Сильные\\ Хотя бы одна $\mathring{\rho}, \mathring{A}_\alpha, \mathring{B}_\alpha$ - терпят разрыв};
		
		\node (neper) [block, below left of=siln] {Материальные точки не переходят через $S(t)$};
		\node (per) [block, below right of=siln] {Материальные точки  переходят через $S(t)$};
		\node (per1) [block1, below left of=neper] {Контактный разрыв\\(ОС не меняются)};
		\node (per2) [block1, right of=per1] {Поверхность контакта\\(ОС меняются)};
		
		\node (per3) [block1,right of=per2] {Ударная волна\\(ОС не меняются)};
		\node (per4) [block1, right of=per3] {Фазовое превращение\\(ОС меняются)};
		
		\draw [arrow] (start) -- (slab);
		\draw [arrow] (start) -- (siln);
		\draw [arrow] (siln) -- (neper);
		\draw [arrow] (siln) -- (per);
		
		\draw [arrow] (neper) -- (per1);
		\draw [arrow] (neper) -- (per2);
		
		\draw [arrow] (per) -- (per3);
		\draw [arrow] (per) -- (per4);
	\end{tikzpicture}
\end{center}
\begin{definition}
	Поверхность разрыва $S(t)$, при переходе через которую терпят разрыв сами неизвестные функции $\rho, \mathbf{u},\mathbf{v},\mathbf{T},\mathbf{F},$ и др. называют \textit{ поверхностью сильного разрыва}, если ж е эти функции остаются непрерывными, а терпят разрыв только их первые производные по t и координатам ($\mathbf{\nabla}\rho,\mathbf{\nabla}\otimes \mathbf{u}, \mathbf{\nabla}\otimes \mathbf{v}$ и т.д.), то $S(t)$ называют \textit{поверхностью слабого разрыва}.
\end{definition}
\begin{definition}
	Поверхность сильного разрыва $S(t)$, через которую для $\forall t\in [t_1,t_2]$ не переходят материальные точки, называют
	\textit{поверхностью контактного разрыва}, если определяющие соотношения в областях $V_+(t)$ и $V_-(t)$ одинаковы для всех рассматриваемых $t$, в противном ж е случае ее называют \textit{поверхностью контакта}.
\end{definition}
\begin{definition}
	Поверхность сильного разрыва $S(t)$, через которую при
	$\forall t\in [t_1,t_2]$ переходят материальные точки, называют \textit{поверхностью ударной волны}, если определяющие соотношения одинаковы для всех рассматриваемых $t$, в противном случае $S(t)$ называют \textit{поверхностью фазового перехода}.
\end{definition}

\begin{definition}
	Поверхность разрыва $S(t)$, при переходе через которую
	терпят разрывы радиусы-векторы $\mathbf{x}$ материальных точек $M \in S(t)$, называют \textit{некогерентной}, в противном случае поверхность называют \textit{когерентной}.
\end{definition}

\paragraph{Аксиома о классе функций при переходе через поверхность разрыва}
Рассмотрим далее поверхность сильного разрыва $\mathring{S}(t)$, разделяющую область $\mathring{V}$ в $\mathring{\mathcal{K}}$ на две подобласти $\mathring{V}_+$ и $\mathring{V}_-$.

\begin{definition}
	Пусть имеется функция $A(\mathring{\mathbf{x}},t)$, определенная в области $\mathring{V}$. \textit{Скачком} функции \textit{через поверхность разрыва $\mathring{S}$} называют следующую величину:
	\[
	[A]=\left.A_+\right|_{\mathring{S}}-\left.A_-\right|_{\mathring{S}}
	\]
	где
	\[
	\left.A_{\pm}\right|_{\mathring{S}}=\lim_{
		\begin{aligned}
			&\mathring{\mathbf{x}}\to\mathring{\mathbf{x}}_\Sigma\\
			\mathring{\mathbf{x}}\in&\mathring{V}_{\pm},\mathring{\mathbf{x}}_\Sigma\in\mathring{S}
	\end{aligned}}A_{\pm}(\mathring{\mathbf{x}},t).
	\]
\end{definition}

\begin{axiom}\label{ax17}
	Для сплошной среды, содержащей в $\mathring{\mathcal{K}}$ поверхность разрыва $\mathring{S}(t)$, которая разделяет область $\mathring{V}$ на части $\mathring{V}_+$ и $\mathring{V}_-$, функций $\mathring{\rho}, \mathring{A}_\alpha$ и $\mathring{B}_\alpha$ в системе законов сохранения предполагаются гладкими в $\mathring{V}_+$ и $\mathring{V}_-$:
	\[
	\begin{aligned}
		\mathring{A}_\alpha=\begin{cases}
			\begin{aligned}
				&\mathring{A}_{\alpha+},&\mathring{\mathbf{x}}\in\mathring{V}_+,\\
				&\mathring{A}_{\alpha-},&\mathring{\mathbf{x}}\in\mathring{V}_-,
			\end{aligned}
		\end{cases}
		\mathring{B}_\alpha=\begin{cases}
			\begin{aligned}
				&\mathring{B}_{\alpha+},&\mathring{\mathbf{x}}\in\mathring{V}_+,\\
				&\mathring{B}_{\alpha-},&\mathring{\mathbf{x}}\in\mathring{V}_-,
			\end{aligned}
		\end{cases}
	\end{aligned}
	\]
	а на поверхности разрыва $\mathring{S}(t)$ могут иметь конечный скачок, т.е.
	\[
	\begin{aligned}
		&[\mathring{A}_\alpha] < \infty,&	[\mathring{B}_\alpha] < \infty.
	\end{aligned}
	\]
	Функции $C_\alpha$ в системе законов сохранения энергии также предполагаются гладкими в $\mathring{V}_+$ и $\mathring{V}_-$, а на поверхности разрыва могут иметь неограниченный скачок типа скачка $\delta$-функции:
	\begin{equation}\label{eq17}
	\begin{aligned}
		&\mathring{\rho}C_\alpha=\mathring{\rho}\tilde{C}_\alpha+\mathring{C}_{\alpha\Sigma}\delta(\mathring{\mathbf{x}}_\Sigma-\mathring{\mathbf{x}}),&\tilde{C}_\alpha=\begin{cases}
			\begin{aligned}
				&\mathring{C}_{\alpha+},&\mathring{\mathbf{x}}\in\mathring{V}_+,\\
				&\mathring{C}_{\alpha-},&\mathring{\mathbf{x}}\in\mathring{V}_-,
			\end{aligned}
		\end{cases}
	\end{aligned}
	\end{equation}
	где $\mathring{C}_{\alpha\Sigma}(\mathring{\mathbf{x}}_\Sigma,t)$ - конечная гладкая по $\mathring{S}(t)$ функция, удовлетворяющая условию:
	\begin{equation}\label{eq18}
	\int_{\mathring{V}}\mathring{C}_{\alpha\Sigma}\delta(\mathring{\mathbf{x}}_\Sigma-\mathring{\mathbf{x}})d\mathring{V}=\int_{\mathring{S}}\mathring{C}_{\alpha\Sigma}d\mathring{\Sigma}
	\end{equation}
	Функция $\mathring{C}_{4\Sigma}$ состоит из двух слагаемых: скачка производства энтропии за счет внешних источников $\mathring{\bar{C}}_{4\Sigma}$ и скачка производства энтропии за
	счет внутренних источников $\mathring{C}_{4\Sigma}^*$, который полагают всегда неотрицательным:
	\[
	\mathring{C}_{4\Sigma}=\mathring{\bar{C}}_{4\Sigma}+\mathring{C}_{4\Sigma}^*,~~~\mathring{C}_{4\Sigma}^*\geq 0.
	\]
\end{axiom}
	\paragraph{Правило дифференцирования объемного интеграла при наличии поверхности разрыва}
	\begin{theorem}\label{th11}
		Для любых функций $\mathring{A}_\alpha$, определенных в области $\mathring{V}$ с поверхность разрыва $\mathring{S}$ и удовлетворяющих аксиоме \ref{ax17}, имеет место \textit{правило дифференцирования объемного интеграла}:
		\begin{equation}\label{eq116}
		\frac{\partial}{\partial t}\int_{\mathring{V}}\mathring{\rho}\mathring{A}_\alpha d\mathring{V}=\int_{\mathring{V}}\mathring{\rho}	\frac{\partial\mathring{A}_\alpha}{\partial t}d\mathring{V}-\int_{\mathring{S}}\left(\mathring{\rho}_+\mathring{A}_{\alpha+}-\mathring{\rho}_-\mathring{A}_{\alpha-}\right)\mathring{D}d\mathring{\Sigma},
		\end{equation}
		где
		\[
			\mathring{D}=\mathring{\mathbf{c}}\cdot\mathring{\mathbf{n}}
		\]
		- нормальная скорость движения поверхности разрыва.
	\end{theorem}