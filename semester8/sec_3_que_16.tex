\que{Матричная  форма  записи  определяющих  соотношений  линейно  теории  упругости  для  общего  случая анизотропии.}

Свойства симметрии компонент тензора модулей упругости $C^{ijkl}$.
\begin{align*}
	&C^{ijkl} = C^{jikl} \quad \text{в силу симметрии $\sigma^{ij} = \sigma^{ji}$}; \\
	&C^{ijkl} = C^{ijlk} \quad \text{в силу симметрии $\varepsilon^{ij} = \varepsilon^{ji}$}; \\
	&C^{ijkl} = C^{klij} \quad \text{в силу $\exists$ упругого потенциала}
\end{align*}

Это является соотношениями, наложенными на компоненты $C^{ijkl}$, например: $C^{1111} = C^{1111}$ --- это тождество, а $C^{1112} = C^{1121}$ --- это соотношения.

Пусть $k$ --- число независимых компонент любого тензора, тогда его можно найти вычитанием числа независимых соотношений, положенных на этот тензор, из общего числа компонент:
\begin{equation*}
	3^{4} = 81; \text{ число соотношений = }60, k = 81 - 60 = 21.
\end{equation*}

Для разных групп симметрии, существуют дополнительно соотношения на $C^{ijkl}$, тогда для конкретных групп симметрий $G_S$: $k \leq 21$.

Рассмотрим еще одно представление соотношений линейной теории упругости.

Образуем из $\sigma_{ij}$ и $\varepsilon_{ij}$ следующие координатные столбцы:
\begin{equation*}
	\{\sigma\} = \begin{pmatrix}
		\sigma_{11} \\
		\sigma_{22} \\ 
		\sigma_{33} \\
		\sqrt{2} \sigma_{23} \\
		\sqrt{2} \sigma_{13} \\
		\sqrt{2} \sigma_{12}
	\end{pmatrix} \text{ и } \{\varepsilon\} = \begin{pmatrix}
		\varepsilon_{11} \\
		\varepsilon_{22} \\
		\varepsilon_{33} \\
		\sqrt{2}\varepsilon_{23} \\
		\sqrt{2}\varepsilon_{13} \\
		\sqrt{2}\varepsilon_{12} \\
	\end{pmatrix}
\end{equation*}

\begin{equation*}
	\left(\tensor[^4]{C}{}\right) = \begin{pmatrix}
		C^{1111} & C^{1122} & C^{1133} & \sqrt{2}C^{1123} & 
		\sqrt{2}C^{1113} & 
		\sqrt{2}C^{1112} \\ 
		& C^{2222} & C^{2233} & \sqrt{2} C^{2223} & \sqrt{2} C^{2213} & \sqrt{2} C^{2212} \\
		& & C^{3333} & \sqrt{2} C^{3323} & \sqrt{2} C^{3313} & \sqrt{2} C^{3312} \\
		& \text{сим.} &  & 2 C^{2323} & 2 C^{2312} & 2 C^{2312} \\
		& & & & 2 C^{2322} & 2 C^{2312} \\
		& & & & & 2 C^{1212} 
	\end{pmatrix}
\end{equation*}

\begin{equation*}
	\{\sigma\} = \left(\tensor[^4]{C}{}\right) \{\varepsilon\}
\end{equation*}

