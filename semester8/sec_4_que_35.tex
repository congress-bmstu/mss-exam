\que{Адиабатические процессы в идеальных газах. Адиабата Пуассона. }

\begin{definition}
  Если в уравнениях движения и энергии идеального газа (жидкости) можно пренебречь членами,
  связанными с массовыми и поверхностными притоками тепла, т.е. $q_m = 0, \mathbf{q} = 0$,
  то говорят, что рассматривается \emph{модель адиабатических процессов в идеальной жидкости
  или газе}.
\end{definition}

% **Следствие из адиабатичности.**

\paragraph{Адиабата Пуассона.}

Рассмотри уравнение баланса энтропии для идеальной жидкости:
\[
  \rho \theta \dfrac{d\eta}{dt} = - \nabla\cdot\mathbf{q} + \rho q_m + w^*,
\]
в котором $w^* = 0$, т.к. жидкость идеальная.
Тогда для адиабатических процессов:
\[
  \rho\theta \dfrac{d\eta}{dt} = 0.
\]
Так как по определению $\rho > 0$ и $\theta > 0$, то
\[
  \dfrac{d\eta}{dt} = 0
\]
в этом выражении $\eta = \eta(x^i, t)$, перейдём к лагранжевому описанию, в котором
$\eta(x^i, t) = \tilde \eta(X^i, t)$. Полная производная $\dfrac{d}{dt}$ расписывается как
\[
  \dfrac{d\eta}{dt} = \dfrac{\partial \eta}{\partial t} + \mathbf{v} \cdot \nabla \eta =
  \dfrac{\partial \tilde \eta (X^i, t)}{\partial t} = 0
\]
тогда:
\[
  \eta(X^i, t) = \mathring{\eta}(X^i)
\]
-- \emph{адиабата Пуассона} в наиболее общем виде.

\paragraph{Физический смысл адиабаты Пуассона.}
Плотность энтропии каждой материальной точки вдоль её траектории не меняется. Поэтому адиабатические процессы также называют \emph{иээнтропическими}.


Вспоминая определяющие соотношения, можно получить соотношение между плотностью и температурой каждой материальной точки:
\[
  \eta = - \dfrac{\partial \psi}{\partial \theta} =
  \eta \left(\rho(X^i, t), \theta(X^i, t) \right) = \mathring{\eta}(X^i).
\]
% -- адибата Пуассона в общем виде.
