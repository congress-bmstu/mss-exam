\que{Различные  формы  записи  (3  формы) тензора  модулей  упругости,  тензора  упругих  податливостей для трансверсально-изотропных сред.}


Для трансверсально-изотропной линейно-упругой среды (группа симетрии $\hat{G}_{s} = T_3$) инвариантное представление упругого потенциала имеет следующий вид:
\begin{gather}
	\Psi = \mathring{\rho} \psi_\theta + \frac{1}{2} \left(\lambda_1 {I^{(3)}_1}^2(\varepsilon) + \widetilde{\lambda}_2 {I^{(3)}_2}^2(\varepsilon)\right) + \widetilde{\lambda}_3 I^{(3)}_{1}(\varepsilon)I^{(3)}_2(\varepsilon) + 2 \widetilde{\lambda}_4 I^{(3)}_3(\varepsilon) + \lambda_5 I_4^{(3)}(\varepsilon), \\
	\widetilde{\lambda}_2 = \lambda_2 + \lambda_1 + 2 \lambda_3 + 2 \lambda_5, \quad \widetilde{\lambda}_3 = \lambda_1 + \lambda_3, \quad \widetilde{\lambda}_4 = \lambda_4 + \lambda_5,\nonumber
\end{gather}
где $\lambda_1, \dots, \lambda_5$ --- независимые упругие константы, число которых для трансверсально-изотропной среды равно пяти. Соответствующий тензор модулей упругости можно представить в виде:
\begin{align}
	\tensor[^4]{\mathbf{C}}{} &= \lambda_1 \mathbf{E} \otimes \mathbf{E} + \lambda_2 \widehat{\mathbf{c}}^2_3 \otimes \widehat{\mathbf{c}}^2_3 + \lambda_3 \left(\mathbf{E} \otimes \widehat{\mathbf{c}}^2_3 + \widehat{\mathbf{c}}^2_3 \otimes \mathbf{E}\right) + \\ &+ \lambda_4 \left(\mathbf{O}_1 \otimes \mathbf{O}_1 + \mathbf{O}_2 \otimes \mathbf{O}_2\right) + 2 \lambda_5 \Delta.\nonumber 
\end{align}

Компонентное представление: по анологии с ортотропным делать, только тут есть $\mathbf{E} \otimes \mathbf{E}$. Его мы будем заменять на $\delta^{ij} \delta^{kl}$; а $\Delta$ на $\delta^{ik}\delta^{jl} + \delta^{il} \delta^{jk}$. 

Матричное представление компонент $\widehat{C}^{ijkl}$ тензора имеет вид:
\begin{align}
	\left(\tensor[^4]{\mathbf{C}}{} \right) &= \left(C_{\alpha\beta}\right) = \begin{pmatrix}
		\widehat{C}^{1111} & \widehat{C}^{1122} & \widehat{C}^{1133} & 0 & 0 & 0 \\ 
		& \widehat{C}^{1111} & \widehat{C}^{1133} & 0 & 0 & 0 \\ 
		& & \widehat{C}^{3333} & 0 & 0 & 0 \\
		& \text{сим.} & & 2 \widehat{C}^{1313} & 0 & 0 \\
		& & & & 2 \widehat{C}^{1313} & 0 \\
		& & & & & \widehat{C}^{1111} - \widehat{C}^{1122}
	\end{pmatrix} = \\
	&= \begin{pmatrix}
		\lambda_1 + 2 \lambda_5 & \lambda_1 & \widetilde{\lambda}_3 & 0 & 0 & 0 \\
		& \lambda_1 + 2 \lambda_5 & \widetilde{\lambda}_3 & 0 & 0 & 0 \\
		& & \widetilde{\lambda}_2 & 0 & 0 & 0 \\ 
		& \text{сим.} & & 2\widetilde{\lambda}_4 & 0 & 0 \\
		& & & & 2 \widetilde{\lambda}_4	& 0 \\
		& & & & & 2 \widetilde{\lambda}_5
	\end{pmatrix},
\end{align}
т.е. в этом случае 
\begin{equation*}
	\widehat{C}^{2222} = \widehat{C}^{1111}, \quad \widehat{C}^{1133} = \widehat{C}^{2233}, \quad \widehat{C}^{1313} = \widehat{C}^{2323}, \quad 2\widehat{C}^{1212} = \widehat{C}^{1111} - \widehat{C}^{1122}.
\end{equation*}

Тензор податливостей $\left(\tensor[^4]{\Pi}{}\right)$ имеет аналогичную структуру:
\begin{align}
	\tensor[^4]{\Pi}{} &= \lambda'_1 \mathbf{E} \otimes \mathbf{E} + \lambda'_2 \widehat{\mathbf{c}}^2_3 \otimes \widehat{\mathbf{c}}^2_3 + \lambda'_3 \left(\mathbf{E} \otimes \widehat{\mathbf{c}}^2_3 + \widehat{\mathbf{c}}^2_3 \otimes \mathbf{E}\right) + \\ &+ \lambda'_4 \left(\mathbf{O}_1 \otimes \mathbf{O}_1 + \mathbf{O}_2 \otimes \mathbf{O}_2\right) + 2 \lambda'_5 \Delta \equiv \widehat{\Pi}^{ijkl} \widehat{\mathbf{c}}_i \otimes \widehat{\mathbf{c}}_j \otimes \widehat{\mathbf{c}}_k \otimes \widehat{\mathbf{c}}_l.\nonumber 
\end{align}
где пять констант $\lambda'_1, \dots, \lambda'_5$ обычно выражают через пять трансверсально-изотропных технических констант:
\begin{align*}
	\lambda_1' = - v / E, \quad \lambda'_1 + \lambda_2' + 2 \lambda_3' + 2 \lambda_5' = 1 / E', \quad \lambda'_1 + \lambda'_3 = - v'/E', \nonumber \\
	\lambda_1' + 2 \lambda_5' = 1 / E, \quad \lambda_4' + \lambda_5' = 1 / \left(2 G'\right), \quad 2 \lambda_5' = 1 / \left(2 G\right).
\end{align*}

Их можно связать с теми, что используются в ортотропных средах:
\begin{align*}
	E' = E_3, \quad E = E_1 = E_2, \\
	v = v_{12}, \quad v'=v_{13}=v_{23}, \\
	G=G_{12}, \quad G'=G_{13}=G_{23},
\end{align*}
их обычно называют $E'$ --- модуль упругости в продольном направлении; $E$ --- модуль упругости в поперечном направлении; $v$ и $G$ --- коэффициент Пуассона и модуль сдвига в плоскости трансверсальной изотропии; $v'$ и $G'$ --- коэффициент Пуассона и модуль сдвига в продольном направлении.

Для трансверсально-изотропной среды $G$ можно выразить через $v$ и $E$:
\begin{equation*}
	G = \frac{E}{2 (1 + v)}
\end{equation*}

Матричное представление компонент $\widehat{\Pi}^{ijkl}$ тензора в базисе $\widehat{\mathbf{c}}_i$ имеет следующий вид:
\begin{align}
	\left(\tensor[^4]{\Pi}{} \right) &= \left(\Pi_{\alpha\beta}\right) = \begin{pmatrix}
		\widehat{\Pi}^{1111} & \widehat{\Pi}^{1122} & \widehat{\Pi}^{1133} & 0 & 0 & 0 \\ 
		& \widehat{\Pi}^{1111} & \widehat{\Pi}^{1133} & 0 & 0 & 0 \\ 
		& & \widehat{\Pi}^{3333} & 0 & 0 & 0 \\
		& \text{сим.} & & 2 \widehat{\Pi}^{1313} & 0 & 0 \\
		& & & & 2 \widehat{\Pi}^{1313} & 0 \\
		& & & & & \widehat{\Pi}^{1111} - \widehat{\Pi}^{1122}
	\end{pmatrix} = \\
	&= \begin{pmatrix}
		1 / E & - v / E & -v'/E' & 0 & 0 & 0 \\
		& 1 / E& - v' / E & 0 & 0 & 0 \\
		& & 1 / E' & 0 & 0 & 0 \\ 
		& \text{сим.} & & 1 / \left(2 G\right) & 0 & 0 \\
		& & & & 1 / \left(2 G'\right)	& 0 \\
		& & & & &1 / \left(2 G'\right)
	\end{pmatrix},
\end{align}