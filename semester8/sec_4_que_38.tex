\que{Соотношения Гюгонио для адиабатических процессов в совершенном газе.}

Пусть в совершенном газе есть поверхность разрыва, до разрыва газ
имеет макроскопические характеристики $\rho_1, p_1, u_1$, после разрыва -- $\rho_2, p_2, u_2$
(по обе стороны от разрыва находится одинаковый газ),
тогда соотношения Гюгонио:
\[
  \begin{cases}
    \rho_1 u_1 = \rho_2 u_2, \\
    \rho_1 u_1^2 + p_1 = \rho_2 u_2^2 + p_2, \\
    \dfrac{k}{k-1} \dfrac{p_1}{\rho_1} + \dfrac{u_1^2}{2} = \dfrac{k}{k-1} \dfrac{p_2}{\rho_2} + \dfrac{u_2^2}{2}.
  \end{cases}
\]
% Пусть $\mathbf{D}$ -- скорость движения поверхности разрыва.

Найдём значения $\rho_1, p_1, u_1$ при известных $\rho_2, p_2, u_2$.

Обозначим $\gamma = \dfrac{\rho_2}{\rho_1}$, тогда
\[
  \begin{cases}
    u_1 = \gamma u_2, \\
    p_1 = p_2 + \rho_2 u_2^2 (1 - \gamma), \\
    \dfrac{k}{k-1} \dfrac{\rho_2}{\rho_1} u_2^2 (1 - \gamma) + \dfrac{k}{k-1} \dfrac{p_2}{\rho_1} \dfrac{\rho_2}{\rho_2} + \dfrac{\gamma^2 u_2^2}{2} - \dfrac{k}{k-1} \dfrac{p_2}{\rho_2} - \dfrac{u_2^2}{2} = 0
  \end{cases}
\]
преобразуем последнее выражение к виду:
\[
 \gamma(1-\gamma) + (\gamma^2 - 1) \dfrac{k-1}{2k}+ \dfrac{p_2}{u_2^2 \rho_2} (\gamma-1) = 0
\]
или (обозначили $B = \dfrac{p_2}{\rho_2 u_2^2}$)
\[
  \gamma^2 - \dfrac{2k}{k+1} (1 + B) \gamma + \dfrac{k-1}{k+1} = 0
\]
тогда
\[
  \gamma_{1, 2} = \dfrac{1}{1+k} \left( k(1+B) \pm (1 - kB) \right)
  \Leftrightarrow
  \gamma_1 = \dfrac{k-1 +2kB}{k+1}, \; \gamma_2 = 1,
\]
случай $\gamma = 1$ не рассматриваем, т.к. в этом случае никакого разрыва нет (можно проверить,
что тогда $\rho_1=\rho_2, u_1=u_2, p_1=p_2$ из соотношений Гюгонио).

Таким образом, при адиабатических процессах в совершенном газе с постоянными теплоёмкостями
и при отсутствии поверхностных эффектов на поверхности разрыва соотношения Гюгонио принимают вид:
\[
  \begin{cases}
    \dfrac{1}{\rho_1} = \dfrac{k-1}{k+1}\dfrac{1}{\rho_2} + \dfrac{2k}{k+1} \dfrac{p_2}{\rho_2^2 u_2^2}, \\
    p_1 = \dfrac{2}{k+1} \rho_2 u_2^2 - \dfrac{k-1}{k+1} p_2, \\
    u_1 = \dfrac{k-1}{k+1} u_2 + \dfrac{2k}{k+1} \dfrac{p_2}{\rho_2 u_2}.
  \end{cases}
  
\]
