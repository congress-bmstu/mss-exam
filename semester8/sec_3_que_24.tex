\que{Постановка задачи теории упругости в напряжениях. Тензор функций напряжений. Формулы Чезаро.}

\subsection*{Постановка задачи теории упругости в напряжениях. Тензор функций напряжений.}

Рассмотрим случай, когда на границе области \( V \) задан только вектор усилий \( \vec{S}_e \):
%картинка 241
\begin{equation}\label{eq:base_system}
    \begin{cases}
        \nabla \cdot \sigma + \rho \vec{f} = 0, & \vec{x} \in V \\
        \varepsilon = \frac{1}{4} \Pi \cdot\cdot \sigma, & \vec{x} \in V \cup \varepsilon \\
        \vec{n} \cdot \sigma |_{\Sigma} = \vec{S}_e, & \vec{x} \in \Sigma \\
        \text{Ink} \, \varepsilon = 0, & \vec{x} \in V
    \end{cases}
\end{equation}

Если уравнения совместности деформаций (УСД) не присоединять, то задача будет недоопределённая. УСД в этом случае не удовлетворяются тождественно.

Соотношения Коши 

\begin{equation}\label{eq:cauchy_relations}
    \varepsilon = \frac{1}{2} (\nabla \otimes \vec{u} + \nabla \otimes \vec{u}^T)
\end{equation}

не включаются в общую систему \eqref{eq:base_system}. Они служат для вычисления перемещений по заданному полю \( \varepsilon \), но решения этих уравнений неоднозначны, так как не заданы граничные условия по перемещению. С учетом УСД система \eqref{eq:base_system} становится переопределенной. 

Решение задачи \eqref{eq:base_system} существует не всегда, а только при определенных ограничениях на \( \vec{f} \) и \( \vec{S}_e \).

Если массовые силы потенциальны:

\begin{equation}\label{eq:mass_forces}
    \vec{f} = \nabla \chi
\end{equation}

можно ввести новый симметричный тензор второго ранга \( \Phi \):

\begin{equation}\label{eq:phi_tensor}
    \Phi_{\sigma} = \text{Ink} \, \Phi - \chi E
\end{equation}

Уравнение \eqref{eq:phi_tensor} представляет собой 6 уравнений относительно 6 компонент тензора \( \Phi \).

После введения тензора \( \Phi \) уравнения равновесия удовлетворяются тождественно для любой области \( V \):

\begin{align}\label{eq:equilibrium}
    \nabla \cdot \sigma 
    &= \nabla \cdot \text{Ink} \, \Phi - \nabla (\chi E) \notag \\
    &= \frac{1}{g} \nabla_s \epsilon^{snj} \epsilon^{pik} \nabla_i \nabla_n \Phi_{kj} \cdot \vec{r}_p 
    - \nabla \chi \notag \\
    &= \vec{r}_p \epsilon^{pik} \nabla_i (\epsilon^{snj} \nabla_s \nabla_n \Phi_{kj}) - \nabla \chi \notag \\
    &= -\nabla \chi
\end{align}

Таким образом:

\begin{equation}\label{eq:balance}
    0 = \nabla \cdot \sigma + \rho \vec{f} = -\nabla \chi + \nabla \chi = 0
\end{equation}

Следовательно, система \eqref{eq:base_system} преобразуется в следующую:

\begin{equation}\label{eq:new_system}
    \begin{cases}
        \text{Ink} \, \varepsilon = 0 \\
        \varepsilon = ^4\Phi \cdot\cdot \sigma \\
        \sigma = \text{Ink} \, \Phi - \chi E
    \end{cases}
\end{equation}

Где:

\begin{equation}\label{eq:boundary_conditions}
    \begin{cases}
        \vec{n} \cdot \text{Ink} \, \Phi |_{\Sigma} = \vec{S}_e + \chi \vec{n}, & \vec{x} \in \Sigma \\
        \text{Ink} \left(^4\Phi \cdot\cdot (\text{Ink} \, \Phi - \chi E)\right) = 0, & \vec{x} \in V
    \end{cases}
\end{equation}

В системе \eqref{eq:new_system} имеется 6 уравнений относительно 6 неизвестных компонент тензора \( \Phi \), то есть система замкнутая. Это задача в «напряжениях» относительно функций тензора напряжения \( \Phi \). Перемещения не могут быть определены из этой задачи.

\subsection*{Формулы Чезаро}
%картинка 242
Эти формулы позволяют найти вектор перемещений \( \vec{u} \), если задано поле тензора деформаций \( \varepsilon \) в области \( V \):

\begin{equation}\label{eq:cesaro_formulas}
    \vec{u}(\vec{x}) = \vec{u}_0 + \vec{\omega}_0 \times (\vec{x} - \vec{x}_0) + 
    \int_{\alpha} \left( \varepsilon + (\tilde{\vec{x}} - \vec{x}) \times \nabla \times \varepsilon \right) d\tilde{\vec{x}}
\end{equation}

Где:
- \( \vec{u}_0 \), \( \vec{\omega}_0 \), \( \vec{x}_0 \) — постоянные векторы;
- \( \alpha \) — произвольная кривая в \( V \).

Эта форма справедлива только для односвязных областей \( V \).

- \( \vec{u}_0 \) — вектор перемещений тела в точке \( \vec{x}_0 \);
- \( \vec{\omega}_0 \) — вектор поворота тела в точке \( \vec{x}_0 \).

Эти величины исключают движение тела как жёсткого целого, делая движение тела определённым.