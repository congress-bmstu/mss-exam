\que{Модель линейно-упругих сред. Постановка квазистатической задачи линейно-теории упругости}

(в этом вопросе есть несостыковки по названиям моделей)

\paragraph{Модель линейно-упругих сред.}
\begin{definition}
  (идеальную) Твёрдую среду с малыми деформациями, для которой оператор свободной энергии Гельмгольца
  $\breve{\psi}$ имеет вид
  \begin{align*}
    \breve{\psi}(\varepsilon, \theta) &= \psi_\theta + \dfrac{1}{2\mathring{\rho}} (\varepsilon - \mathring{\varepsilon}) \cdot\cdot {}^4 C \cdot\cdot (\varepsilon - \mathring{\varepsilon}), \\
    \psi_\theta &= \psi_0 + \int\limits_{\theta_0}^\theta c_v(\theta') \, d\theta' - \theta \int\limits_{\theta_0}^\theta \dfrac{c_v(\theta')}{\theta'} \, d\theta',
  \end{align*}
  называют \emph{линейно-упругой средой}.

  Здесь $\psi_0$ -- начальное значение свободной энергии, $\psi_0 = e_0 - \theta \eta_0$;
  $c_v(\theta)$ -- теплоёмкость;
  ${}^4 C (\theta)$ -- тензор 4-го ранга, называемый \emph{тензором модулей упругости};
  $\mathring{\varepsilon}(\theta)$ -- тензор 2-го ранга, называемый \emhp{тензором тепловой деформации}.
  Все эти величины являются заданными.
\end{definition}
\begin{remark}
  В книжке почему-то не было указано, что эта среда идеальная, но Димитриенко сказал, что
  слово <<упругая>> подразумевает идеальность, соответственно дальше это определение было
  подставлено в общий вид определяющих соотношений для идеальной твёрдой среды с малыми
  деформациями.
\end{remark}

Для линейно-упругой среды определяющие соотношения принимают вид:
\begin{align*}
  \sigma &= {}^4 C \cdot\cdot (\varepsilon - \mathring{\varepsilon}), \\
  \eta &= \eta_0 + \int\limits_{\theta_0}^\theta \dfrac{c_v}{\theta'} \, d\theta' + \dfrac{1}{\mathring{\rho}} \alpha \cdot\cdot {}^4 C \cdot\cdot (\varepsilon - \mathring{\varepsilon}) - \dfrac{1}{2\mathring{\rho}} (\varepsilon - \mathring{\varepsilon}) \cdot\cdot {}^4 C_\theta \cdot\cdot (\varepsilon - \mathring{\varepsilon}), \\
  e &= \psi + \theta \eta = e_0 + \int\limits_{\theta_0}^\theta c_v \, d\theta' + \dfrac{1}{2\mathring{\rho}} (\varepsilon - \mathring{\varepsilon}) \cdot\cdot \left( {}^4 C - \theta \cdot {}^4 C_\theta \right) \cdot\cdot (\varepsilon - \mathring{\varepsilon}) + \dfrac{1}{\mathring{\rho}} \alpha \theta \cdot\cdot{}^4 C \cdot\cdot (\varepsilon - \mathring{\varepsilon}),
\end{align*}
где $\mathring{\varepsilon} = \int\limits_{\theta_0}^\theta \alpha(\theta') \, d\theta'$,
${}^4 C = \int\limits_{\theta_0}^\theta {}^4 C_\theta(\theta') \, d\theta'$,
тензор $\alpha$ называют \emph{тензором теплового расширения}. 

Если относительное изменение температуры $(\theta - \theta_0) / \theta_0$ не слишком велико,
то для большинства твёрдых сред тензоры ${}^4 C$ и $\alpha$ можно считать не зависящими
от температуры, в этом случае
\[
  \alpha = \alpha_0 = \operatorname{const}, \quad {}^4 C_\theta = 0, \quad \mathring{\varepsilon} = \alpha ( \theta - \theta_0 ).
\]
такие среды называют \emph{линейно-термоупругими средами с независящими от температуры свойствами}.

Если рассматривают изотермические процессы с $\theta \equiv \theta_0$, то
$\mathring{\varepsilon} \equiv 0$ и модель называют \emph{моделью линейно-упругой среды}.
(что????)

Модель линейно-термоупругой среды предполагают всегда обратимой, т.е. всегда существует
обратное соотношение к обобщённому закону Гука:
\[
  \varepsilon = {}^4 \Pi \cdot\cdot \sigma,
\]
где тензор ${}^4 \Pi$, называемый \emph{тензором упругих податливостей}, является обратным 
к ${}^4 C$: ${}^4 C \cdot\cdot {}^4 \Pi = E$.

\paragraph{Постановка квазистатической задачи линейной теории упругости.}

Рассмотрим постановку квазистатической (можно пренебречь
$ \dfrac{\partial \mathbf{v}}{\partial t} $ по сравнению с $\nabla \cdot \sigma$ и
$\mathring{\rho} \mathbf{f}$) задачи линейной теории упругости при изотермическом процесса
($\theta = \theta_0 = \operatorname{const}$), т.е. тогда, когда $q_m = 0$ -- отсутствие массовых
источников, $\theta_e = \theta_0$ -- внешняя температура равна внутренней,
$\mathbf{q}_{en} = \mathbf{0}$ -- нормальные потоки тепла равны нулю. Тогда получаем уравнение,
также называемое \emph{условием равновесия твёрдого тела}:
\[
  \nabla \cdot \sigma + \mathring{\rho} \mathbf{f} = 0, \\
\]

Если задача не является изотермической, то необходимо также рассматривать задачу теплопроводности,
которая будет содержать $\sigma$, а значит будет связанной с уравнением выше, такую задачу
называют \emph{связанной квазистатической задачей линейной термоупругости}.

Обычно в квазистатических задачах линейной термоупругости считают известным поле температуры
$\theta(\mathbf{x}, t)$, тогда
\[
  \nabla \cdot \sigma =
  \nabla \cdot ( {}^4 C \cdot\cdot \varepsilon - {}^4 C \cdot\cdot \mathring{\varepsilon} ) =
  \nabla \cdot ( {}^4 C \cdot\cdot \varepsilon - {}^4 C \cdot\cdot \alpha (\theta-\theta_0) )
\]
и второе слагаемое удобно включить в плотность внешних массовых сил:
\[
  \mathbf{f}' = \mathbf{f} - \dfrac{1}{\mathring{\rho}} \nabla \cdot (\beta \nu),
  \quad
  \mathbf{t}_{ne}' = \mathbf{t}_{ne} + \mathbf{n} \cdot \beta \nu,
\]
где $\beta = {}^4 C \cdot\cdot \alpha$, $\nu = \theta - \theta_0$; $\mathbf{f}'$ называют
плотностью обобщённых массовых сил, $\mathbf{t}_{ne}'$ -- обобщённым вектором поверхностных усилий.
Тогда квазистатическая задача линейной теории упругости принимает вид:
\[
  \begin{cases}
    \nabla \cdot \left( {}^4 C \cdot\cdot \nabla \otimes \mathbf{u} \right) + \mathring{\rho} \mathbf{f}' = 0, \\
    \mathbf{n} \cdot ({}^4 C \cdot\cdot \nabla \otimes \mathbf{u}) |_{\Sigma_\sigma} = \mathbf{t}_{ne}', \\
    \mathbf{u} |_{\Sigma_u} = \mathbf{u}_e.
  \end{cases}
  
\]

