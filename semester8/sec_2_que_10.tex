\que{Cоотношения на поверхности идеального контакта.}

Если поверхность $S$ является \textit{поверхностью контакта} (т.е. определяющие соотношения различны по обе стороны от $S$ и $\mathring{M} = 0$), тогда имеем:
\begin{equation}
	\begin{cases}
		[\mathring{\rho}] \not= 0, \\
		\mathbf{n} \cdot [\mathbf{\sigma}] + \mathring{C}_{2\Sigma} = 0, \\
		- \mathbf{n} \cdot [\mathbf{q}] + \mathbf{n} \cdot [\mathbf{\sigma} \cdot \mathbf{v}] + \mathring{C}_{3\Sigma} = 0, \\
		n \otimes [\mathring{\rho} \mathbf{v}] + \mathring{C}_{6\Sigma} = 0.
	\end{cases}
\end{equation}

В частности, если поверхность гомотермическая, когерентная ($\mathring{C}_{6\Sigma} = 0$) и поверхностными эффектами можно принебречь (т.е. $\mathring{C}_{2\Sigma} = 0$, $\mathring{C}_{3\Sigma} = 0$), то получим:
\begin{equation}
	\begin{cases}
		\mathbf{n} \cdot [\sigma] = 0, \\
		\mathbf{n} \cdot [\mathbf{q}] = 0, \\
		[\mathbf{u}] = 0, \\
		[\theta] = 0.
	\end{cases}
\end{equation}

Условие непрерывности вектора перемещений является следствием допущения о когерентности поверхности $S$.

Отметим, что поскольку для рассматриваемого случая поверхность $S$ является неподвижной ($\mathring{D} = 0$) и в рамках малых деформаций $S$ может изменить положение только из-за собственного движения в конфигурации $\mathring{\mathcal{K}}$, но не за счет перемещения материальных точек поверхности, тогда:
\begin{equation}
	[\mathbf{v}] = 0.
\end{equation} 

Это условие можно присоединить к последней системе вместо последного уравнения в ней. 

Такие \textit{условия называют условиями идеального контакта} двух твердых тел с малыми деформациями. 