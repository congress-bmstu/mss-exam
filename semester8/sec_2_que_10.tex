\que{Cоотношения на поверхности идеального контакта.}


$\mathring{M} = \mathring{D} \mathring{\rho}_{+}$ --- массовая скорость движения поверхности разрыва в $\mathring{\mathcal{K}}$

\begin{equation*}
    \Rightarrow \mathring{D} \rho_{+} = \mathring{D} \rho_{-} = \mathring{M}, \rho_{+} \Bigl(D - \vec{v}_{\Sigma_{+}} \cdot \vec{n}\Bigr) = \rho_{-} \Bigl(D - \vec{v}_{\Sigma_{-}} \cdot \vec{n}\Bigr) = M \text{ (в $\mathcal{K}$)}.
\end{equation*}

\begin{equation*}
    M = \frac{\kappa_{+}\rho_{+}}{\mathring{\rho}_{+}} \mathring{M}; \quad \kappa_{\pm} = \sqrt{\vec{n} \cdot g^{-1}_{\pm} \cdot \vec{n}} = \sqrt{\vec{n} \cdot F_{\pm} \cdot F^{T}_{\pm} \cdot \vec{n}}; \quad \Bigl[\frac{\kappa_{\rho}}{\beta}\Bigr] = 0; \quad \Bigl[\frac{\rho}{\mathring{\kappa}\mathring{\rho}}\Bigr] = 0.
\end{equation*}

Имеем соотношение: $M[A_{\alpha}] + \vec{n} \cdot [B_{\alpha}] + C_{\alpha} = 0, \, \alpha = \overline{2, 6}$. Материальные точки не проходят через поверхность разрыва, т.е. $\mathring{M} = M = 0$, то на $\mathring{S}$ и $S$ имеют место соотношения:
\begin{align*}
    &\mathring{\vec{n}} \cdot [\mathring{B_{\alpha}}] + \mathring{C}_{\alpha \Sigma} = 0, \, \mathring{\vec{x}} \in \mathring{S}; \\
    &\vec{n} \cdot [B_{\alpha}] + C_{\alpha \Sigma} = 0, \, x \in S, \alpha = \overline{2, 6}.
\end{align*}

или

\begin{equation*}
    \begin{cases}
        \mathring{\vec{n}} \cdot [P] + \mathring{C}_{2\Sigma} = 0, \\
        \mathring{\vec{n}} \cdot [P \cdot \vec{v} - \mathring{\vec{q}}] + \mathring{C}_{3\Sigma} = 0,
        \mathring{\vec{n}} \cdot [\vec{q} / \theta] + \mathring{C}_{4\Sigma} = 0, \\
        \mathring{C}_{5\Sigma} = 0,
        \vec{n} \otimes [\mathring{\rho} \vec{v}] + \mathring{C}_{6\Sigma} = 0, \, \mathring{\vec{x}} \in \mathring{S}.
    \end{cases}
\end{equation*}

Аналогично для $\vec{x} \in S$.

