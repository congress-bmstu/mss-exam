\que{Применение интеграла Бернулли для адиабатических процессов в совершенном газе. Изэнтропические формулы, критическая скорость.}

Пренебрежём массовыми силами $\chi = 0$, рассмотрим модель адиабатических процессов в совершенном
газе, тогда интеграл Бернулли вдоль некоторой линии тока $\mathcal{L}$ принимает вид:
\[
  \dfrac{v^2}{2} + c_p \theta = i^*.
\]

Пусть на линии тока имеется \emph{точка торможения}, т.е. такая точка, в которой скорость 
$\mathbf{v} = 0$. Тогда $i^* = c_p \theta^*$ -- \emph{энтальпия торможения}. В таком случае
можно найти скорость
\[
  v = \sqrt{2 c_p (\theta^* - \theta)},
\]
тогда можно сделать следующие выводы:
\begin{enumerate}
  \item температура на линии тока не может быть больше температуры в точке торможения;
  \item модуль скорости жидкости ограничен: $v \leqslant v_{\max} = \sqrt{2 c_p \theta^*}$.
\end{enumerate}

Тогда перепишем интеграл Бернулли в виде
\[
  \dfrac{v^2}{2} + \mathcal{P} = \dfrac{v_{\max}^2}{2}.
\]

При построении функции давления выберем в качестве точки $\mathcal{M}_1$ точку торможения,
т.е. в формулах для функции давления $\rho_1 = \rho^*, p_1 = p^*, \theta_1 = \theta^*$:
\begin{align*}
  \mathcal{P} &= c_p \theta^* \left( \dfrac{p}{p^*} \right)^{(k-1) / k} = \dfrac{v_{\max}}{2} \left( \dfrac{p}{p^*} \right)^{(k-1)/k}, \\
  \mathcal{P} &= c_p \theta^* \left( \dfrac{\rho}{\rho^*} \right)^{k-1} = \dfrac{v_{\max}}{2} \left( \dfrac{\rho}{\rho^*} \right)^{k-1}, \\
  \mathcal{P} &= c_p \theta.
\end{align*}

Подставляя эти формулы в интеграл Бернулли, получаем следующие соотношения:
\begin{align*}
  \dfrac{p}{p^*} &= \left( 1 - \dfrac{v^2}{v_{\max}^2} \right)^{k / (k-1)}, \\
  \dfrac{\rho}{\rho^*} &= \left( 1 - \dfrac{v^2}{v_{\max}^2} \right)^{1 / (k-1)}, \\
  \dfrac{\theta}{\theta^*} &= 1 - \dfrac{v^2}{v_{\max}^2}.
\end{align*}

Введём понятие \emph{число Маха} $M = \dfrac{v}{a}$, где $a$ -- скорость звука.
По определению, скорость звука вычисляется по формуле
$a = \sqrt{ \dfrac{\partial p}{\partial \rho} |_{\eta} }$, где подпись $\eta$ означает, что
производная берёться вдоль адиабаты Пуассона, т.е. от функции, связывающей $p$ и $\rho$.
Для совершенного газа скорость звука: $a^2 = k R \theta = c_p (k-1) \theta = \dfrac{kp}{\rho}$.

Используя выражение для скорости звука запишем интеграл Бернулли в виде
\[
  \dfrac{v^2}{2} + c_p \theta = \dfrac{v_{\max}^2}{2} \Rightarrow
  v^2 + \dfrac{2}{k-1} a^2 = v_{\max}^2
\]
тогда
\[
  \left( \dfrac{v}{v_{\max}} \right)^2 = \dfrac{M^2 (k-1)/2}{1 + M^2 (k-1)/2}
  \Rightarrow
  1 - \left( \dfrac{v}{v_{\max}} \right)^2 = \left( 1 + M^2 (k-1)/2 \right)^{-1}
\]

Тогда перепишем полученные формулы в виде:
\begin{align*}
  \dfrac{p}{p^*} &= \left( 1 + \dfrac{k-1}{2} M^2 \right)^{- k / (k-1)}, \\
  \dfrac{\rho}{\rho^*} &= \left( 1 + \dfrac{k-1}{2} M^2 \right)^{-1 / (k-1)}, \\
  \dfrac{\theta}{\theta^*} &= \left( 1 + \dfrac{k-1}{2} M^2 \right)^{-1}.
\end{align*}
эти выражения называют \emph{изэнтропическими формулами}.

Движение газа называют дозвуковым, если $M < 1$, звуковым, если $M = 1$,
сверхзвуковым, если $M > 1$, гиперзвуковым, если $M > 3$.

Если в выражение $a^2 = \dfrac{kp}{\rho}$ подставить выражения для $p / p^*$ и $\rho / \rho^*$,
то можно получить следующую формулу для скорости звука:
\[
  \dfrac{a}{a^*} = \left( 1 - \dfrac{v^2}{v_{\max}^2} \right)^{1/2}
\]
где $a^* = \sqrt{ k\dfrac{p^*}{\rho^*} } = \sqrt{k R \theta^*} = \sqrt{ \dfrac{k-1}{2} v_{\max} }$
-- скорость звука в точке торможения.

Найдём такую скорость, при которой движение будет звуковым ($M = 1$), такую скорость называют 
\emph{критической скоростью} $v_\text{кр}$:
\[
  v_\text{кр} = a(v_\text{кр})
  \Rightarrow
  v_\text{кр} = a^* \left( 1 - \dfrac{v_\text{кр}^2}{v_{\max}^2} \right)^{1/2}
  \Rightarrow
  v_\text{кр}^2 = \left( \dfrac{1}{v_{\max}^2} + \dfrac{1}{a^{*2}} \right)^{-1}
\]
используя соотношение $a^* = \sqrt{ \dfrac{k-1}{2} v_{\max} }$ можно получить
\[
  v_\text{кр}^2 = a_\text{кр}^2 = \dfrac{k-1}{k+1} v_{\max}^2.
\]
