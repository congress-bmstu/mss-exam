\que{Определение квазистатических процессов в жидкостях. Закон Паскаля. Задача о равновесии жидкости в поле силы тяжести.}

\begin{definition}
  Если в системе уравнений МЖГ для идеальной сжимаемой жидкости или в системе для несжимаемой
  жидкости пренебрегают членами, содержащими скорость $\mathbf{v}$, т.е. полагают
  $\mathbf{v} \equiv 0$, то говорят, что принята модель \emph{квазистатических процессов в
  жидкости}.
\end{definition}

В таком случае идеальная сжимаемая жидкость удовлетворяет уравнениям
\[
  \begin{cases}
    \dfrac{\partial \rho}{\partial t} = 0, \\
    \nabla p = \rho \mathbf{f}, \\
    \rho \dfrac{\partial e}{\partial t} + \nabla \cdot \mathbf{q} = \rho q_m.
  \end{cases}
\]
Второе уравнение этой системы называют \emph{уравнением равновесия жидкости}.

\paragraph{Закон Паскаля.} Если внешние массовые силы отсутствуют, то при квазистатических
процессах уравнение равновесия принимает вид $\nabla p = 0$, то есть давление $p = p(t)$ --
не зависит от координат и одинаково для всех точек в данный момент времени.
Это называют \emph{законом Паскаля}.

\paragraph{Задача о равновесии жидкости в поле силы тяжести.} 
Рассмотрим следующую задачу: пусть жидкость находится в равновесии в поле силы тяжести
$\mathbf{f}$.
Так как рассматриваем небольшой участок планеты, то будем считать ускорение свободного падения
постоянным и равным $g_\Sigma$ (для Земли оно примерно равно $9.8 \text{м}/\text{с}^2$). 
Декартову систему координат введём таким образом, чтобы ось $\bar{\mathbf{e}}_3$ была направлена
вверх, соответственно в такой системе координат $\mathbf{f} = - g_\Sigma \bar{\mathbf{e}}_e$,
тогда условие равновесия можно расписать покомпонентно:
\[
  \begin{cases}
    \dfrac{\partial p}{\partial x^1} = \dfrac{\partial p}{\partial x^2} = 0, \\
    \dfrac{\partial p}{\partial x^3} = - \rho g_\Sigma = - \dfrac{p g_\Sigma}{R\theta},
  \end{cases}
\]
(здесь использовано соотношение Менделеева-Клапейрона $\rho = p/(R\theta)$).

Решением этого уравнения будет являтся так называемая \emph{барометрическая формула}:
\[
  p = p_0 \exp \left( - \int\limits_{x^3_0}^{x^3} \dfrac{g_\Sigma \, dx}{R\theta} \right),
\]
где $p_0$ -- давление при $x^3 = x^3_0$.

Третье уравнение системы для квазистатических процессов в совершенном газе перепишем в виде:
\[
  \rho c_v \dfrac{\partial \theta}{\partial t} = \lambda \Delta \theta + \rho q_m,
\]
которое называется \emph{уравнением теплопроводности для совершенного газа в состоянии
равновесия}.

Если процесс распространения тепла в газе является стационарным (установившимся), т.е. 
$\partial \theta / \partial t = 0$, то получаем стационарное уравнение теплопроводности:
\[
  \Delta \theta + \dfrac{pq_m}{\lambda R \theta} = 0.
\]


