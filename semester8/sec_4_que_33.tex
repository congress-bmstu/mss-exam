\que{Соотношения на поверхности разрыва в идеальном газе без перехода материальных точек через поверхность}

Соотношения Гюгонио были установлены для случая $M \not = 0$, т.е. при наличии перехода материальных точек через поверхность $S(t)$ (случай ударных волн и фазовых превращений). 

Если материальные точки не переходят через поверхность $S(t)$, то $M = 0$ и из соотношений из теоремы из предыдущего вопроса \ref{3} следует, что:
\begin{equation*}
	-M = \rho_1 u_1 = \rho_1 \left(v_{n_1} - D\right) = \rho_2 \left(v_{n_2} - D\right) = 0.
\end{equation*}

Поскольку $\rho_1 \not = 0$ и $\rho_2 \not = 0$, то нормальные составляюзие скорости при переходе через $S$ являются непрерывными:
\begin{equation*}
	M = 0, \quad v_{n_1} = v_{n_2} = D, \quad u_1 = u_2 = 0.
\end{equation*}

Тогда находим:
\begin{equation*}
	p_1 - p_2 = C_{n\Sigma}.
\end{equation*}

Если поверхностные усилия отсутствуют: $C_{n\Sigma} = 0$, то давление при переходе через $S(t)$ также остается непрерывными:
\begin{equation*}
	p_1 = p_2.
\end{equation*}

Третье соотношение, все той же системы, в случае $M = 0$ сводится к простому условию:
\begin{equation*}
	C'_{3\Sigma} - C_{n\Sigma} D = 0,
\end{equation*}
или
\begin{equation*}
	q_{n_1} - q_{n_2} = C_{3\Sigma} - D C_{n\Sigma}.
\end{equation*}

Если поверхностные усилия и энергия отсутствуют: $C_{3\Sigma} = -. C_{n\Sigma} = 0$, то получаем условие непрерывности нормальной составляющей теплового потока:
\begin{equation*}
	q_{n_1} = q_{n_2}.
\end{equation*}
Таким образом, доказана следующая теорема.

\begin{theorem}
	В случае отсутствия перехода материальных точек черех поверхность разрыва $S(t)$ в идеальном газе давление $p$, нормальные составляющие скорости $v_n$ и теплового потока $q_n$ остаются непрерывными при переходе через $S$, остальные же функции $v_{\tau_\alpha}$, $e$, $\theta$ и $\rho$ могут терпеть разрыв. 
\end{theorem}