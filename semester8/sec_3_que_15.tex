\que{Безиндексная  и  компонентная  формы  представления  определяющих  соотношений  линейной  теории упругости.}

Основное определяющее соотношение линейной теории упругости это обобщённый закон Гука
\[
  \sigma = {}^4 C \cdot\cdot \varepsilon
\]
-- безиндексная форма представления.

Пусть базис $\mathbf{c}_\alpha$ -- ортонормированный базис главных осей анизотропии, тогда в этом базисе
$\sigma = \sigma^{ij} \mathbf{c}_i \otimes \mathbf{c}_j$,
$\varepsilon = \varepsilon^{ij} \mathbf{c}_i \otimes \mathbf{c}_j$, 
${}^4 C = C^{ijkl} \mathbf{c}_i \otimes \mathbf{c}_j \otimes \mathbf{c}_k \otimes \mathbf{c}_l$, и обобщённый закон Гука записывается в компонентной форме:
\[
  \sigma^{ij} = C^{ijkl} \varepsilon^{kl}.
\]

Причём в этом базисе $\sigma^{ij} = \sigma^{ji}$, $\varepsilon^{ij} = \varepsilon^{ji}$,
тогда и $C^{ijkl} = C^{jikl} = C^{ijlk} = C^{jilk}$. Также, т.к. ${}^4 C$ образует
квадратичную форму в выражении для $\psi$ (см. определение линейно-упругой среды), то
$C^{ijkl} = C^{klij}$.
