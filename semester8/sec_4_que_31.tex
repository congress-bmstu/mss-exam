\que{Модель идеальной несжимаемой жидкости.}

\begin{definition}
  Сплошную среду называют \emph{несжимаемой}, если в любой актуальной конфигурации $\mathcal{K}$
  её плотность $\rho$ совпадает с плотностью $\mathring{\rho}$ в отсчётной конфигурации:
  \[
    \rho = \mathring{\rho} = \operatorname{const}, \forall t \geqslant 0, \forall \mathcal{K}
  \]
\end{definition}

Из уравнения неразрывности в переменных Лагранжа следует, что для несжимаемых сред объём любого
элементарного параллелепипеда $dV$ не изменяется:
\[
  \rho = \mathring{\rho} \Rightarrow dV = d\mathring{V},
\]
а, следовательно, и объём $|V(t)|$ области, которую занимает несжимаемая среда, не изменяется:
$|V(t)| = |\mathring{V}| = \operatorname{const}$.

Из уравнения неразрывности следует также, что для несжимаемой среды всегда существует
дополнительное условие на градиент деформации:
\[
  \det F = 1.
\]

Используя полярное разложение, находим, что для несжимаемой среды тензоры искажений всегда имеют 
единичный детерминант:
\[
  \det V = \det F \cdot \det O^T = \det F = 1, \quad \det U = 1.
\]
Тогда, находим, что детерминант всех энергетических и квазиэнергетических мер деформации тоже
всегда имеет постоянное значение:
\begin{align*}
  \det \overset{(n)}{G} &= \det \left( \dfrac{1}{n - III} U^{n - III} \right) = 
  \dfrac{1}{(n-III)^3} \det U^{n - III} = \dfrac{1}{(n-III)^3}, \\
  \det \overset{(n)}{g} &= \det \left( \dfrac{1}{n - III} V^{n - III} \right) = 
  \dfrac{1}{(n-III)^3}, \quad n = I, II, III, IV, V.
\end{align*}

Уравнение неразрывности для несжимаемой жидкости (т.н. \emph{уравнение несжимаемости}):
\[
  \nabla \cdot \mathbf{v} = 0.
\]
Уравнение движения для несжимаемой жидкости:
\[
  \dfrac{\partial \mathbf{v}}{\partial t} + \nabla \cdot \left( \mathbf{v} \otimes \mathbf{v} + \dfrac{p}{\rho} E \right) = \mathbf{f}.
\]
Уравнение энергии для несжимаемой жидкости:
\[
  \dfrac{\partial \varepsilon}{\partial t} + \nabla \cdot \left( \mathbf{v}\cdot \left( \varepsilon+\dfrac{p}{\rho} \right) + \dfrac{\mathbf{q}}{\rho} \right) = \mathbf{f}\cdot\mathbf{v} + q_m.
\]

Определяющие соотношения для несжимаемой жидкости:
\[
  \begin{cases}
    \varepsilon = e + \dfrac{|\mathbf{v}|^2}{2}, \\
    e = \psi + \theta \eta = \psi - \theta \dfrac{\partial \psi}{\partial \theta} = e(\theta), \\
    \psi = \psi(\rho, \theta) = \psi(\theta), \\
    \mathbf{q} = - \lambda \nabla \theta
  \end{cases}
\]
Определяющее соотношение $p = \rho^2 \dfrac{\partial \psi}{\partial \rho}$ пропадает, а
давление становится самостоятельной независимой функцией.

% \paragraph{Модель совершенной несжимаемой жидкости.}

