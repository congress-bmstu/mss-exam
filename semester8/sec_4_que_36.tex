\que{Адиабата Пуассона для совершенного газа.}

\paragraph{Адиабата Пуассона для совершенного газа.}
В выражении для $\eta$ в совершенном газе обозначим:
\[
  \eta\left(\rho(X^i, t), \theta(X^i, t)\right) =
  - \dfrac{\partial \psi}{\partial \theta}  =
  \eta_0 +
  % \int\limits_{\theta_0}^\theta C_v(\tilde\theta) \, d\tilde\theta +
  \underbrace{\int\limits_{\theta_0} \dfrac{c_v(\tilde \theta)}{\tilde \theta} \, d\tilde \theta + R \ln \dfrac{\rho_0}{\rho}}_{\Delta \eta} =
  \eta_0 + \Delta \eta =
  \mathring{\eta}(X^i)
\]
т.е.
\[
  \int\limits_{\theta_0} \dfrac{c_v(\tilde \theta)}{\tilde \theta} \, d\tilde \theta + R \ln \dfrac{\rho_0}{\rho} =
  \eta - \Delta \eta =
  \operatorname{const}
\]

\paragraph{Для модели совершенного газа с постоянной теплоёмкостью.}
Если теперь $c_v = \operatorname{const}$, то интегралы в последнем выражении можно найти
и получить
\[
  c_v \ln \dfrac{\theta}{\theta_0} = R \ln \dfrac{\rho}{\rho_0} + \Delta \eta,
\]
или
\[
  \dfrac{\theta}{\theta_0} = e^{\dfrac{\Delta \eta}{c_v}} \left(\dfrac{\rho}{\rho_0}\right)^{\dfrac{R}{c_v}}
\]
обозначая в последнем $A_0 = e^{\dfrac{\Delta\eta}{c_v}}$; вспоминая коэффициент Пуассона $k = \dfrac{c_p}{c_v} = \dfrac{c_v + R}{c_v}$, получаем
\begin{equation}\label{eq:poisson_adiabata_0}
  \dfrac{\theta}{\theta_0} = A_0 \left( \dfrac{\rho}{\rho_0} \right)^{k-1}
\end{equation}
-- адиабата Пуассона для идеального совершенного газа с постоянными теплоёмкостями
в форме $\theta \sim \rho$.

\textit{Другие формы адиабаты Пуассона для совершенного газа с постоянными теплоёмкостями.}

Отметим, что адиабата Пуассона \eqref{eq:poisson_adiabata_0} содержит параметры газа
$\theta_0, \rho_0, \eta_0$ -- табличные значения в некотором известном (<<начальном>>) состоянии.
Получим теперь это же соотношение, считая известными параметры газа в некоторой конфигурации
$\mathcal{K}_1$, в которой газ имеет параметры $\theta_1, \rho_1, \eta_1$. Запишем адиабату
Пуассона два раза: для известного состояния $\mathcal{K}_1$ и для некоторой произвольной
конфигурации $\mathcal{K}$:
\[
  \begin{cases}
    \dfrac{\theta_1}{\theta_0} = A_0 \left( \dfrac{\rho_1}{\rho_0} \right)^{k-1} \\
    \dfrac{\theta}{\theta_0} = A_0 \left( \dfrac{\rho}{\rho_0} \right)^{k-1}
  \end{cases}
\]
поделим второе выражение на первое и получим
\begin{equation}\label{eq:poisson_adiabata_1}
  \dfrac{\theta}{\theta_1} = \left( \dfrac{\rho}{\rho_1} \right)^{k-1}
\end{equation}
-- адиабата Пуассона для совершенного газа с постоянными теплоёмкостями, выраженная через
некоторую конфигурацию $\mathcal{K}_1$.

\textit{Адиабата Пуассона в форме $p \sim \rho$}
Из соотношения Менделеева-Клапейрона $p = \rho R \theta$ выразим температуру $\theta = \dfrac{p}{\rho R}$, тогда адиабата Пуассона \eqref{eq:poisson_adiabata_1} принимает вид:
\[
  \dfrac{p \rho_1 R}{p_1 \rho R} = \left(\dfrac{\rho}{\rho_1} \right)^{k-1},
\]
сокращая всё что можно сократить, получаем
\begin{equation}\label{eq:poisson_adiabata_2}
  \dfrac{p}{p_1} = \left( \dfrac{\rho}{\rho_1} \right)^{k}
\end{equation}
-- ещё одна адиабата Пуассона.

Её ещё можно преобразовать, введя $A = \dfrac{p_1}{\rho_1^k}$, к виду
\begin{equation}\label{eq:poisson_adiabata_3}
  p = A \rho^k
\end{equation}

Ещё можно ввести \emph{удельный объём} $V = \dfrac{1}{\rho}$, тогда
\begin{equation}\label{eq:poisson_adiabata_4}
  \dfrac{p}{p_1} = \left( \dfrac{V_1}{V} \right)^{k}
\end{equation}

\paragraph{Внутренняя энергия совершенного идеального газа с постоянными теплоёмкостями.}
Для совершенного газа было получено:
\[
  e = e_0 + c_v(\theta - \theta_0) = \tilde e_0 + c_v \theta
\]
используя соотношение Менделеева-Клапейрона в виде $\theta = \dfrac{p}{\rho R}$ получаем:
\[
  e = \tilde e_0 + \dfrac{c_v p}{\rho R} = \tilde e_0 + \dfrac{p}{\rho (k-1)} =
  \tilde e_0 + \dfrac{pV}{k-1}
\]
