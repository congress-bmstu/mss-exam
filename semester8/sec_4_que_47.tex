\que{Установившиеся процессы в идеальных газах. Функция давления, выражения для нее при адиабатических процессах в совершенном газе (3 формы). Функция давления для несжимаемой жидкости.}

\begin{definition}
  Если для идеальной жидкости (как сжимаемой, так и несжимаемой) принимают допущение, 
  что функции $\mathbf{v}, p, \rho, \theta$ зависят только от координат, но не от времени, то
  говорят, что принята модель \emph{установившихся} (или \emph{стационарных}) процессов в
  жидкости.
\end{definition}

Для сжимаемой жидкости система уравнений МЖГ при установившихся процессах принимает вид:
\[
  \begin{cases}
    \nabla \cdot (\rho \mathbf{v}) = 0, \\
    \nabla \cdot \left( \rho \mathbf{v} \otimes \mathbf{v} + p E \right) = \rho \mathbf{f}, \\
    \nabla \cdot \left( \rho \mathbf{v} (\varepsilon + \dfrac{p}{\rho}) + \mathbf{q} \right) = \rho \mathbf{f} \cdot \mathbf{v} + \rho q_m.
  \end{cases}
\]

Для несжимаемой:
\[
  \begin{cases}
    \nabla \cdot \mathbf{v} = 0, \\
    \nabla \cdot \left( \mathbf{v} \otimes \mathbf{v} + \dfrac{p}{\mathring{\rho}} E \right) = \mathbf{f}, \\
    \nabla \cdot \left( \mathbf{v} \left( \varepsilon + \dfrac{p}{\mathring{\rho}} \right) + \mathbf{q} \right) = \mathbf{f} \cdot \mathbf{v} + q_m.
  \end{cases}
\]

Эти системы допускают существования первого интеграла, если ввести т.н. \emph{функцию давления}.

\paragraph{Функция давления.}
Пусть есть некоторая несамопересекающаяся кривая $\mathcal{L}$, которая проходит через
точку $\mathcal{M}_1 ( x^i_1 )$ и имеет уравнение
$x^i = x^i_\mathcal{L} (x^j_1, \tau), \tau \in [\tau_1, \tau_2]$. Тогда в установившемся
процессе, когда $\rho$ и $p$ являются функциями только координат, можно найти их как
зависимости от $\tau$: $\rho = \rho(x^i(x^j_1, \tau)) = \rho(\tau, \mathcal{L})$,
$p = p(x^i(x^j_1, \tau)) = p(\tau, \mathcal{L})$, тогда если эти зависимости взаимнооднозначные,
то можно построить зависимость $\rho = \rho(p, \mathcal{L})$.

Введём \emph{функцию давления} $\mathcal{P} (p, \mathcal{L})$:
\[
  \mathcal{P} (p, \mathcal{L}) = \int\limits^p \dfrac{dp'}{\rho(p', \mathcal{L})}
  \Leftrightarrow
  \dfrac{d\mathcal{P}}{dp} = \dfrac{1}{\rho(p, \mathcal{L})},
\]
где нижний предел не важен.

Можно доказать следующую теорему:
\begin{theorem}\footnote{эта теорема не до конца сформулирована в третьем томе Димитриенко}
  В баротропной в шикором смысле жидкости при установившихся процессах 
  функция давления не зависит от кривой $\mathcal{L}$, а только от значения давления
  и плотности в точке $\mathcal{M}_1$:
  \[
    \mathcal{P}(p, \mathcal{L}) = \mathcal{P}(p, \rho_1, p_1) = \int\limits^p \dfrac{dp'}{\rho(p, \rho_1, p_1)}
  \]
\end{theorem}

\paragraph{Функция давления при абиабатических процессах в совершенном газе.}
При адиабатических процессах в совершенном газе с постоянными теплоёмкостями (которые
являются баротропными в широком смысле) функция давления может быть найдена:
\[
  \mathcal{P}(p, \rho_1, p_1) = \int\limtis^p \left( \dfrac{p'}{p_1} \right)^{- 1/k} \dfrac{dp'}{\rho_1} = \dfrac{p_1}{\rho_1} \dfrac{k}{k-1} \left( \dfrac{p}{p_1} \right)^{(k-1)/k} =
  c_p \theta_1 \left( \dfrac{p}{p_1} \right)^{(k-1)/k}
\]

Другие формы для функции давления при адиабатических процессах в совершенном газе:
воспользуемся соотношением Мендлеева-Клапейрона $p / p_1 = \left( \rho / \rho_1 \right)^k $,
тогда:
\[
  \mathcal{P} = \dfrac{p_1}{\rho_1} \dfrac{k}{k-1} \left( \dfrac{\rho}{\rho_1} \right)^{k-1} =
  c_p \theta_1 \left( \dfrac{\rho}{\rho_1} \right)^{k-1},
\]
так как $p / p_1 = \left( \rho / \rho_1 \right)^k \Rightarrow \dfrac{p}{\rho} = p_1 \dfrac{\rho^{k-1}}{\rho_1^k}$, то можно получить ещё одну форму функции давления:
\[
  \mathcal{P} = \dfrac{k}{k-1} \dfrac{p}{\rho} = c_p \theta.
\]
