\que{Соотношения на поверхностях сильных разрывов в пространственном описании.}

Установим теперь соотношения для скачков функций $\rho_\alpha, A_\alpha, B_\alpha$ и $C_\alpha$ на когерентной поверхности разрыва $S(t)$, соответствующей поверхности разрыва $\mathring{S}(t)$ в $\mathring{\mathcal{K}}$. Для этого умножим соотношение (\ref{eq125}) на элементарную площадку $d\mathring{\Sigma}$:
\begin{equation}\label{eq21}
-[\mathring{\rho}\mathring{A}_\alpha]\mathring{\mathbf{c}}\cdot\mathring{\mathbf{n}}d\mathring{\Sigma}=\mathring{\mathbf{n}}\cdot[\mathring{B}_\alpha]d\mathring{\Sigma}+\mathring{C}_{\alpha\Sigma}d\mathring{\Sigma}
\end{equation}

\begin{theorem}
	Для функций $\mathring{B}_\alpha$ и $B_\alpha$ имеют место соотношения:
	\begin{equation}
		\mathring{\mathbf{n}}\cdot[\mathring{B}_\alpha]d\mathring{\Sigma}=\mathbf{n}\cdot[B_\alpha]d\Sigma,~~~\alpha=1,\dots,6.
	\end{equation}
\end{theorem}
\paragraph{Доказательство.}
Действительно, при $\alpha=2$ имеем: $B_2=\mathbf{T}, \mathring{B}_2=\mathbf{P}$, и поэтому:
\[
	\mathring{\mathbf{n}}\cdot[\mathbf{P}]d\mathring{\Sigma}=\mathbf{n}\cdot[\mathbf{T}]d\Sigma
\]

При $\alpha=3$ имеем $B_3=\mathbf{T}\cdot\mathbf{v}-\mathbf{q}, \mathring{B}_3=\mathbf{P}\cdot\mathbf{v}-\mathring{\mathbf{q}}$, и, следовательно:
\[
\mathring{\mathbf{n}}\cdot[\mathbf{P}\cdot\mathbf{v}-\mathring{\mathbf{q}}]d\mathring{\Sigma}=\mathbf{n}\cdot[\mathbf{T}\cdot\mathbf{v}-\mathbf{q}]d\Sigma.
\]

При $\alpha=4$ имеем $B_4=\mathbf{q}/\theta,\mathring{B}_4=\mathring{\mathbf{q}}/\theta$, поэтому выполняется соотношение:
\[
\mathring{\mathbf{n}}\cdot[\mathring{\mathbf{q}}/\theta]d\mathring{\Sigma}=\mathbf{n}\cdot[\mathbf{q}/\theta]d\Sigma.
\]

При $\alpha=6$ имеем $B_6=\rho\mathbf{F}\otimes\mathbf{v},\mathring{B}_6=\mathring{\rho}\mathbf{E}\otimes\mathbf{v}$, получаем:
\[
\mathbf{n}\cdot[\rho\mathbf{F}\otimes\mathbf{v}]d\Sigma = \mathring{\mathbf{n}}\cdot[\rho\sqrt{g/\mathring{g}}\mathbf{E}\otimes\mathbf{v}]d\mathring{\Sigma}=\mathring{\mathbf{n}}\cdot[\mathring{\rho}\mathbf{E}\otimes\mathbf{v}]d\mathring{\Sigma}
\]
что и требовалось доказать. $\blacksquare$

\begin{theorem}
	Скорости $\mathring{\mathbf{c}}$ и $\mathbf{c}$ движения когерентной поверхности разрыва
	$S(t)$ в отсчетной и актуальной конфигурациях, определенные по $\mathbf{\mathring{c}}=d\mathring{\mathbf{x}}_\Sigma/dt$ и $\mathbf{c}=d\mathbf{x}_\Sigma/dt$, связаны следующими соотношениями:
	\begin{equation}
		\mathbf{c}=\mathbf{v}_{\Sigma\pm}+\mathbf{F}_\pm\cdot \mathring{\mathbf{c}},
	\end{equation}
	или
	\begin{equation}\label{eq131}
		\mathring{\mathbf{c}}=\mathbf{F}_\pm^{-1}\cdot(\mathbf{c}-\mathbf{v}_{\Sigma\pm}).
	\end{equation}
\end{theorem}

Для того чтобы преобразовать левую часть в (\ref{eq21}) используем соотношение (\ref{eq131}), тогда имеют место следующие соотношения:
\[
\mathring{\rho}\mathring{D}d\mathring{\Sigma}=\mathring{\rho}_+\mathring{\mathbf{c}}\cdot\mathring{\mathbf{n}}d\mathring{\Sigma}=\mathring{\rho}_+\mathring{\mathbf{n}}\cdot\mathbf{F}_+^{-1}\cdot(\mathbf{c}-\mathbf{v}_{\Sigma+})d\mathring{\Sigma}=\mathring{\rho}_-\mathring{\mathbf{n}}\cdot\mathbf{F}_-^{-1}\cdot(\mathbf{c}-\mathbf{v}_{\Sigma-})d\mathring{\Sigma}.
\]

Но, учитывая свойство преобразования элементарных площадок,
примененное к площадкам на поверхности разрыва:
\[
\mathring{\rho}_\pm\mathring{\mathbf{n}}\cdot \mathbf{F}_\pm^{-1}d\mathring{\Sigma}=\rho_\pm\mathbf{n}d\Sigma
\]
получаем
\[
\mathring{\rho}_\pm\mathring{D}d\mathring{\Sigma}=\rho_\pm\mathbf{n}\cdot\mathbf{c}d\Sigma-\rho_\pm\mathbf{n}\cdot\mathbf{v}_{\Sigma\pm}d\Sigma=\rho_\pm(D-\mathbf{n}\cdot\mathbf{v}_{\Sigma\pm})d\Sigma
\]
где обозначена нормальная скорость $D$ движения поверхности раздела $\mathcal{K}$:
\[
D=\mathbf{c}\cdot\mathbf{n}.
\]

Подставляя в (\ref{eq21}), приходим к следующей теореме:
\begin{theorem}
	Соотношения (\ref{eq125}) на когерентной поверхности разрыва $\mathring{S}$ в $\mathring{\mathcal{K}}$ эквивалентны соотношениям
	\begin{equation}
		-[\rho A_\alpha]D+\mathbf{n}\cdot[\rho\mathbf{v}\otimes A_\alpha]-\mathbf{n}\cdot[B_\alpha]=C_{\alpha\Sigma}
	\end{equation}
\end{theorem}