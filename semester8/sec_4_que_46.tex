\que{Задача о равновесии стандартной атмосферы.}

Рассматривается задача о равновесии в поле силы тяжести на высотах
$0 \leqslant z = x^3 - x_0^3 \leqslant H$, как в прошлом вопросе,
в которой плотность внешних массовых сил $\mathbf{f} = - g_\Sigma \bar{\mathbf{e}}_3$,
при этом для нахождения давления предполагают, что температура равна некоторой константе
$\bar{\theta}$, тогда используя барометрическую формулу (см. предыдущий вопрос):
\[
  p = p_0 \exp \left( - \int\limits_0^z \dfrac{g_\Sigma \, dx}{R \bar{\theta}} \right) =
  p_0 \exp \left( - \beta_0 z \right), \quad \beta_0 = \dfrac{g_\Sigma}{R\bar{\theta}}, \quad
  \rho = \rho_0 \exp \left( - \beta_0 z \right).
\]

Теперь найдём стационарное распределение температуры от высоты. Для этого необходимо
определить чему равна плотность внешних массовых источников тепла $q_m$, которая
в реальности в основном определяется поглощением излучения атмосферы. 
Зададим эту плотность кусочно-постоянной, т.е. разобьём $[0, H]$ на подотрезки
$[H_i, H_{i+1}], i = 0, 1, \dots, n-1$, где $0 = H_0 < H_1 < \dots < H_n = H$.
На каждом таком подотрезке $q_m = q_{mi}, z \in [H_i, H_{i+1}]$. Найдём температуру из 
стационарного уранвнения теплопроводности:
\[
  \dfrac{d^2\theta}{dz^2} + \dfrac{\rho q_m}{\lambda} = 0
  \Leftrightarrow
  \dfrac{d^2 \theta}{dz^2} = - \dfrac{q_m \rho_0}{\lambda} \exp \left( - \beta_0 z \right),
\]
тогда решение будет представляться в виде:
\[
  \theta = \theta^{(i)} = - \dfrac{\rho_0 q_{mi}}{\lambda \beta_0^2} \exp \left( - \beta_0 z \right) + a_i z + b_i, \; z \in [H_i, H_{i+1}],
\]
а коэффициенты $a_i, b_i$ будут определяться из условий непрерывности $\theta$ и
$\partial \theta / \partial z$ в точках $H_i$, а также из граничных условий $\theta(0) = \theta_0$,
$\theta(H) = \theta_H$.
