% !TeX spellcheck = ru_RU
\que{Основные постановки задач в теории малых упругих деформаций: динамическая и квазистатическая задачи в перемещениях. Основные типы граничных условий.}

\subsection{Динамическая задача в перемещениях}

Если рассматривают только изотермические процессы, когда во вей области $V$ при всех $t\geqslant0$ температура остается постоянной ($\theta=\theta_0$), то динамическая постановка задачи МДТТ имеет вид:
\[
	\vec{x}\in\mathring{V},\;t>0:\quad \mathring{\rho} \frac{\partial \vec{v}}{\partial t} = \nabla \cdot \vec{\sigma} + \mathring{\rho}\vec{f},
\]
\begin{itemize}[label*=]
	\item[где] $\mathring{\rho}$ -- плотность в конфигурации $\mathring{\mathcal{K}}$,
	\item $\vec{\sigma}$ -- тензор напряжений,
	\item $\vec{f}$ -- плотность массовых сил,
	\item $\vec{t}_n$ -- вектор напряжений.
\end{itemize}

\subsection{Интегральная формулировка закона изменения количества движения}
\[
	\underbrace{\int\limits_{V} \mathring{\rho} \frac{\mathrm{d}\vec{v}}{\mathrm{d}t}\;\mathrm{d}V}_{\text{внутр.масс.силы(инерц.)}} =
	\underbrace{\int\limits_{\Sigma} \vec{n}\sigma\;\mathrm{d}\Sigma}_{\text{внеш.поверх.силы}} +
	\underbrace{\int\limits_{V}\mathring{\rho} f \;\mathrm{d}V}_{\text{внеш.масс.силы}}
\]

\[
\vec{x}\in\mathring{V}\cup\Sigma,\;t>0:\quad\begin{cases}
	\vec{v}=\frac{\partial\vec{u}}{\partial t}, &\\
	\varepsilon=\frac{1}{2}(\nabla\otimes\vec{u}+\nabla\otimes\vec{u}^T),&\text{-- соотношение Коши}\\
	\sigma(\varepsilon) = F(\varepsilon,\theta_0)=\rho\frac{\partial\psi(\varepsilon,\theta_0)}{\partial\varepsilon},&\text{-- определяющее соотношение}
\end{cases}
\]
Здесь $\sigma(\varepsilon)=\sigma(\varepsilon,\theta_0)$ - оператор, не зависящий от $\theta$.

Граничные условия:
\begin{align*}
	&x\in\Sigma_\sigma,\; t>0:\quad n\cdot\sigma=t_{ne},\\
	&x\in\Sigma_u,\; t>0:\quad u=u_e
\end{align*}

Начальные условия:
\[
	x\in V,t=0:\quad \begin{cases}
		u=u_0,\\
		v=v_0.
	\end{cases}
\]

\subsection{Квазистатическая задача в перемещениях}

Если в уравнении движения
\[
	\mathring{\rho}\frac{\partial\vec{v}}{\partial t}=\nabla\cdot\sigma+\mathring{\rho}\vec{f},\quad\vec{v}=\frac{\partial\vec{u}}{\partial t}\implies
	\mathring{\rho}\frac{\partial^2\vec{u}}{\partial t^2}=\nabla\cdot\sigma+\mathring{\rho}f
\]
можно пренебречь инерционным членом $\mathring{\rho}\frac{\partial\vec{v}}{\partial t}$ по сравнению с $\nabla\cdot\sigma$, то тогда рассматривается модель квазистатических процессов в деформируемом твердом теле. Для таких процессов постановка квазистатической задачи МДТТ в перемещениях имеет вид:
\[
\begin{cases}
	\nabla\cdot\sigma+\mathring{\rho}f = 0, &\text{-- уравнение равновесия твердого тела}\\
	\sigma(\varepsilon) = F(\varepsilon,\theta_0)=\rho\frac{\partial\psi(\varepsilon,\theta_0)}{\partial\varepsilon},&\text{-- определяющее соотношение}\\
	\varepsilon=\frac{1}{2}(\nabla\otimes\vec{u}+\nabla\otimes\vec{u}^T),&\text{-- соотношение Коши}\\
	\left. \vec{n}\sigma\right|_{\Sigma_\sigma} = \vec{t}_{ne}, &\text{-- граничное условие}\\
	\left. u\right|_{\Sigma_u} = u_e, &\text{-- граничное условие}\\
\end{cases}
\]

Начальные условия:
\[
x\in V,t=0:\quad \begin{cases}
	u=u_0,\\
	v=v_0.
\end{cases}
\]

\subsection{Основные типы граничных условий}

\begin{itemize}
	\item Условие 1-го рода:
	Если на $\Sigma$ задана температура $\theta_e$ или $u_e$;
	\item Условие 2-го рода:
	Если на $\Sigma$ задан вектор усилий $t_{ne}$ или тепловой поток $q_{ne}$;
	\item Условие 3-го рода:
	\begin{itemize}
		\item При конвективном теплообмене. Если на $\Sigma$ задан конвективный тепловой поток:
		\[
			q_{ne} = \alpha^T\left(\theta_e-\theta_w\right),
		\]
		\begin{itemize}[label*=]
			\item[где] $\alpha^T$ -- коэффициент теплообмена,
			\item $\theta_e$ -- температура внешней среды,
			\item $\theta_w$ -- температура на поверхности.
		\end{itemize}
		\item При теплообмене излучением
		\[
			-\vec{n}\cdot\lambda\cdot\nabla\theta=\alpha^T\left(\theta_e-\theta_w\right)-\varepsilon_\theta\sigma_{SB}\theta^4,
		\]
		\begin{itemize}[label*=]
			\item[где] $\lambda$ -- коэффициент теплопроводности,
			\item $\varepsilon_\theta$ -- степень черноты поверхности (интегральный коэффициент излучения),
			\item $\sigma_{SB}$ -- постоянная Стефана-Больцмана.
		\end{itemize}
	\end{itemize}
\end{itemize}