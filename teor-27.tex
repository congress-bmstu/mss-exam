\que{Незамкнутость системы законов сохранений МСС. Понятие об определяющих соотношениях. Принципы построения определяющих соотношений. Основное термодинамическое тождество.}
\footnote{Димитриенко Ю.И. -- Нелинейная механика сплошных сред, стр. 149}

\paragraph{Незамкнутость системы законов сохранений МСС.}
Полная система законов сохранения в полных дифференциалах:
\begin{equation*}
  \begin{cases}
    \rho \dfrac{d \bar{A}_\alpha}{dt} = \nabla \cdot \bar{B}_\alpha + \rho C_\alpha, 
    \alpha = 1..6 \\

    \bar{A}_\alpha = \begin{pmatrix}
      1/\rho \\ \mathbf{v} \\ e + |v|^2 / 2 \\ \eta \\ \mathbf{u} \\ \mathbf{F}^T
    \end{pmatrix}, \quad

    \bar{B}_\alpha = \begin{pmatrix}
      \mathbf{v} \\ \mathbf{T} \\ \mathbf{T} \cdot \mathbf{v} - \mathbf{q} \\ -\mathbf{q} / \theta \\
      \mathbf{0} \\ \rho \mathbf{F} \otimes \mathbf{v}
    \end{pmatrix}, \quad

    C_\alpha = \begin{pmatrix}
      0 \\ \mathbf{f} \\ \mathbf{f} \cdot \mathbf{v} + q_m \\ 
      (q_m + q^*) / \theta \\ 
      \mathbf{v} \\ \mathbf{0}
    \end{pmatrix} 
  \end{cases}
\end{equation*}
данная система содержит 18 скалярных уравнений и 29 скалярных неизвестных
$\rho, \mathbf{v}, \mathbf{u}, \mathbf{T}, e, \eta, \theta, \mathbf{q}, \mathbf{F}, q^*$.
Эта система одинакова для любой сплошной среды.

\paragraph{Определяющие соотношения.}
Для замыкания системы уравнений необходимы дополнительные соотношения. Эти дополнительные
соотношения называют \emph{определяющими соотношениями}, поскольку именно они определяют, чем одна
сплошная среда отличается от другой (универсальные законы сохранения «не различают» типы
сплошных сред — они одинаковы для всех тел). Если заданы каким-либо образом определяющие
соотношения, то говорят, что задана \emph{модель сплошной среды}.

\paragraph{Принципы построения ОС.}
Вывод определяющих соотношений основан на привлечении некоторых дополнительных принципов,
т.е. физических допущений общего характера, которые, вообще говоря, не формулируются в виде
дифференциальных уравнений в частных производных. Основными такими принципами являются:
\begin{itemize}
  \item принцип термодинамически согласованного детерминизма,
  \item принцип локальности,
  \item принцип равноприсутствия,
  \item принцип материальной индифферентности (объективности),
  \item принцип материальной симметрии,
  \item принцип Онзагера.
\end{itemize}
Кроме того, для частных моделей сред формулируют дополнительные принципы.

\paragraph{Основное термодинамическое тождество.}
% TODO Нелинейный Димитриенко, стр 177.
Выпишем законы изменения энергии и закон притока тепла в полных дифференциалах:
\[
  \begin{cases}
    \rho \dfrac{de}{dt} =
    \mathbf{T} \cdot\cdot(\nabla \otimes \mathbf{v})^T +
    \rho q_m - \nabla\cdot\mathbf{q}, \\
    \rho\theta \dfrac{d\eta}{dt} =
    \rho q_m - \nabla \cdot \mathbf{q} + w^*,
  \end{cases}
\]
исключая $\nabla \cdot \mathbf{q}$ из этих уравнений, и вспоминая обозначение $w_{(i)} = \mathbf{T} \cdot\cdot (\nabla \otimes \mathbf{v})^T$, получим:
\[
  \rho \left( \dfrac{de}{dt} - \theta \dfrac{d\eta}{dt} \right) - w_{(i)} + w^* = 0.
\]

Сюда вместо $w_{(i)}$ можно подставить выражение через энергетические пары тензоров
(см. следующий вопрос), и тогда
это соотношение можно мыслить себе как некоторое соотношение, связывающее изменение трёх
основных величин: $e, \theta$ и $\stackrel{(n)}{\mathbf{C}}$
(или $e, \theta$ и $\stackrel{(n)}{\mathbf{G}}$,
или $e, \theta, \stackrel{(n)}{\mathbf{A}}$ и $\mathbf{O}^T$, 
или $e, \theta, \stackrel{(n)}{\mathbf{g}}$ и $\mathbf{O}^T$)
в локальной точке сплошной среды.

Различают также $e$- и $\psi$-формы ОТТ, то выражение, которое написано выше, называют $e$-формой,
потому что оно содержит энергию $e$. Если ввести т.н. \emph{свободную энергию Гельмгольца} 
$\psi = e - \theta \eta$, то получим выражение в $\psi-$ форме:
\[
  \rho \dfrac{d\psi}{dt} + \rho\eta \dfrac{d\theta}{dt} - w_{(i)} + w^* = 0.
\]

Вообще говоря, ОТТ можно записать ещё в 100 других видах, вводя всё новые и новые (и никому не
нужные) замены, но важно, что в ОТТ некоторые переменные входят через производные -- такие
будет называть \emph{реактивными} $\mathcal{R}$, а другие назовём \emph{активными} $\Lambda$.
