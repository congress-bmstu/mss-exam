\que{Вектор перемещений,  соотношения между перемещениями и градиентом
деформаций, перемещениями и тензорами деформаций. Соотношения Коши в случае
малых деформаций.}
\begin{definition*}
  \emph{Вектором перемещения} точки $ \mathcal M $ называется вектор  
  \[
    \mathbf{u} = \mathbf{x} - \mathring{\mathbf{x}}.
  \]
\end{definition*}

\begin{theorem*}[соотношения между перемещениями и градиентом деформаций]
  \begin{align*}
  &\begin{aligned}
    F &= E + (\mathring{\nabla}\otimes \mathbf{u})^{\mathsf T}, & F^{-1} &= E -
    (\nabla \otimes \mathbf{u})^{\mathsf T},\\
    F^{\mathsf T} &= E + \mathring{\nabla}\otimes \mathbf{u}, & F^{-1\mathsf{T}}
                  &= E - \nabla \otimes \mathbf{u},
  \end{aligned}\\
  &C = \frac{1}{2} ( \mathring{\nabla}\otimes \mathbf{u} +
  \mathring{\nabla}\otimes \mathbf{u}^{\mathsf T} + \mathring{\nabla}\otimes
\mathbf{u} \cdot \mathring{\nabla}\otimes \mathbf{u}^{\mathsf T} ),\\
  &A = \frac{1}{2} (\nabla \otimes \mathbf{u} + \nabla \otimes
  \mathbf{u}^{\mathsf T} - \nabla \otimes \mathbf{u}\cdot \nabla\otimes
  \mathbf{u}^{\mathsf T}),\\
  &\Lambda = \frac{1}{2}(\nabla \otimes \mathbf{u} + (\nabla \otimes
  \mathbf{u})^{\mathsf T} - \nabla \otimes \mathbf{u}^{\mathsf T} \cdot \nabla
  \otimes \mathbf{u}),\\
  &J = \frac{1}{2} ( \mathring{\nabla} \otimes \mathbf{u} + \mathring{\nabla}
  \otimes \mathbf{u}^{\mathsf T} + \mathring{\nabla} \otimes \mathbf{u}^{\mathsf
  T}\cdot \mathring{\nabla}\otimes \mathbf{u}).
\end{align*}
\end{theorem*}
\begin{proof}
  Докажем для примера одну из цепочек. Для начала  
  \[
    F^{\mathsf T} = \mathring{\nabla}\otimes \mathbf{x} =
    \mathring{\nabla}\otimes (\mathring{\mathbf{x}} + \mathbf{u}) =
    \mathring{\mathbf{r}}^i \otimes \mathring{\mathbf{r}}_i +
    \mathring{\nabla}\otimes \mathbf{u} = E + \mathring{\nabla} \otimes
    \mathbf{u}.
  \]
  Тогда (почему-то считается, что $ (\mathring{\nabla}\otimes
  \mathbf{u})^{\mathsf T} = \mathring{\nabla}\otimes \mathbf{u}^{\mathsf T} $)
 \[
   C = \frac{1}{2} ( (E + \mathring{\nabla} \otimes \mathbf{u}) \cdot
   (E+\mathring{\nabla}\otimes \mathbf{u}^{\mathsf T}) - E) = \frac{1}{2}
   (\mathring{\nabla}\otimes \mathbf{u} + \mathring{\nabla}\otimes
   \mathbf{u}^{\mathsf T} + \mathring{\nabla}\otimes \mathbf{u}\cdot
   \mathring{\nabla}\otimes \mathbf{u}^{\mathsf T}).
 \]
\end{proof}

Вектор $ \mathbf{u} = \mathring{u}^i \mathring{\mathbf{r}}_i = u^i \mathbf{r}_i
$ можно разложить по любому из базисов, как и его производную. Тогда 
\begin{align*}
  \mathring{\nabla}\otimes \mathbf{u} &= \mathring{\mathbf{r}}^i \otimes
  \frac{\partial \mathbf{u}}{\partial X^i} = \mathring{\nabla}_i \mathring{u}^k
  \mathring{\mathbf{r}}^i \otimes \mathring{\mathbf{r}}_k = \mathring{\nabla}^i
  \mathring{u}^k \mathring{\mathbf{r}}_i \otimes \mathring{\mathbf{r}}_k,\\
  {\nabla}\otimes \mathbf{u} &= {\mathbf{r}}^i \otimes
  \frac{\partial \mathbf{u}}{\partial X^i} = {\nabla}_i {u}^k
  {\mathbf{r}}^i \otimes {\mathbf{r}}_k = {\nabla}^i
  {u}^k {\mathbf{r}}_i \otimes \mathbf{r}_k.
\end{align*}
Отсюда 
\begin{align*}
  F &= (\delta^k_i + \mathring{\nabla}_i \mathring{u}^k)\mathring{\mathbf{r}}_k
  \otimes \mathring{\mathbf{r}}^i = \tensor{\mathring{F}}{^k_i}
  \mathring{\mathbf{r}}_k\otimes \mathring{\mathbf{r}}^i,\\
  F^{\mathsf T} &= (\delta^k_i +
  \mathring{\nabla}^k\mathring{u}_i)\mathring{\mathbf{r}}_k\otimes
  \mathring{\mathbf{r}}^i = \tensor{\mathring{F}}{_i^k}\mathring{\mathbf{r}}_k\otimes
  \mathring{\mathbf{r}}^i,\\
  F^{-1} &= (\delta^k_i -
  {\nabla}_i {u}^k) {\mathbf{r}}_k\otimes
  {\mathbf{r}}^i = \tensor{{{F}^{-1}}}{^k_i} {\mathbf{r}}_k\otimes
  {\mathbf{r}}^i,\\
  F^{-1\mathsf T} &= (\delta^k_i +
  {\nabla}^k{u}_i){\mathbf{r}}_k\otimes
  {\mathbf{r}}^i = \tensor{{{F}^{-1}}}{_i^k}{\mathbf{r}}_k\otimes
  {\mathbf{r}}^i.
\end{align*}

Подставляя теперь это в выражения для $ C $ и $ A $, получаем  
\begin{align*}
  \varepsilon_{ij} &= \frac{1}{2}(\mathring{\nabla}_i \mathring{u}_j +
  \mathring{\nabla}_j\mathring{u}_i + \mathring{\nabla}_i\mathring{u}^k
  \mathring{\nabla}_j \mathring{u}_k),\\
  \varepsilon_{ij} &= \frac{1}{2}({\nabla}_i {u}_j +
  {\nabla}_j{u}_i + {\nabla}_i{u}^k
  {\nabla}_j {u}_k).
\end{align*}
А подставляя в выражения для $ \Lambda $ и $ J $, получим
\begin{align*}
  \varepsilon^{ij} &= \frac{1}{2}(\mathring{\nabla}^i \mathring{u}^j +
  \mathring{\nabla}^j\mathring{u}^i + \mathring{\nabla}^k\mathring{u}^i
  \mathring{\nabla}_k \mathring{u}_j),\\
  \varepsilon^{ij} &= \frac{1}{2}({\nabla}^i {u}^j +
  {\nabla}^j{u}^i + {\nabla}^k{u}^i
  {\nabla}_k {u}^j).
\end{align*}

Кроме того, 
\begin{align*}
  g_{ij} &= \mathring{g}_{ij} + \mathring{\nabla}_i \mathring{u}_j +
  \mathring{\nabla}_j\mathring{u}_i + \mathring{\nabla}_i\mathring{u}^k
  \mathring{\nabla}_j \mathring{u}_k = \mathring{g}_{ij} + {\nabla}_i {u}_j +
  {\nabla}_j{u}_i + {\nabla}_i{u}^k
  {\nabla}_j {u}_k,\\
  g^{ij} &= \mathring{g}^{ij}+\mathring{\nabla}^i \mathring{u}^j +
  \mathring{\nabla}^j\mathring{u}^i + \mathring{\nabla}^k\mathring{u}^i
  \mathring{\nabla}_k \mathring{u}_j = \mathring{g}^{ij} + {\nabla}^i {u}^j +
  {\nabla}^j{u}^i + {\nabla}^k{u}^i
  {\nabla}_k {u}^j.
\end{align*}


