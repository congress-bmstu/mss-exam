\que{Симметричные, кососимметричные и ортогональные тензоры. Собственные значения и собственные векторы тензоров 2-го ранга.}

\begin{definition}[Симметричный тензор]
	\begin{equation*}
		A=A^T;\;A^{ij}=A^{ji}
	\end{equation*}
\end{definition}

\begin{definition}[Собственный вектор, собственное значение]
	Будем говорить, что $\mathbf{p}_{A\alpha}$ -- правый собственный вектор для тензора второго ранга $A$, если
	он удовлетворяет уравнению $A \cdot \mathbf{p}_{A\alpha} = \lambda_{A\alpha} \cdot \mathbf{p}_{A\alpha}$, а числа $\lambda_{A\sigma}$ -- правое собственное значение.
	
	Аналогично, левым собственным вектором называется такой вектор $\mathbf{p}^*_{A\alpha}$, что:
	$\mathbf{p}^*_{A\alpha} \cdot A = \lambda^*_{A\alpha} \cdot \mathbf{p}^*_{A\alpha}$.
\end{definition}

\begin{corollary}
	Если $A$ -- симметричный, то:
	\begin{itemize}
		\item все собственные значения вещественные;
		\item все собственные векторы вещественнозначные, т.е. все компоненты в декартовом базисе
		вещественные.
		\item $\mathbf{p}_{A\alpha} = \mathbf{p}^*_{A\alpha}$
		\item $\mathbf{p}_{A\alpha} \cdot \mathbf{p}_{A\beta} = \delta_{\alpha\beta}.$
	\end{itemize}
	
	Если он еще и положительно определенный, то все собственные значения еще и положительны.
\end{corollary}
\begin{corollary}
	Представим симметричный тензор в собственном базисе. Если $A$ -- симметричный,
	$\mathbf{p}_{A\alpha}$ -- собственные векторы, $\lambda_{A\alpha}$ -- собственные значения,
	тогда
	\[
	A = \sum_{\alpha=1}^3 \lambda_{A\alpha} \mathbf{p}_{A\alpha} \otimes \mathbf{p}_{A\alpha}
	= A^{ij} \mathbf{r}_i \otimes \mathbf{r}_j.
	\]
	\begin{proof}
		Для доказательства подставим в определение собственных векторов для симметричного тензора:
		\[
		A \cdot \mathbf{p}_{A\alpha}
		= \sum_{\beta=1}^3 \lambda_{A\beta} (\mathbf{p}_{A\alpha}\otimes \mathbf{p}_{A\beta}) \cdot \mathbf{p}_{A\alpha}
		= \sum_{\beta=1}^3 \lambda_{A\beta} \mathbf{p}_{A\alpha} \otimes \delta_{\alpha\beta}
		= \sum_{\beta=1}^3 \lambda_{A\beta} 
		= \lambda_{A\alpha} \mathbf{p}_{A\alpha}
		\]
	\end{proof}
\end{corollary}

\begin{definition}[Ортогональный тензор]
	\begin{equation*}
		O\in\mathcal{E}_3^{(2)}: O^T=O^{-1}
	\end{equation*}
\end{definition}
Введем компоненты в диадном базисе:
\begin{equation*}
	O = \tensor{O}{^i_j}\mathbf{e}_i\otimes\mathbf{e}^j,
\end{equation*}
тогда
\begin{equation*}
	\tensor{O}{^i_j}\tensor{O}{_k^j}=\tensor{O}{^i^k}\tensor{O}{_k_j}=\delta^i_k \Leftrightarrow
	\tensor{O}{^i_j}\tensor{O}{^m_k}g_{im}=g_{jk}.
\end{equation*}
В ортонормированном базисе:
\begin{equation*}
	\tensor{O}{^i_j}\tensor{O}{^m_k}\delta_{im}=\delta_{jk}.
\end{equation*}
Определитель:
\begin{equation*}
	\left(\det O\right)^2 = \det(O^T)\det(O) = \det(O^TO)=\det E = 1 \Rightarrow \det O = \pm1
\end{equation*}
Строки и столбцы умноженные сами на себя дают единицу, а перемноженные попарно дают нуль:
\begin{equation*}
	\sum_{\alpha=1}^{3}\tensor{O}{^\alpha_\beta}^2=1,\quad\sum_{\alpha=1}^{3}\tensor{O}{^\alpha_\beta}\tensor{O}{^\alpha_\gamma}=1,\quad \alpha,\beta,\gamma=1,2,3.
\end{equation*}

Ортогональный тензор всегда имеет одно действительное собственное значение, равное 1, и два, вообще говоря, комплексных.

