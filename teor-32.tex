\que{Группы симметрии сплошной среды. Принцип материальной симметрии. Определение жидких и твердых сред.}

\paragraph{Группы симметрии сплошной среды.}

\begin{definition*}
	Множество всех $H$-преобразований $\mathring{\mathcal{K}} \to \overset{\ast}{\mathcal{K}}$ отсчетной конфигурации $\mathring{\mathcal{K}}$ сплошной среды (или множество $\mathbf{H}$-тензоров) с определяющими соотношениями, для которого выполнены соотношения $\mathbf{H}$-преобразований, образует группу, называемую \textit{группой симметрии} $\mathring{G}_{s}$ сплошной среды. 
	
	Эту группу иногда называют \textit{группой эквивалентности.}
\end{definition*}

\begin{utv*}
	Можно показать, что множество всех $H$-преобразований является группой. Это делает Димитриенка на стр. 197.
\end{utv*}

\begin{axiom*}
	Для любой сплошной среды с определяющими соотношениями для движения $\mathring{\mathcal{K}} \to \mathcal{K}$ и с произвольной отсчетной конфигурацией $\mathring{\mathcal{K}}$ существует соответствующая группа симметрии $\mathring{G}_{s}$ --- группа $H$-преобразований отсчетной конфигурации $H : \mathring{\mathcal{K}} \to \overset{\ast}{\mathcal{K}}$, которые не изменяют определяющих соотношений, т.е. для любого движения $\overset{\ast}{\mathcal{K}} \to \mathcal{K}$ имеет место:
	\begin{equation*}
		\overset{\ast}{\Lambda} = \overset{\cup}{f}(\overset{\ast}{\mathcal{R}}).
	\end{equation*} 
\end{axiom*}

\paragraph{Определение жидких и твердых сред.} 

\begin{theorem*}
	Всякий $\mathbf{H}$-тензор, соответствующий $H$-преобразованию, является унимодулярным, т.е. удовлетворяет соотношению 
	\begin{equation*}
		\det{\mathbf{H}} = \pm 1.
	\end{equation*}
\end{theorem*}

\begin{theorem*}
	Группа симметрии $\mathring{G}_s$ любой сплошной среды является подгруппой полной унимодулярной группа:
	\begin{equation*}
		\mathring{G}_s \subset U.
	\end{equation*}
\end{theorem*}

\begin{definition*}
	Сплошную среду, которая для любой отсчетной конфигурации $\mathring{\mathcal{K}}$ имеет группу симметрии $\mathring{G}_s$, совпадающую с полной унимодулярной группой 
	\begin{equation*}
		\mathring{G}_{s} = U,
	\end{equation*}
	нызывают \textit{жидкостью} (жидкой средой).
\end{definition*}

\begin{definition*}
	Сплошную среду, для которой существует отсчетная конфигурация $\hat{\mathcal{K}}$, такая что ее группа симметрии $\hat{G}_s$ является подгруппой полной ортогональной группы
	\begin{equation*}
		\hat{G}_s \subset I,
	\end{equation*}
	называют \textit{твердой средой} (твердым телом). 
\end{definition*}
