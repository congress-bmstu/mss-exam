\que{Уравнение  движения  в  пространственном и материальном описании.}

Подставляя обозначения для внутренних поверхностных сил и формулу Коши в закон изменения количества движения в интегральной форме, получим еще одну часто используеиую формулировку закона изменения количества движения:
\begin{equation*}
	\frac{d}{dt} \int\limits_{V} \rho \mathbf{v} \, dV = \int\limits_{\Sigma} \mathbf{n} \cdot \mathbf{T} \, d\Sigma + \int\limits_{V} \rho \mathbf{f} \, dV.
\end{equation*}

Теперь подставим все то же в \eqref{omega} (вопрос №16):
\begin{equation*}
	\int\limits_{V} \left(\rho \frac{d\mathbf{v}}{dt} - \rho \mathbf{f}\right) = \int\limits_{\Sigma} \mathbf{n} \cdot \mathbf{T} \, d\Sigma,
\end{equation*}
и преобразуя поверхностный интеграл к объемному по формуле Гаусса-Остроградского, придем к следующему соотношению:
\begin{equation*}
	\int\limits_{V} \left(\rho \frac{d \mathbf{v}}{dt} - \rho \mathbf{f} - \nabla \cdot \mathbf{T}\right) \, dV = 0.
\end{equation*}

Отсюда, в силу произвольности объема $V$, заключаем, что подынтегральное выражение должно всегда обращаться в ноль. Итак мы доказали следующую теорему.

\begin{theorem*}
	Если функции $\mathbf{F}$, $\mathbf{v}$, $\mathbf{T}$ и $\mathbf{f}$, удовлетворяющие закону изменения количества движения и зависящие от $x^i$, $t$ являются непрерывно-дифференцируемыми в $V(t)$ для всех рассматриваемых $t \geqslant 0$, то в каждой точке $\mathcal{M} \in V(t)$ имеет место \textbf{уравнение движения в эйлеровом описании} (т.е. в $\mathcal{K}$):
	\begin{equation*}
		\rho \frac{d\mathbf{v}}{dt} = \nabla \cdot \mathbf{T} + \rho \mathbf{f}.
	\end{equation*}
\end{theorem*}


Используя свойство полной производной, уравнение движения можно записать в \textit{дивергентной форме}:
\begin{equation*}
	\frac{\partial \rho \mathbf{v}}{\partial t} + \nabla \cdot \rho \mathbf{v} \otimes \mathbf{v} = \nabla \cdot \mathbf{T} + \rho \mathbf{f}.
\end{equation*}

Преобразуем теперь уравнение движения:
\begin{equation*}
	\int\limits_{\mathring{V}} \mathring{\rho} \left(\frac{d\mathbf{v}}{dt} - \mathbf{f}\right) \, d\mathring{V} - \int\limits_{\mathring{\Sigma}} \mathring{\mathbf{n}} \cdot \mathbf{P} \, d\mathring{\Sigma} = 0. 
\end{equation*}

С помощью теоремы Гаусса-Остроградского преобразуем это уравнение к виду:
\begin{equation*}
	\int\limits_{\mathring{V}} \left(\mathring{\rho} \left(\frac{d\mathbf{v}}{dt} - \mathbf{f}\right) - \mathring{\nabla} \cdot \mathbf{P}\right) \, d\mathring{V} = 0.
\end{equation*}

Откуда в силу произвольности объема $\mathring{V}$, получаем еще одну теорему. 

\begin{theorem*}
	Если выполнены условия предыдущей теоремы, то в каждой точке $\mathcal{M} \in \mathring{V}$ для всех рассматриваемых $t \geqslant 0$ имеет место \textbf{уравнение движения в лагранжевом (материальном) описании} (т.е в $\mathring{\mathcal{K}}$):
	\begin{equation*}
		\mathring{\rho} \frac{d\mathbf{v}}{dt} = \mathring{\rho} \mathbf{f} + \mathring{\nabla} \cdot \mathbf{P}.
	\end{equation*}
	
	Так как в лагранжевом описании полная производная по времени совпадает с частной, то дивергентная форма уравнения движения в $\mathring{\mathcal{K}}$ совпадает с приведенной выше формулой:
	\begin{equation*}
		\mathring{\rho} \frac{\partial\mathbf{v}}{\partial t} = \mathring{\rho} \mathbf{f} + \mathring{\nabla} \cdot \mathbf{P}.
	\end{equation*}
\end{theorem*}