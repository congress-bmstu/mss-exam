\documentclass[12pt]{article}

\usepackage{svg}
\svgpath{{img/}}
\usepackage{wrapfig}
\usepackage{caption}
\usepackage{subcaption}
\usepackage{float}

\usepackage[T2A]{fontenc}
\usepackage[utf8]{inputenc}
\usepackage[english,russian]{babel}

\usepackage{amsmath}
\usepackage{amsthm, mathrsfs, mathtools, amssymb}
\usepackage{enumitem}

\usepackage{physics}

\usepackage{epigraph}

\usepackage{tikz}
\usepackage{subcaption}
\usepackage{caption}
\usepackage{tensor}
\usepackage{float}
\usepackage{multirow}
\usepackage{multicol}
\usepackage{wrapfig}
\usepackage{hyperref}
\hypersetup{
	colorlinks,
	citecolor=black,
	filecolor=black,
	linkcolor=black,
	urlcolor=black
}
\usepackage{titlesec}
\DeclareMathOperator{\supp}{supp}

\usepackage{geometry}
\geometry{verbose,a4paper,tmargin=1cm,bmargin=2cm,lmargin=1.5cm,rmargin=1.5cm}

\newtheoremstyle{example}% name
{0.7cm}% Space above
{0.7cm}% Space below
{\small}% Body font
{}% Indent amount
{\small\scshape}% Theorem head font
{.}% Punctuation after theorem head
{.5em}% Space after theorem head
{}% Theorem head spec (can be left empty, meaning ‘normal’)

\theoremstyle{example}
\newtheorem{example}{Пример}

\theoremstyle{plain}
\newtheorem{theorem}{Теорема}
\newtheorem{corollary}{Следствие}
\newtheorem*{corollary*}{Следствие} 
\newtheorem{lemma}{Лемма}
\newtheorem{utv}{Утверждение}
\newtheorem*{utv*}{Утверждение}

\theoremstyle{definition}
\newtheorem{definition}{Определение}
\newtheorem*{definition*}{Определение}
\newtheorem{question}{Вопрос}

\theoremstyle{remark}
\newtheorem{remark}{Замечание}
\newtheorem*{remark*}{Замечание}
\numberwithin{remark}{section}

\frenchspacing

\usepackage[labelsep=period]{caption}
\captionsetup{font = small}

\newcommand{\Hom}{\mathrm{Hom}}
\newcommand{\Spl}{\mathrm{Spl}}
\newcommand{\spl}{\mathrm{spl}}

\DeclareMathAlphabet{\mathbfit}{OML}{cmm}{b}{it}


\newcounter{problem} 
\newenvironment{problem}[1][]
{
	\refstepcounter{problem} 
	\par \vspace{0.7em} \noindent
	\textbf{Задача \theproblem}\ifx&#1&\else\ (#1)\fi. 
}
{
	\vspace{1em}	
}

\newenvironment{solution}
{
	\vspace{0.3em}
	\par\textsc{Решение.}
}
{
	\qed
}

\newcommand{\que}[1]{%
	\subsection{#1}
}
\renewcommand{\thesubsection}{\arabic{subsection}}

\titleformat{\section}{\centering\LARGE\bf\scshape}{}{1em}{}

\begin{document}
%	\tableofcontents 
	
	\begin{center}
		\Huge \bf	
		\textsc{Механика сплошных сред}
		\rule{\textwidth}{0.4pt}
	\end{center}
	
  % Элементы тензорного исчисления
  \que{Локальные  векторные базисы, метрические матрицы, преобразования  координат.}
\paragraph{ Координаты и локальные векторы базиса. Локальные векторные базисы, метрические матрицы, преобразования координат.}
Введем в трехмерном пространстве прямоугольную декартову систему координат $x^i, і= 1, 2, 3$ с началом в точке 0, тогда каждой точке пространства $М$ будет взаимооднозначно соответствовать радиус-вектор $х$ с началом в точке $O$ и концом в точке $М$. Выберем такой ортонормированный базис $\bar{\textbf{e}}_i$, укоторого линии действия векторов совпадают с осями $Ox^i$, а начало -- с точкой $О$ декартовой системы координат. Такой базис будем называть декартовым. Радиус-вектор $x$ всегда можно разложить по базису $\bar{\textbf{e}}_i$:

\begin{equation}
\label{eq:11}
    \textbf{x} = x^i\bar{\textbf{e}}_i
\end{equation}
где $x^i$ - координаты точки $М$ в декартовой системе координат. Трехмерное пространство, в котором существует единая для всех точек прямоугольная декартова система координат, называют трехмерным евклидовым.
Введем криволинейные координаты $X^i$, которые связаны с $x^i$ функциями вида:
\begin{equation}
\label{eq:12}
    \textbf{x} = \textbf{x}(X^j)
\end{equation}
Тогда радиус-вектор точки $М$ может быть представлен как функция координат 
$X^j$:
\begin{equation}
\label{eq:13}
    x^i = x^i(X^j)
\end{equation}
Будем далее предполагать, что функции $x^i$ непрерывно дифференцируемы и взаимнооднозначны, тогда их можно обратить:
\begin{equation}
\label{eq:14}
    X^j = X^j(x_i)
\end{equation}
Ввиду дифференцируемости функций $\textbf{x}$ можно ввести производные:

\begin{equation}
\label{eq:15}
    \textbf{R}_j = \partial\textbf{x}/\partial X^j
\end{equation}
которые образуют векторы, назваемые локальными векторами базиса. Эти векторы в $\textbf{R}_j$ образуют \textit{локальные вектора базиса}, они направлены по касательным к координатнымлиниям $X_j=const$ вточке $М$.
Вотличие от $\bar{\textbf{e}}_i$, векторы базиса $\textbf{R}_j$  меняются при переходе от одной точки $М$ к другой $М'$.
Заметим, что хотя формально $\bar{\textbf{e}}_i$, и $\textbf{R}_j$ определялись вразных точках, их можно привести к одной точке, так как векторы $\bar{\textbf{e}}_i$, являются свободными.


\paragraph {Якобиевы матрицы}
Можно связать векторы базисов $\bar{\textbf{e}}_i$, и $\textbf{R}_j$  Из \eqref{eq:11}  и \eqref{eq:15} находим выражение:
\begin{equation}
\label{eq:16}
    \textbf{R}_j = \frac{\partial\textbf{x}}{\partial X^j} = \frac{\partial x^i}{\partial X^j} \bar{\textbf{e}}_i = Q^i_j \bar{\textbf{e}}_i
\end{equation}
где введен объект $\bar{Q}^j_i$ с двумя индексами
\begin{equation}\label{jacobi-matrix}
    \bar{Q}^j_i= \frac{\partial x^j}{\partial X^i}
\end{equation}
Такие объекты всегда можно записать в виде матрицы. Матрицу \eqref{jacobi-matrix} называют матрицей преобразования или якобиевой матрицей.

В силу взамной однозначности функций \eqref{eq:14}, в любой точке $X^i$ детерминант якобиевой матрицы всегда отличен от нуля.

Обратную матрицу будем также иногда обозначать как 
\begin{equation}
\label{eq:19}
    \bar{P}^i_j\bar{Q}^j_k = \bar{Q}^j_i\bar{P}^j_k  = \delta^i_k
\end{equation}
где $\delta^i_k$ смешанный символ Кронекера
\begin{equation}
\label{eq:110}
    \delta^i_k = \begin{cases}
    1, i = k\\
    0, i \ne k
    \end{cases}
\end{equation}
Также обозначают
\begin{equation}
    \bar{P}^i_j= (\bar{Q^{-1}})^i_j   = (\bar{Q}^i_j)^{-1} 
\end{equation}
Введем также ковариантный символ Кронекера $\delta_{ik}$  и контравариантный символ Кронекера $\delta^{ik}$ , значения которых совладают с $\delta^i_k$ :
\begin{equation}
\label{eq:110}
	\textbf{????}
\end{equation}

Для якобиевой матрицы $\bar{Q}^j_i$ обратная акобиева матрица имеет вид:
\begin{equation}
\label{eq:111}
    \bar{P}^j_i= \frac{\partial X^j}{\partial x^i}
\end{equation}
Спомощью $\bar{P}^j_i$, можно выразить векторы декартова базиса $e_k$ через $\textbf{R}_i$ . Домножая (1.6) на $\bar{P}^i_k$, получаем
\begin{equation}
\label{eq:112}
    e_k = \bar{P}^i_k \textbf{R}_i
\end{equation}
\paragraph{Метрические матрицы}
В силу ортонормированности векторов е;, их скалярное произведе- ние можно записать с помощью символов Кронекера:
\begin{equation}
\label{eq:113}
    \bar{\textbf{e}}_k \cdot \bar{\textbf{e}}_j = \delta_{kj}
\end{equation}
Тогда скалярное произведение векторов $\textbf{R}_i, \textbf{R}_j$ с помощью уравнений \eqref{eq:16} и \eqref{eq:113}  можно представить в виде:
\begin{equation}
\label{eq:114}
    \textbf{R}_i\cdot \textbf{R}_j = 
    \bar{Q}^k_i \bar{Q}^l_j \bar{\textbf{e}}_k \bar{\textbf{e}}_l =
    \bar{Q}^k_i \bar{Q}^l_j \delta_{kl} =
    g_{ij}
\end{equation}
 Матрица $g_{ij}$, введенная по формуле \eqref{eq:114}, называется
метрической или фундаментальной.

Определитель метрической матрицы обозначим следующим образом:
\begin{equation}
\label{eq:115}
    g = \det(g_{ij}) = (\det(\bar{Q}^k_i))^2
\end{equation}
Метрическая матрица, очевидно, является симметричной по индексам $і,j$;

В силу \eqref{eq:118} всегда $g \ne 0$, поэтому для $g_{ij}$; всегда существует обратная метрическая матрица $g^{kl}$
Непосредственной подстановкой выражения $g_{ij} = \bar{Q}^k_i \bar{Q}^l_j \delta_{kl}$ в \eqref{eq:117}  можно убедиться в том, что обратная метрическая матрица $g^{ij}$ выражается через обратную матрицу Якоби:
\begin{equation}
\label{eq:118}
    g^{ij} = \bar{P}^i_k \bar{P}^j_l \delta^{kl}
\end{equation}
\paragraph{Векторы взаимного базиса}
 помощью $g^{ij}$ определяем векторы локального взаимного базиса $\textbf{R}_i =g^{ik}\textbf{R}_k$
\begin{equation}
\label{eq:119}
    \textbf{R}_i = \delta^k_i\textbf{R}_k =
    g^{kl}g_{li}\textbf{R}_k = g_{li}\textbf{R}^l
\end{equation}
  \que{Диадные  базисы, алгебраическое определение  тензора, геометрическое представление тензора }

\begin{definition}[алгебраическое, \textbf{из википедии}]
	Тензором типа $(^s _r)$ называется \textit{полилинейная функция} (полилинейная форма) $$T \colon  (V^*)^s \times V^r\to R,$$ то есть числовая функция от $s+r$ аргументов следующего вида $$T(f_1,f_2,...,f_s,x_1,x_2,...,x_r),$$ где $f_i$ -- линейные функционалы на $V$, а $x_j$ -- векторы пространства $V$.
	
	Координатами тензора в некотором базисе будут значения полилинейной функции на различных комбинациях базисных векторов: $T^{i_1i_2...i_s}_{j_1j_2...j_r}=T(e^{i_1},e^{i_2},...,e^{i_s},e_{j_1},e_{j_2},...,e_{j_r})$
\end{definition}
\begin{remark}[Определение полилинейной функции]
	\textbf{Просто для понимания записал, чтобы в определении выше не раскидываться терминами. Думаю, что в ответ это не стоит писать.}
	
	Основным объектом полилинейной алгебры является полилинейное ($n$-линейное) отображение:
	 $$f : V_1 \times \dots \times V_n \rightarrow W,$$
	где $V_1, \dots, V_n$ и $W$ -- векторные пространства над определённым полем $K$. Условие $n$ -- линейности означает, строго говоря, что для каждого $i = 1, \dots, n$ семейство отображений
	$$(\pi_if)_{\{x_k | k \ne i\}} : V_k \rightarrow W; \quad 
	(\pi_if)_{\{x_k | k \ne i\}}(x_i) = f(x_1, \dots, x_n),$$
	зависящее от $n - 1$ переменных $\{x_k | k \ne i\}$ как от параметров, состоит из линейных отображений. Можно также определить $n$-линейное отображение рекурсивно (по индукции), как <<линейное>> отображение из $V_n$ в векторное пространство $(n - 1)$-линейных отображений.
\end{remark}

\begin{figure}[ht!]
	\centering
	\includesvg[scale = 0.8]{que2}
	\caption{Геометрическое представление тензора}
	\label{fig:que2}
\end{figure}
\begin{definition}[геометрическое]
	Пусть
	\begin{enumerate}
		\item $\mathcal{L}_n$ --- линейное (векторное) пространство --- \emph{порождающее пространство}.
		Выберем две системы векторов в $\mathcal{L}_n$: $\mathbf{a}_i, \mathbf{b}^{[i]}, i = \overline{1, n}$.
		(квадратные скобки --- просто обозначение).
		
		\item Построим формальный векторный набор из $\mathbf{a}_i,
		\mathbf{b}^{[j]}$ длины $2n$.
		Векторы из $\mathbf{a}_i$ называем левыми, а из $\mathbf{b}^{[i]}$
		правыми. Таким образом\footnote{Далее скобки и запятые будут опускаться.}, 
		$A \equiv ((\mathbf{a}_1, \mathbf{b}^{[1]}), (\mathbf{a}_2,
		\mathbf{b}^{[2]}), \dots (\mathbf{a}_n, \mathbf{b}^{[n]}))
		\equiv ((\mathbf{a}_i, \mathbf{b}^{[i]}))$.
		Здесь у нас $ n $ пар векторов (и в каждой паре $ i $ есть левый $\mathbf{a}_i$ и правый вектор
		$\mathbf{b}^{[i]}$).
		
		\item Введем теперь операции с векторными наборами.
		\begin{enumerate}
			\item \textsc{Сложение} однотипных векторных наборов. \emph{Однотипными} будем называть такие наборы, у 
			которых совпадают либо все левые, либо все правые векторы:
			\[
			A_1 \equiv \mathbf{a}_i \mathbf{b}^{[i]} \leftrightarrow A_2 = \mathbf{a}_i \mathbf{c}^{[i]}; 
			\quad
			A_1 \leftrightarrow A_3 = \mathbf{d}_i \mathbf{b}^{[i]}.
			\]
			Тогда
			\[
			A_1+A_2 = \mathbf{a}_i (\mathbf{b}^{[i]} + \mathbf{c}^{[i]}); \quad
			A_1+A_3 = (\mathbf{a}_i + \mathbf{d}_i) \mathbf{b}^{[i]}.
			\]
			Это частичная операция, то есть такая, которая определена только на
			подмножестве элементов (векторов). К тому же, как легко видеть, она рефлексивна,
			симметрична, но не транзитивна, поэтому не может быть названа
			эквивалентностью.
			
			\item \textsc{Умножение} на число $s \in \mathbb{R}$.
			\[
			sA = (s \mathbf{a}_i) \mathbf{b}^{[i]} = \mathbf{a}_i (s \mathbf{b}^{[i]}).
			\]
			Данное выражение показывает, что мы называем равными не только
			полностью совпадающие наборы.
			
			\item \textsc{Эквивалентность} векторных наборов.
			Векторные наборы $A$ и $B$ называются \emph{эквивалентными}, если выполняется хотя бы одно 
			из следующих условий:
			\begin{enumerate}
				\item Векторные наборы $A$ и $B$ состоят из одних и тех же пар, но упорядоченных
				произвольным образом.
				
				Например,
				$A = \mathbf{a}_1 \mathbf{b}^{[1]} \mathbf{a}_2 \mathbf{b}^{[2]},
				\, B = \mathbf{a}_2 \mathbf{b}^{[2]} \mathbf{a}_1 \mathbf{b}^{[1]}$, откуда $A \sim B$.
				
				\item Набор $A$ может быть получен из другого набора с помощью согласованной операции
				умножения левых и правых векторов:
				\[
				A = \mathbf{a}_i \mathbf{b}^{[i]} \sim B = (s \mathbf{a}_i)
				(s^{-1}\mathbf{b}^{[i]})
				\quad \forall s \in \mathbb{R},\ s \neq 0.
				\]
				
				\item Если в $A$ и $B$ все векторы $\mathbf{a}_i$ и $\mathbf{b}^{[i]}$ совпадают, кроме тех пар,
				у которых хотя бы один вектор нулевой.
				Например,
				$A=\mathbf{a}_1 \mathbf{b}^{[1]} (\mathbf{a}_i \mathbf{0}) \sim B = \mathbf{a}_1 \mathbf{b}^{[1]} (\mathbf{0} \mathbf{b}^{[2]}) \sim \mathbf{a}_1 \mathbf{b}^{[1]} (\mathbf{c}_{2} \mathbf{0})$.
			\end{enumerate}
			
			\item Пусть теперь есть некоторый векторный набор $A = \mathbf{a}_i
			\mathbf{b}^{[i]}$. Введём 
			множество всех векторных наборов $B = \mathbf{c}_i \mathbf{d}^{[i]}$, эквивалентных $A$ и
			обозначим его $T = [A] = [\mathbf{a}_i \mathbf{b}^{[i]}]$. \end{enumerate}
		Таким
		образом определён
		\emph{тензор второго ранга}.
		
		\item \textsc{Диады и базисные диады}. Пусть $A = \mathbf{a}_i
		\mathbf{b}^{[i]}$, где
		существует не более чем одна пара ненулевых векторов $\mathbf{a}_i
		\mathbf{b}^{[i]}$. Положим по определению
		\[
		[\mathbf{a}_1 \mathbf{b}^{[1]} \mathbf{0} \mathbf{0} \dots \mathbf{0} \mathbf{0}] = \mathbf{a}_1 \otimes \mathbf{b}^{[1]}; \quad
		[\mathbf{0} \mathbf{0} \mathbf{a}_2 \mathbf{b}^{[2]} \dots \mathbf{0} \mathbf{0}] = \mathbf{a}_2 \otimes \mathbf{b}^{[2]}; \quad
		[\mathbf{0} \mathbf{0} \mathbf{0} \mathbf{0} \dots \mathbf{a}_1 \mathbf{b}^{[1]}] = \mathbf{a}_n \otimes \mathbf{b}^{[n]};
		\]
		
		Иными словами, мы научились по любой паре векторов конструировать диаду. Пусть $\mathbf{e}_i$ и
		$\mathbf{h}_j$ -- базисы в $\mathcal{L}_n$. Первый выберем в качестве левых векторов, а
		второй --- правых. Набор $[\mathbf{e}_1 \mathbf{0} \mathbf{e}_2 \mathbf{0}
		\dots \mathbf{e}_i \mathbf{h}_j \dots \mathbf{e}_n \mathbf{0}] =
		\mathbf{e}_i \otimes \mathbf{h}_j$ --- базисная диада. (Вместо всех
		$\mathbf{e}_k, k\neq i$ можно было поставить нули.) 
		
		\item \textsc{Диадный базис.}
		\begin{equation*}
			\mathbf{h}_j = \mathbf{e}_j \longrightarrow \mathbf{e}_i \otimes
			\mathbf{e}_j.
		\end{equation*}
		
		\begin{theorem}
			Любой тензор второго ранга $ T = [\mathbf{a}_i \mathbf{b}^{[i]}]$ можно представить в виде линейной комбинации базисных диад
			\begin{equation}\label{lec_2:eq:tensor_basis}
				T = T^{ij} \mathbf{e}_i \otimes \mathbf{e}_j,
			\end{equation}
			причём
			\[
			T = (T^{ij} \mathbf{e}_i) \otimes \mathbf{e}_j
			= [\mathbf{a}^{[j]} \mathbf{e}_j]
			= [\mathbf{e}_i \mathbf{b}^{[j]}].
			\]
		\end{theorem}
		\begin{corollary*}
			Множество всех тензоров (второго ранга) образует линейное пространство,
			где сложение векторов подчиняется правилам
			\begin{gather*}
				T_1 = [\mathbf{a}_i \mathbf{b}^{[i]}], \quad
				T_2 = [\mathbf{c}_j \mathbf{d}^{[j]}]; \\
				\mathbf{a}_i = a^j_i \mathbf{e}_j, \quad
				\mathbf{b}^{[i]} = b^{ik} \mathbf{e}_k,\quad
				\mathbf{c}_i = c^{j}_{i} \mathbf{e}_j, \quad
				\mathbf{d}^{[i]} = d^{ik} \mathbf{e}_k,  \\
				T_1 = [(a^i \mathbf{e}_i) (b^{[ik]} \mathbf{e}_k)] = a^j_i, \quad
				T_1 + T_2 = (a^j_i b^{ik} + c^{j}_i d^{jk}_i) \mathbf{e}_j \otimes \mathbf{e}_k
				= (T_1^{ij} + T_2^{ij}) \mathbf{e}_i \otimes \mathbf{e}_j,
			\end{gather*}
			а базисные диады образуют базис в тензорном пространстве.
		\end{corollary*}
	\end{enumerate}
\end{definition}


  \que{Алгебраические операции с  тензорами, геометрическое представление операций с тензорами.}

  \que{Симметричные, кососимметричные и ортогональные тензоры. Собственные значения и собственные векторы тензоров 2-го ранга.}

\begin{definition}[Симметричный тензор]
	\begin{equation*}
		A=A^T,\quad A^{ij}=A^{ji}.
	\end{equation*}
\end{definition}

\begin{definition}[Кососимметричный тензор]
	\begin{equation*}
		\mathbf{A} = -\mathbf{A}^T.
	\end{equation*}
\end{definition}

\begin{definition}[Собственный вектор, собственное значение]
	Будем говорить, что $\mathbf{p}_{A\alpha}$ --- \emph{правый собственный
  вектор} для тензора второго ранга $A$, если
	он удовлетворяет уравнению $A \cdot \mathbf{p}_{A\alpha} = \lambda_{A\alpha}
  \cdot \mathbf{p}_{A\alpha}$; числа $\lambda_{A\sigma}$ будем называть
  \emph{правыми собственными значениями.}
	
  Аналогично, \emph{левым собственным вектором} называется такой вектор $\mathbf{p}^*_{A\alpha}$, что
	$\mathbf{p}^*_{A\alpha} \cdot A = \lambda^*_{A\alpha} \cdot \mathbf{p}^*_{A\alpha}$.
\end{definition}

\begin{corollary}
	Если $A$ --- симметричный, то
  \begin{itemize}[label=--]
		\item все собственные значения вещественные;
    \item все собственные векторы вещественнозначные, т.\,е. все компоненты в
      декартовом базисе
		вещественные;
		\item $\mathbf{p}_{A\alpha} = \mathbf{p}^*_{A\alpha}$;
		\item $\mathbf{p}_{A\alpha} \cdot \mathbf{p}_{A\beta} = \delta_{\alpha\beta}.$
	\end{itemize}
	
	Если он к тому же положительно определенный, то все собственные значения еще и положительны.
\end{corollary}
\begin{corollary}
	Представим симметричный тензор в собственном базисе. Если $A$ --- симметричный,
	$\mathbf{p}_{A\alpha}$ --- собственные векторы, $\lambda_{A\alpha}$ --- собственные значения,
	тогда
	\[
	A = \sum_{\alpha=1}^3 \lambda_{A\alpha} \mathbf{p}_{A\alpha} \otimes \mathbf{p}_{A\alpha}
	= A^{ij} \mathbf{r}_i \otimes \mathbf{r}_j.
	\]
	\begin{proof}
		Для доказательства подставим в определение собственных векторов для симметричного тензора:
		\[
		A \cdot \mathbf{p}_{A\alpha}
		= \sum_{\beta=1}^3 \lambda_{A\beta} (\mathbf{p}_{A\alpha}\otimes \mathbf{p}_{A\beta}) \cdot \mathbf{p}_{A\alpha}
		= \sum_{\beta=1}^3 \lambda_{A\beta} \mathbf{p}_{A\alpha} \otimes \delta_{\alpha\beta}
		% = \sum_{\beta=1}^3 \lambda_{A\beta} %TODO
		= \lambda_{A\alpha} \mathbf{p}_{A\alpha}.
		\]
	\end{proof}
\end{corollary}

\begin{definition}[Ортогональный тензор]
	\begin{equation*}
		O\in\mathcal{E}_3^{(2)}\colon O^T=O^{-1}.
	\end{equation*}
\end{definition}
Введем компоненты в диадном базисе:
\begin{equation*}
	O = \tensor{O}{^i_j}\mathbf{e}_i\otimes\mathbf{e}^j,
\end{equation*}
тогда
\begin{equation*}
	\tensor{O}{^i_j}\tensor{O}{_k^j}=\tensor{O}{^i^k}\tensor{O}{_k_j}=\delta^i_k \Leftrightarrow
	\tensor{O}{^i_j}\tensor{O}{^m_k}g_{im}=g_{jk}.
\end{equation*}
В ортонормированном базисе:
\begin{equation*}
	\tensor{O}{^i_j}\tensor{O}{^m_k}\delta_{im}=\delta_{jk}.
\end{equation*}
Определитель:
\begin{equation*}
	\left(\det O\right)^2 = \det(O^T)\det(O) = \det(O^TO)=\det E = 1 \Rightarrow \det O = \pm1
\end{equation*}
Строки и столбцы умноженные сами на себя дают единицу, а перемноженные попарно дают нуль:
\begin{equation*}
	\sum_{\alpha=1}^{3}\tensor{O}{^\alpha_\beta}^2=1,\quad\sum_{\alpha=1}^{3}\tensor{O}{^\alpha_\beta}\tensor{O}{^\alpha_\gamma}=1,\quad \alpha,\beta,\gamma=1,2,3.
\end{equation*}

Ортогональный тензор всегда имеет одно действительное собственное значение, равное 1, и два, вообще говоря, комплексных.


  \que{Векторное произведение, его свойства, символы Леви-Чивиты.}

\begin{definition}[Символ Леви-Чивиты]
	\begin{equation*}
		\epsilon_{ijk},\epsilon^{ijk} = \begin{cases}
			0, &\text{если среди $i,j,k$ есть повторения};\\
			1, &\text{если $(ijk)$ - четная подстановка};\\
			-1,&\text{если $(ijk)$ - четная подстановка}.\\
		\end{cases}
	\end{equation*}
\end{definition}

\begin{definition}[Определитель]
	Для матрицы 3х3 верна формула:
	\begin{equation*}
		\det(\tensor{A}{^i_j}) = \frac{1}{6}\epsilon_{ijk}\epsilon^{mnl}\tensor{A}{^i_m}\tensor{A}{^j_n}\tensor{A}{^k_l}.
	\end{equation*}
	(легко можно проверить подставив в лоб)
	
	Расписав покомпонентно один из символов Леви-Чевиты и упростив легко выделить другое выражение
	\begin{equation*}
		\det(\tensor{A}{^i_j}) = \epsilon_{ijk}\tensor{A}{^i_\alpha}\tensor{A}{^j_\beta}\tensor{A}{^k_\gamma},\quad \alpha\neq\beta\neq\gamma,\quad\alpha,\beta,\gamma=1,2,3.
	\end{equation*}
\end{definition}

Для метрической матрицы $g_{ij}, g^{ij}$ формулы выше также применимы, тогда получим, что:
\begin{align*}
	g &= \frac{1}{6}\epsilon^{ijk}\epsilon^{mnl}g_{mi}g_{nj}g_{kl},\\
	\sqrt{g}\epsilon_{ijk}&=\frac{1}{\sqrt{g}}\epsilon^{mnl}g_{mi}g_{nj}g_{lk}.
\end{align*}

\begin{definition}[Векторное произведение]
	Векторным произведением векторов $\mathbf{a}, \mathbf{b}$ из $\mathcal{E}_3$ называют следующий вектор $\mathbf{c}$ из $\mathcal{E}_3$:
	\begin{equation*}
		\mathbf{c} = \mathbf{a} \times \mathbf{b} = \sqrt{g} \epsilon_{ijk} a^ib^j\mathbf{e}^k = \frac{1}{\sqrt{g}}\epsilon^{ijk} a_ib_j\mathbf{e}_k
	\end{equation*}
\end{definition}

\begin{theorem}
	Для векторного произведения имеет место формула 
	\begin{equation*}
		\mathbf{a} \times \mathbf{b} = S\mathbf{n},
	\end{equation*}
	\begin{equation*}
		S = |\mathbf{a}||\mathbf{b}|\sin{\varphi},
	\end{equation*}
	где $\varphi$ угол между $\mathbf{a}$ и $\mathbf{b}$ ($0\leqslant\varphi\leqslant\pi$), а $\mathbf{n}$ единичный вектор, ортогональный к $\mathbf{a}$ и $\mathbf{b}$. Иллюстрация, см. Рисунок \ref{fig:que5}.
	
	\begin{proof}
		Если один из векторов $\mathbf{a}$ и $\mathbf{b}$ нулевой, то формула, очевидно, выполняется. Если векторы $\mathbf{a}$ и $\mathbf{b}$ коллинеарны, то же самое. Рассмотрим случай, когда $\mathbf{a}$ и $\mathbf{b}$ ненулевые и неколлинеарные.
		
		$\mathbf{a}$ и $\mathbf{b}$ ненулевые и неколлинеарные. Построим базис $\mathbf{e}'_i$ в $E_2$:
		\begin{equation*}
			\mathbf{e}'_1 = \mathbf{a},\quad\mathbf{e}'_2=\mathbf{b},\quad\mathbf{e}'_3=\mathbf{n}
		\end{equation*}
		Компоненты векторов $\mathbf{a}$ и $\mathbf{b}$ в этом базисе:
		\begin{equation*}
			a'i=\begin{pmatrix}
				1\\0\\0
			\end{pmatrix},\quad
			b'i=\begin{pmatrix}
				0\\1\\0
			\end{pmatrix},
		\end{equation*}
		а метрическая матрица:
		\begin{equation*}
			g'_{ij}=\mathbf{e}'_i\cdot\mathbf{e}'_j=\begin{pmatrix}
				|\mathbf{a}|^2            & \mathbf{a}\cdot\mathbf{b} & 0 \\
				\mathbf{a}\cdot\mathbf{b} & |\mathbf{b}|^2            & 0 \\
				0                         & 0                         & 1
			\end{pmatrix}
		\end{equation*}
		Тогда
		\begin{equation*}
			S=\sqrt{g'_{11}g'_{22}}\sin{\varphi}=\sqrt{g'_{11}g'_{22}}\sqrt{1-\cos^2{\varphi}},
		\end{equation*}
		где
		\begin{equation*}
			\cos\varphi = \frac{\mathbf{a}\cdot\mathbf{b}}{|\mathbf{a}||\mathbf{b}|}=\frac{g'_{12}}{\sqrt{g'_{11}g'_{22}}}, \quad\sqrt{1-\cos^2{\varphi}} = \sqrt{\frac{g'}{g'_{11}g'_{22}}}.
		\end{equation*}
		Таким образом
		\begin{equation*}
			S = \sqrt{g'_{11}g'_{22}}\sqrt{\frac{g'}{g'_{11}g'_{22}}} = \sqrt{g'},
		\end{equation*}
		При этом имеем
		\begin{equation}\label{vec prod}
			\mathbf{a} \times \mathbf{b} = \sqrt{g'} \epsilon_{123} a'^1b'^2\mathbf{e}'^3=\sqrt{g'}e'^3
		\end{equation}
		Подставляя в \eqref{vec prod} всё имеющееся, действительно получаем
		\begin{equation*}
			\mathbf{a} \times \mathbf{b} = S\mathbf{n}.
		\end{equation*}
	\end{proof}
\end{theorem}
\begin{figure}[ht!]
	\centering
	\includegraphics[width=0.5\linewidth]{img/que5}
	\caption{Геометрическое изображение векторного произведения в пространстве $E_3$}
	\label{fig:que5}
\end{figure}
\begin{theorem}[О связи векторов взимного и основного базисов в $\mathcal{E}_3$]
	Векторы взимного и основного базисов в $\mathcal{E}_3$ связаны с помощью операции векторного произведения:
	\begin{equation*}
		\mathbf{e}^\gamma = \frac{1}{\sqrt{g}}\mathbf{e}_\alpha\times\mathbf{e}_\beta,\quad\mathbf{e}_\gamma=\sqrt{g}\mathbf{e}^\alpha\times\mathbf{e}^\beta,\quad \alpha\neq\beta\neq\gamma,\quad\alpha,\beta,\gamma=1,2,3.
	\end{equation*}
	\begin{proof}
		\begin{equation*}
			\mathbf{e}_n\times\mathbf{e}_m=\sqrt{g}\epsilon_{ijk}\delta^i_n\delta^j_m\mathbf{e}^k=\sqrt{g}\epsilon_{nmk}\mathbf{e}^k
		\end{equation*}
	\end{proof}
\end{theorem}
  \que{Ковариантное дифференцирование тензоров.}

\paragraph{Кратко}
Основные операции с наблой:

\begin{itemize}
	\item с скаляром:
	\begin{itemize}
		\item $\nabla \varphi=\mathbf{r}_i\frac{\partial\varphi}{\partial X^i}=\bar{\mathbf{e}}_i\frac{\partial \varphi}{\partial x^i}$ - градиент скаляра --- вектор;
	\end{itemize}
	\item с вектором:
	\begin{itemize}
		\item $\nabla\cdot\mathbf{a}=\nabla_ja^j$ - дивергенция вектора --- скаляр;
		\item $\nabla\times\mathbf{a}=\frac{1}{\sqrt{g}}\epsilon^{ijk}\nabla_ja_j\mathbf{r}_k$ - ротор вектора --- вектор;
		\item $\nabla\otimes\mathbf{a}=\mathbf{r}^i \otimes \frac{\partial \mathbf{a}}{\partial X^i} $ - тензорное произведение;
	\end{itemize}
	\item с тензором:
	\begin{itemize}
		\item $\nabla\cdot\mathbf{T}=\nabla_iT^{ij}\mathbf{r}_j$ - дивергенция тензора --- вектор;
	\end{itemize}
\end{itemize}

\begin{definition}[Кристоффель]
	\begin{equation*}
		\Gamma^{m}_{ij}=\frac{1}{2}g^{km}\left(
		\frac{\partial g_{kj}}{\partial X^i} + \frac{\partial g_{ki}}{\partial X^j} -
		\frac{\partial g_{ij}}{\partial X^k}
		\right)
	\end{equation*}
\end{definition}

\begin{definition}[Производные]
	
	Ковариантные:
	\begin{equation*}
		\nabla_ka_i=\frac{\partial a_i}{\partial X^k} - \Gamma^m_{ik}a_m,\quad
		\nabla_ka^i=\frac{\partial a^i}{\partial X^k} + \Gamma^i_{km}a^m.
	\end{equation*}
	Контравариантные:
	\begin{equation*}
		\nabla^ka^i=g^{km}\nabla_ma^i.
	\end{equation*}
\end{definition}


\paragraph{Из лекций}

Введем набла оператор в $\mathcal{K}$ -- символический дифференциальный оператор:
\[
\nabla \equiv \mathbf{r}^i \frac{\partial }{\partial X^i} 
= \mathbf{r}^1 \frac{\partial }{\partial X^1} + \mathbf{r}^2 \frac{\partial }{\partial X^2} 
+ \mathbf{r}^3 \frac{\partial }{\partial X^3};
\]

Применение его:
\begin{enumerate}
	\item К скаляру: градиент скаляра -- вектор
	\[
	\nabla \varphi = \mathbf{r}^i \frac{\partial \varphi}{\partial X^i}.
	\]
	Из свойств отметим, что он инвариантен, то есть 
	\[
	\nabla \varphi = \bar{\mathbf{e}}^i \frac{\partial \varphi}{\partial x^i} 
	= \frac{\partial \varphi}{\partial x^1} \bar{\mathbf{e}}_1 + \frac{\partial \varphi}{\partial x^2} \bar{\mathbf{e}}_2 + \frac{\partial \varphi}{\partial x^3} \bar{\mathbf{e}}_3.
	\]
	
	\item К вектору:
	\begin{itemize}
		\item тензорно: градиент вектора -- тензор второго ранга:
		\[
		\nabla \otimes \mathbf{a}
		= \mathbf{r}^i \frac{\partial }{\partial X^i} \otimes \mathbf{a}
		= \mathbf{r}^i \otimes \frac{\partial \mathbf{a}}{\partial X^i} 
		\]
		Рассмотрим
		\[
		\frac{\partial \mathbf{a}}{\partial X^i} 
		= \frac{\partial a^j \mathbf{r}_j}{\partial X^i} 
		= \frac{\partial a^j}{\partial X^i} \mathbf{r}_j
		+ a^j \frac{\partial \mathbf{r}_j}{\partial X^i} 
		= \left( \frac{\partial a^k}{\partial X^i} + a^j \Gamma^k_{ji} \right) \mathbf{r}_k,
		\]
		где введено обозначение:
		$ \frac{\partial \mathbf{r}_j}{\partial X^i} 
		= \Gamma^k_{ji} \mathbf{r}_k $
		-- символы Кристоффеля.
		
		Обозначим: $\nabla_i a^k \equiv \frac{\partial a^k}{\partial X^i} + a^j \Gamma^k_{ji}$
		-- ковариантная производная от контравариантных компонент вектора.
		Тогда $ \frac{\partial \mathbf{a}}{\partial X^i} = (\nabla_i a^k) \mathbf{r}_k$.
		
		Если $\mathbf{r}_i \equiv \bar{\mathbf{e}}_i$, то $\Gamma^k_{ji} \equiv 0$. То есть
		ковариантная производная совпадёт с частной производной.
		
		Подставим полученное в начало:
		\[
		\nabla \otimes \mathbf{a} = \mathbf{r}^i \otimes (\nabla_i a^k) \mathbf{r}_k 
		= (\nabla_i a^k) \mathbf{r}^i \otimes \mathbf{r}_k
		\]
		Компоненты этого тензора в смешанном диадном локальном базисе. (Также связь между градиентом и ковариантоной производной.)
		
		\textbf{Теорема Риччи}: $\nabla_i g_{jk} \equiv 0$, из нее следует, что можно опускать и 
		поднимать индексы под знаком контравариантной производной.
		
		\[
		\nabla^i a_k \equiv g^{ij} \nabla_j a_k
		\]
		-- контравариантая производная от ковариантных компонент вектора.
		
		\[
		\nabla \otimes \mathbf{a}
		= (\nabla_i a^k) \mathbf{r}^i \otimes \mathbf{r}_k
		= (\nabla^i a^k) \mathbf{r}_i \otimes \mathbf{r}_k
		\]
		
		Отметим, что $ \frac{\partial a_k}{\partial X^i} $ -- не являются компонентами какого-то
		тензора, но $\nabla_i a_k$, $\nabla^i a^k$ -- являются компонентами тензора 2-го ранга.
		
	\end{itemize}
\end{enumerate}


  \que{Физические  компоненты тензоров.}

Если криволинейные координаты $X^i$ являются ортогональными, то векторы $\mathring{\mathbf{r}}_i$ -- ортогональны: $ \mathring{\mathbf{r}}_i\cdot\mathring{\mathbf{r}}_j=\delta_{ij}$, а матрицы $\mathring{g}_ij$ и $\mathring{g}^ij$ -- диагональные. Тогда можно ввести параметры Ламе (см. подробнее раздел \ref{Локальные базисы и метрические матрицы в конфигурациях} параграф \nameref{Локальные базисы и метрические матрицы в конфигурациях}): $\mathring{H}_\alpha=\sqrt{\mathring{g_{\alpha\alpha}}},\,\alpha=1,2,3$, и \textit{физический ортонормированный базис}:
\begin{equation*}
	\widehat{\mathring{\mathbf{r}}}_\alpha=\frac{\mathring{\mathbf{r}}_\alpha}{\mathring{H}_\alpha}=\mathring{\mathbf{r}}^\alpha\mathring{H}_\alpha.
\end{equation*}
Компоненты вектора $\mathbf{a}$ в этом базисе называют физическими:
\begin{equation*}
	\mathbf{a} = \widehat{\mathring{a}}^{\,i}~\widehat{\mathring{r}}_i.
\end{equation*}

Актуальный базис $\mathbf{r}_i$ в общем случае не является ортогональным, даже если $\mathring{\mathbf{r}}_i$ -- ортогональный, поэтому нельзя ввести соответствующий ему физический базис в $\mathcal{K}$. Заметим однако, что в $\mathcal{K}$ всё-таки вводят физический базис, но по другому ... (\textit{взято из димитриенки}).

  % Кинематика сплошных сред
  \que{Лагранжево и эйлерово описания движения сплошных сред. Актуальная  и отсчетные конфигурации. Локальные базисы и метрические матрицы   в конфигурациях.}
\paragraph{Лагранжево и эйлерово описание.} Пусть дана сплошная среда $ \mathcal
P$ и некоторая её материальная точка $ \mathcal
M$ с радиусом-вектором $ \mathbf{x} $. 

Движение этой материальной точки может
рассматриваться в обычном декартовом базисе $ O\mathbf{\bar{e}}_i $, то есть как
вектор-функция $ \mathbf{x}(t) =
x^i(t)\mathbf{\bar{e}}_i  $. Такой подход к описанию движения называют
\emph{Эйлеровым}.

Другой подход состоит в том, чтобы связать с телом $ \mathcal{P} $ систему
криволинейных координат $ X^i $, которая будет <<двигаться>> вместе с телом
так, что координаты $ X^i $ точки $ \mathcal M $ в любой момент времени будут
одни и те же. Эти криволинейные координаты принадлежат телу только в случае,
если принадлежат некоторой области изменения, $ X^k \in V_X $. Этот подход
называется \emph{Лагранжевым}.

От криволинейных координат требуется регулярность, поэтому существуют следующие
соотношения: 
\[
  \mathbf{x} = \mathbf{x}(X^k, t), \qquad X^k = X^k(x^i, t).
\]
Здесь под $ X^k $ подразумевается полный набор криволинейных координат (в
трёхмерном пространстве их количество варьируется от одной до трёх).
Естественно, эти функции определены, только если $ X^k \in V_X $, $ \mathbf x \in
\mathcal P(t) $.

Соответственно, скалярные, векторные и тензорные поля (в том числе переменные)
тоже можно раскрывать как функции
криволинейных координат и времени либо же декартовых координат и времени. При
этом для твёрдых тел чаще используют лагранжево описание (следят, как меняется
фиксированная точка тела), а для жидкостей и газов --- Эйлерово (следят, как
материальные точки тела <<проходят>> через фиксированную точку пространства).

\paragraph{Актуальная и отсчётная конфигурации.} Для сплошной среды $ \mathcal P
$ определены отображения в точечно-евклидово пространство (типа строим мат. модель) $
V = V(\mathcal P, t)$, причём $ V $ является замкнутым множеством без
изолированных точек (т. н. \emph{совершенное} тело). 

Собственно, \emph{отсчётной} конфигурацией называется множество $ \mathring V =
V(\mathcal P,
0)$, а \emph{актуальной} конфигурацией --- множество $ V = V(\mathcal P, t_0) $ для
некоторого интересующего нас момента $ t_0 > 0 $.

\paragraph{Локальные базисы и метрические матрицы в конфигурациях.}\label{Локальные базисы и метрические матрицы в конфигурациях} С
криволинейными координатами $ X^i $ (точнее, с обратным отображением $ \mathbf
x(X^i) $) связана \emph{матрица Якоби} $
\tensor{Q}{^i_j} = \left( \frac{\partial x^i}{\partial X^j}\right)  $. Причём
предполагается, что Якобиан ($ \det Q $) отличен от нуля (условия регулярности),
то есть имеет обратную матрицу, которую назовём $ \tensor{P}{^i}{_j} $.
Заметим, что, вообще говоря\footnote{Если криволинейные координаты не являются
прямолинейными, то есть преобразование не линейное.}, эта матрица зависит от точки тела $ \mathcal
P$ и от времени (поскольку сами отображения $ \mathbf{x}(X^i, t) $ зависят в том
числе от времени).

Для фиксированного момента времени и точки $ \mathcal M \in \mathcal P $ назовём
столбцы этой матрицы \emph{локальным базисом} в точке $ \mathcal M $, обозн. $
\mathring{\mathbf{r}}_k $ для нулевого момента времени и $ \mathbf{r}_k $ иначе. 

Каждому локальному базису соответствует \emph{метрический тензор} $ g_{ij} =
\mathbf{r}_i \cdot \mathbf{r}_j $ (аналогично для $ \mathring g_{ij} $, далее не
подчёркивается) и
\emph{обратный метрический тензор} $ g^{ij} = g^{-1}_{ij} $. После последнего
определения нам стало доступно <<жонглирование>> индексами, в том числе
локального базиса. Именно, определим \emph{взаимный локальный базис $
\mathbf{r}^i $} по формуле $ \mathbf{r}^i = g^{ij}\mathbf{r}_j $.

Частный случай, когда векторы $ \mathbf{r}_i $ ортогональны, то есть матрица $
g_{ij} $ диагональна, дополняется определением \emph{коэффициентов Ламэ} $
H_\alpha $ как $ H_\alpha = \sqrt{g_{\alpha\alpha}} $, то есть длины
соответствующего базисного вектора.

  \que{Градиент деформаций. Тензоры деформации Альманзи и Коши-Грина. }


  \que{Физический  смысл компонент тензора  деформаций.  Преобразование
ориентированной площадки при деформации сплошной среды. Геометрическая картина
преобразования малой окрестности.}
\paragraph{Физический смысл.} Имеем  
\[
  \varepsilon_{\alpha\beta} =
  \frac{1}{2}(|\mathbf{r}_\alpha||\mathbf{r}_\beta|\cos\psi_{\alpha\beta} -
  |\mathring{\mathbf{r}}_\alpha||
  \mathring{\mathbf{r}}_\beta|\cos\mathring{\psi}_{\alpha\beta}).
\]
Далее нестандартный анализ. Введём длины элементарных радиусов-векторов: $ ds^2
= d\mathbf{x}\cdot d\mathbf{x}$, $ d\mathring{s}^2 = d\mathring{\mathbf{x}}\cdot
d\mathring{\mathbf{x}}$.

При этом $ d\mathring{\mathbf{x}} $ будем выбирать вдоль одного из векторов
$\mathring{\mathbf{r}}_\alpha$. Тогда $ d\mathbf{x} $ тоже будет направлен вдоль
$ \mathbf{r}_\alpha $. В этом случае  
\begin{align*}
  |d\mathring{\mathbf{x}}| &= d\mathring{s}_\alpha = \left| \frac{\partial
  \mathring{\mathbf{x}}}{\partial X^\alpha} dX^\alpha \right| =
  |\mathring{\mathbf{r}}_\alpha | dX^\alpha,\\
    |d\mathbf{x}| &= ds_\alpha = \left| \frac{\partial x}{\partial
  X^\alpha}dX^\alpha \right| = |\mathbf{r}_\alpha | dX^\alpha.
\end{align*}
Отсюда находим  
\[
  ds_\alpha/d\mathring{s}_\alpha =
  |\mathbf{r}_\alpha|/|\mathring{\mathbf{r}}_\alpha| = \delta_\alpha + 1,
\]
где $ \delta_\alpha $ называют \emph{относительным удлинением}. Получаем  
\[
  |\mathring{\mathbf{r}}_\alpha| = |\mathring{\mathbf{r}}_\alpha|(1 +
  \delta_\alpha)
\]
и  
\begin{align*}
  \varepsilon_{\alpha\beta} &= \frac{1}{2}
  |\mathring{\mathbf{r}}_\alpha||\mathring{\mathbf{r}}_\beta|
  ((1+\delta_\alpha)(1+\delta_\beta)\cos\psi_{\alpha\beta} -
  \cos\mathring{\psi}_{\alpha\beta}), \\
  \varepsilon_{\alpha\alpha} &=
  \frac{1}{2}|\mathring{\mathbf{r}}_\alpha|^2((1+\delta_\alpha)^2 - 1) =
  \frac{\mathring{g}_{\alpha\alpha}}{2}((1+\delta_\alpha)^2 - 1).
\end{align*}

Если $ X^i $ есть декартовы координаты, то $ \mathring{g}_{\alpha\beta} =
\delta_{\alpha\beta} $, и при $ \delta_\alpha \ll 1 $ имеем $
\varepsilon_{\alpha\alpha} \approx \delta_\alpha $. При этом  
\[
  \varepsilon_{\alpha\beta} =
  \frac{1}{2}(1+\delta_\alpha)(1+\delta_\beta)\sin\chi_{\alpha\beta},
\]
где $ \chi_{\alpha\beta} = \mathring{\psi}_{\alpha\beta} - \psi_{\alpha\beta} =
(\pi/2) - \psi_{\alpha\beta}$ --- изменение угла между базисными векторами $
\mathbf{r}_\alpha $ и $ \mathbf{r}_\beta $. При $ \delta_\alpha \ll 1 $, $
\chi_{\alpha\beta} \ll 1 $ имеем $ \varepsilon_{\alpha\beta} \approx
\chi_{\alpha\beta}/2 $.

\paragraph{Преобразование ориентированной площадки.} Далее какой-то бред.
Рассмотрим некоторую гладкую поверхность $ \Sigma $, которой принадлежат
какие-либо две из координатных линий $ X^\alpha $ и $ X^\beta $.

Тогда можно ввести вектор нормали  
\[
  \mathbf{n} = \frac{1}{\sqrt{\tilde{g}}} \mathbf{r}_\alpha \times
  \mathbf{r}_\beta,
\]
где $ \tilde g $ --- определитель метрической матрицы, составленной только из
выбранных векторов $ \mathbf{r}_\alpha $, $ \mathbf{r}_\beta $. Эта нормаль
является единичной.

Рассмотрим элементарную площадку $ d\Sigma $, построенную на элементарных
радиусах-векторах $ d\mathbf{x}_\alpha $, направленных по векторам локального
базиса, то есть $ d\mathbf{x}_\alpha = \mathbf{r}_\alpha dX^\alpha $. Назовём
величину  
\[
  d\Sigma = \sqrt{\tilde g} dX^\alpha dX^\beta
\]
\emph{площадью элементарной площадки} $ d\Sigma $ (обозначение супер),
построенной на векторах $ d\mathbf{x}_\alpha $, $ d\mathbf{x}_\beta $. Тогда  
\[
  \mathbf{n} d\Sigma = \mathbf{r}_\alpha dX^\alpha \times \mathbf{r}_\beta
  dX^\beta = d\mathbf{x}_\alpha \times d\mathbf{x}_\beta.
\]
Величину $ \mathbf{n}d\Sigma $ называют \emph{ориентированной площадкой}.

Площадке $ d\Sigma $ соответствует площадка $ d\mathring{\Sigma} $, построенная
на $ d\mathring{\mathbf{x}}_\alpha $, $ d\mathring{\mathbf{x}}_\beta$, 
\[
  \mathring{\mathbf{n}}d\mathring{\Sigma} = \mathring{\mathbf{r}}_\alpha
  dX^\alpha \times \mathring{\mathbf{r}}_\beta dX^\beta =
  \mathring{\mathbf{r}}_\alpha \times \mathring{\mathbf{r}}_\beta dX^\alpha
  dX^\beta.
\]

Так как $ \mathbf{r}^\gamma = F^{-1\mathsf T}\cdot \mathring{\mathbf{r}}^\gamma
$, то  
\begin{equation}\label{eq:govno}
  \mathbf{n}d\Sigma = \sqrt{g}\epsilon_{\alpha\beta\gamma}F^{-1\mathsf T} \cdot
  \mathring{\mathbf{r}}^\gamma dX^\alpha dX^\beta =
  \sqrt{g/\mathring{g}} F^{-1 \mathsf T}\cdot
  \mathring{\mathbf{r}}_\alpha\times\mathring{\mathbf{r}}_\beta dX^\alpha
  dX^\beta = 
  \sqrt{g/\mathring{g}} F^{-1\mathsf
  T}\cdot \mathring{\mathbf{n}}d\mathring{\Sigma}.
\end{equation}
Умножим это уравнение скалярно само на себя и получим  
\[
  d\Sigma^2 = \frac{g}{\mathring{g}}(\mathring{\mathbf{n}}\cdot F^{-1}\cdot
  F^{-1\mathsf T}\cdot \mathring{\mathbf{n}})d\mathring{\Sigma}^2 =
  \frac{g}{\mathring g}(\mathring{\mathbf{n}}\cdot G^{-1}\cdot
  \mathring{\mathbf{n}})d\mathring{\Sigma}^2,
\]
то есть 
\[
  d\Sigma/d\mathring{\Sigma} = \sqrt{g/\mathring{g}}(\mathring{\mathbf{n}}\cdot
  G^{-1}\cdot \mathring{\mathbf{n}})^{1/2}.
\]

С другой стороны, из той же формулы \eqref{eq:govno} можно сначала выразить $
\mathring{\mathbf{n}} $, а затем получившееся соотношение умножить скалярно само
на себя. Тогда  
\[
  d\mathring{\Sigma}^2 = \frac{\mathring{g}}{g}(\mathbf{n}\cdot F\cdot
  F^{\mathsf T}\cdot \mathbf{n})d\Sigma^2.
\]
Отсюда  
\[
  d\mathring{\Sigma}/d\Sigma = \sqrt{\mathring g/g}(\mathbf{n}\cdot g^{-1}\cdot
  \mathbf{n})^{1/2}.
\]
Иными словами,  
\[
  (\mathbf{n}\cdot g^{-1}\cdot \mathbf{n})^{1/2} = (\mathring{\mathbf{n}}\cdot
  G^{-1}\cdot \mathring{n})^{-1/2}.
\]
После подстановки полученных выражений в \eqref{eq:govno} получаем  
\[
    (\mathring{\mathbf{n}}\cdot
    G^{-1}\cdot \mathring{n})^{1/2} \mathbf{n} = F^{-1\mathsf T}\cdot
    \mathring{\mathbf{n}}, \qquad (\mathbf{n}\cdot g^{-1}\cdot \mathbf{n})^{1/2}
    \mathring{\mathbf{n}} = F^{\mathsf T} \cdot \mathbf{n}.
\]

\paragraph{Геометрическая картина преобразования малой окрестности.} 
Известно соотношение
\[ 
  d \mathbf{x}=\mathbf{F} \cdot d\mathring{\mathbf{x}}.
\]

Запишем это соотношение в декартовых координатах
\[ d x^{i}=\bar{F}_{m}^{i} d \mathring{x}^{m}, \]
где $\bar{F}_{m}^{i}$ - компоненты градиента деформации в декартовом базисе:
\[ \bar{F}_{m}^{i}=\left(\partial x^{i} / \partial \mathring{x}^m\right). \]
Фактически, это аффинное преобразование.
Из общих свойств аффинных преобразований следует, параллелограммы при аффинном
преобразовании перейдут в параллелограммы, а сферы перейдут в эллипсоиды.

Отношение длин $d s_{\alpha} / d\mathring{s}$
произвольного отрезка (элементарного радиуса-вектора $d \mathbf{x}$ в
$\mathcal{K}$ и $\mathring{\mathcal{K}}$ ) не зависит от начальной
длины $d \mathring{s}$ этого отрезка (так как относительное удлинение
$\delta_{\alpha}$ не зависит от $d s_{\alpha}$ ).

Согласно полярному разложению, указанное преобразование из $\mathring{\mathcal{K}}$ в
$\mathcal{K}$ всегда можно представить суперпозицией двух преобразований: 
\[
  d\mathbf{x} = O \cdot d\mathring{\mathbf{x}}', \qquad d\mathring{\mathbf{x}}'
  = U \cdot d\mathring{\mathbf{x}},
\]
или

\[ d \mathbf{x}=\mathbf{V} \cdot d \mathbf{x}^{\prime}, \quad d
\mathbf{x}^{\prime}=\mathbf{O} \cdot d \mathbf{x} \]

Тензор искажений $\mathbf{U}$, обладающий тремя собственными направлениями
$\mathring{\mathbf{p}}_{\alpha}$, изменяет малую окрестность точки
$\mathcal{M}$, сжимая или растягивая ее вдоль этих трех направлений
$\mathring{\mathbf{p}}_{\alpha}$. Тензор поворота $ O $ поворачивает
деформированную вдоль $\mathring{\mathbf{p}}_{\alpha}$ окрестность
<<жестким образом>>, переводя направление $\mathring{\mathbf{p}}_{\alpha}$
в $\mathbf{p}_{\alpha}$. Если используем левый тензор искажений $\mathbf{V}$, то
вначале осуществляется поворот осей $\mathring{\mathbf{p}}_{\alpha}$ в
$\mathring{\mathcal{K}}$ до их совпадения с $\mathbf{p}_{\alpha}$ (с
точностью до параллельного переноса), а затем сжатие/растяжение малой
окрестности вдоль направления $\mathbf{p}_{\alpha}$. Результат, очевидно, будет
одинаковым.

Если точка $\mathcal{M}_{\alpha}$ связана с $\mathcal{M}$ радиусом-вектором $d
\mathbf{x}_{\alpha}$, ориентированным по собственному направлению
$\mathring{\mathbf{p}}_{\alpha}$ (заметим, заранее до деформации
неизвестному), то в $\mathcal{K}$ эта точка $\mathcal{M}_{\alpha}$ будет связана
с $\mathcal{M}$ радиусом-вектором $d \mathbf{x}_{\alpha}$, ориентированным вдоль
соответствующего собственного направления $\mathbf{p}_{\alpha}$.

Если малую окрестность точки $\mathcal{M}$ в $\mathring{\mathcal{K}}$
взять в виде сферы, то в $\mathcal{K}$ эта сфера перейдет в
эллипсоид, главные оси которого направлены по собственным направлениям
$\mathbf{p}_{\alpha}$.

Таким образом, преобразование малой окрестности каждой точки $\mathcal{M}$
сплошной среды при деформации всегда можно представить в виде растяжения/сжатия
вдоль собственных направлений и поворота как жесткого целого, а также
перемещения как жесткого целого.

  \que{Полярное разложение, тензоры искажений и поворота. Собственные значения и собственные вектора тензоров искажений.}
Очевидно, тензор $ \mathbf{F} $ невырожден.

\begin{theorem*}
Всякий невырожденный тензор второго ранга
$\mathbf{F}$ можно представить в виде скалярного произведения двух тензоров
второго ранга: \[
\mathbf{F}=\mathbf{O} \cdot \mathbf{U} \quad \text { или } \quad
\mathbf{F}=\mathbf{V} \cdot \mathbf{O}, 
\]
где $ \mathbf{U} $ и $ \mathbf{V} $ --- симметричные, положительно определенные
тензоры, а $ \mathbf{O} $ --- ортогональный тензор, причем каждое из
представлений единственное.
\end{theorem*}
\begin{proof}
Построим тензоры $\mathbf{U}, \mathbf{V}$ и $ \mathbf{O} $. Для этого рассмотрим свертки
тензора $\mathbf{F}$ со своим транспонированным: $\mathbf{F}^{{T}} \cdot
\mathbf{F}$ и $\mathbf{F} \cdot \mathbf{F}^{{T}}$. Оба эти тензора
являются симметричными, так как
\[
\left(\mathbf{F}^{\mathsf{T}} \cdot
\mathbf{F}\right)^{\mathsf{T}}=\mathbf{F}^{\mathsf{T}}
\cdot\left(\mathbf{F}^{\mathsf{T}}\right)^{\mathsf{T}}=\mathbf{F}^{\mathsf{T}}
\cdot \mathbf{F} \text { и }\left(\mathbf{F} \cdot
\mathbf{F}^{\mathsf{T}}\right)^{\mathsf{T}}=\left(\mathbf{F}^{\mathsf{T}}\right)^{\mathsf{T}}
\cdot \mathbf{F}^{\mathsf{T}}=\mathbf{F} \cdot \mathbf{F}^{\mathsf{T}} \]
а также положительно определенными:
\[
\mathbf{a} \cdot\left(\mathbf{F}^{\mathsf{T}} \cdot \mathbf{F}\right) \cdot
\mathbf{a}=\left(\mathbf{a} \cdot \mathbf{F}^{\mathsf{T}}\right)
\cdot(\mathbf{F} \cdot \mathbf{a})=(\mathbf{F} \cdot \mathbf{a})
\cdot(\mathbf{F} \cdot \mathbf{a})=\mathbf{b} \cdot
\mathbf{b}=|\mathbf{b}|^{2}>0 \]
для любого ненулевого вектора $\mathbf{a}$, где $\mathbf{b}=\mathbf{F} \cdot
\mathbf{a}$. Но у всякого симметричного положительно определенного тензора все
три собственные значения вещественны и положительны тогда собственные значения
тензоров $\mathbf{F}^{\mathsf{T}} \cdot \mathbf{F}$ и $\mathbf{F} \cdot
\mathbf{F}^{\mathsf{T}}$ можно обозначить как $\lambda_{\alpha}^{2}$ и
$\lambda_{\alpha}^{2}$. Эти тензоры являются диагональными в собственных
базисах, т.е. имеют следующие представления:

\[
  \mathbf{F}^{\mathsf{T}} \cdot \mathbf{F}=\sum_{\alpha=1}^{3}
  \mathring{\lambda}_\alpha^2
\mathring{\mathbf{p}}_{\alpha} \otimes \mathring{\mathbf{p}}_{\alpha}, \quad
\mathbf{F} \cdot \mathbf{F}^{\mathsf{T}}=\sum_{\alpha=1}^{3}
\lambda_{\alpha}^{2} \mathbf{p}_{\alpha} \otimes \mathbf{p}_{\alpha},
\] 
где $\mathring{\mathbf{p}}_{\alpha}$ - соб́ственные векторы тензора
$\mathbf{F}^{\mathsf{T}} \cdot \mathbf{F}$, а $\mathbf{p}_{\alpha}-$ тензора
$\mathbf{F} \cdot \mathbf{F}^{\mathsf{T}}$, являющиеся вещественнозначными и
ортонормированными:
\[
\mathring{\mathbf{p}}_{\alpha} \cdot \mathring{\mathbf{p}}_{\beta}=\delta_{\alpha \beta}, \quad \mathbf{p}_{\alpha} \cdot \mathbf{p}_{\beta}=\delta_{\alpha \beta}
\]

Правые части представляют собой квадраты некоторых тензоров $\mathbf{U}$ и
$\mathbf{V}$, определенных как 
\[
  U = \sum_{\alpha=1}^3 \mathring{\lambda}_\alpha
  \mathring{\mathbf{p}}_\alpha\otimes \mathring{\mathbf{p}}_\alpha, \quad
  \mathring{\lambda}_\alpha > 0; \qquad V = \sum_{\alpha = 1}^3
  \lambda_\alpha\mathbf{p}_\alpha \otimes \mathbf{p}_\alpha, \quad
  \lambda_\alpha > 0,
\]
где знаки у $\lambda_{\alpha}$ выбираем всегда положительными.

При этом имеют место соотношения:

\[
\mathbf{F}^{\mathsf{T}} \cdot \mathbf{F}=\mathbf{U}^{2}, \quad \mathbf{F} \cdot \mathbf{F}^{\mathsf{T}}=\mathbf{V}^{2}
\]

Построенные тензоры V и U являются симметричными, что следует из формулы, а также положительно определенными, так как для любого ненулевого вектора а выполнено:
\[
\mathbf{a} \cdot \mathbf{U} \cdot \mathbf{a}=\sum_{\alpha=1}^{3} \mathring{\lambda}_{\alpha} \mathbf{a} \cdot \mathring{\mathbf{p}}_{\alpha} \otimes \mathring{\mathbf{p}}_{\alpha} \cdot \mathbf{a}=\sum_{\alpha=1}^{3} \mathring{\lambda}_{\alpha}\left(\mathbf{a} \cdot \mathring{\mathbf{p}}_{\alpha}\right)^{2}>0
\]

ввиду того, что $\mathring{A}_{\alpha}>0$. Аналогично доказываем положительную определенность тензора V.

Оба тензора $ \mathbf{V} $ и $\mathbf{U}$ невырождены, так как по условию
теоремы $\mathbf{F}$ невырожден, тогда
\[
(\operatorname{det} \mathbf{U})^{2}=\operatorname{det} \mathbf{U}^{2}=\operatorname{det}\left(\mathbf{F}^{\mathsf{T}} \cdot \mathbf{F}\right)=(\operatorname{det} \mathbf{F})^{2} \neq 0
\]

Тогда существуют обратные тензоры $\mathbf{U}^{-1}$ и $\mathbf{V}^{-1}$, с помощью которых можно построить еще два новых тензора
\[
\mathring{\mathbf{O}}=\mathbf{F} \cdot \mathbf{U}^{-1}, \quad \mathbf{O}=\mathbf{V}^{-1} \cdot \mathbf{F}
\]

являющихся ортогональными. В самом деле,
\[
\mathring{\mathbf{O}}^{\mathsf{T}} \cdot \mathring{\mathbf{O}}=\left(\mathbf{F}
\cdot \mathbf{U}^{-1}\right)^{\mathsf{T}} \cdot\left(\mathbf{F} \cdot
\mathbf{U}^{-1}\right)=\mathbf{U}^{-1} \cdot \mathbf{F}^{\mathsf{T}} \cdot
\mathbf{F} \cdot \mathbf{U}^{-1}=\mathbf{U}^{-1} \cdot \mathbf{U}^{2} \cdot
\mathbf{U}^{-1}=\mathbf{E},
\]
что по определению означает ортогональность тензора $ O $.

Таким образом, мы действительно построили тензоры $\mathbf{U}$ и $\mathring{\mathbf{O}}$, а также $\mathbf{V}$ и $\mathbf{O}$, произведение которых образует исходный тензор $\mathbf{F}$ :

\[
\mathbf{F}=\mathring{\mathbf{O}} \cdot \mathbf{U}=\mathbf{V} \cdot \mathbf{O},
\]

причем $\mathbf{U}$ и  - симметричные, положительно определенные, а $\mathbf{O}$ и $\mathbf{\text { O }}-$ ортогональные.

Покажем единственность каждого из разложений. Пусть противное, т.е. существует еще одно разложение, например,

\[
\mathbf{F}=\mathring{\mathbf{O}} \cdot \widetilde{\mathbf{U}}
\]

Но тогда

\[
\mathbf{F}^{\mathsf{T}} \cdot \mathbf{F}=\widetilde{\mathbf{U}}^{2}=\mathbf{U}^{2},
\]

откуда следует, что $\widetilde{\mathbf{U}}=\mathbf{U}$, так как разложение
тензора $\mathbf{F}^{\mathsf{T}} \cdot \mathbf{F}$ по собственному базису
единственно, а знаки у $\mathring{A}_{\alpha}$ и $\widetilde{\lambda}_{\alpha}$
по условию выбираем положительными. Совпадение $ U $ и $ \widetilde U $ влечёт
за собой совпадение $ \mathring{\widetilde{O}} $ и $ \mathring{O} $, так как
\[
  \mathring{\widetilde{\mathbf{O}}}=\mathbf{F} \cdot
  \widetilde{\mathbf{U}}^{-1}=\mathbf{F}
\cdot \mathbf{U}^{-1}=\mathring{\mathbf{O}},
\]
что и доказывает единственность разложения. Единственность разложения $\mathbf{F}=\mathbf{V} \cdot \mathbf{O}$ доказывается аналогично.

Нам осталось только показать, что ортогональные тензоры $ \mathring{O} $ и $ O $
совпадают. Для этого образуем тензор
\[
\mathbf{F} \cdot \mathring{\mathbf{O}}^{\mathsf{T}}=\mathring{\mathbf{O}} \cdot \mathbf{U} \cdot \mathring{\mathbf{O}}^{\mathsf{T}}
\]

Для этого тензора выполнено соотношение:
\[
  \mathring{\mathbf{O}} \cdot \mathbf{U} \cdot \mathring{\mathbf{O}}^{\mathsf{T}}=\mathbf{V} \cdot \mathbf{O} \cdot \mathring{\mathbf{O}}^{\mathsf{T}} .
\]
Тензор $ \mathbf{O} \cdot \mathbf{O}^{\mathsf{T}}$ является ортогональным, так как

\[
\left(\mathbf{O} \cdot \mathring{\mathbf{O}}^{\mathsf{T}}\right)^{\mathsf{T}} \cdot\left(\mathbf{O} \cdot \mathring{\mathbf{O}}^{\mathsf{T}}\right)=\mathring{\mathbf{O}} \cdot \mathbf{O}^{\mathsf{T}} \cdot \mathbf{O} \cdot \mathring{\mathbf{O}}^{\mathsf{T}}=\mathring{\mathbf{O}} \cdot \mathring{\mathbf{O}}^{\mathsf{T}}=\mathbf{E}
\]

Тогда на соответствующее соотношение можно смотреть как на полярное разложение тензора $\mathring{\mathbf{O}} \cdot \mathbf{U} \cdot \mathbf{O}^{\mathsf{T}}$. Но этот тензор симметричен, так как
\[
\left(\mathring{\mathbf{O}} \cdot \mathbf{U} \cdot \mathring{\mathbf{O}}^{\mathsf{T}}\right)^{\mathsf{T}}=\left(\mathring{\mathbf{O}}^{\mathsf{T}}\right)^{\mathsf{T}} \cdot(\mathring{\mathbf{O}} \cdot \mathbf{U})^{\mathsf{T}}=\mathring{\mathbf{O}} \cdot \mathbf{U} \cdot \mathring{\mathbf{O}}^{\mathsf{T}}
\]

Тогда формальное равенство

\[
\mathring{\mathbf{O}} \cdot \mathbf{U} \cdot \mathring{\mathbf{O}}^{\mathsf{T}}=\mathring{\mathbf{O}} \cdot \mathbf{U} \cdot \mathring{\mathbf{O}}^{\mathsf{T}}
\]

--- еще одно его полярное разложение. Однако выше мы показали единственность полярного разложения, значит должны иметь место соотношения:

\[
\mathbf{V}=\mathring{\mathbf{O}} \cdot \mathbf{U} \cdot \mathring{\mathbf{O}}^{\mathsf{T}} \quad \text { и } \quad \mathbf{O} \cdot \mathring{\mathbf{O}}^{\mathsf{T}}=\mathbf{E},
\]

откуда и вытекает совпадение ортогональных тензоров $\mathbf{O}=\mathring{\mathbf{O}}$.
\end{proof}

Тензоры $\mathbf{U}$ и $\mathbf{V}$ называют правым и левым тензорами искажений
соответственно, а $ O $ --- тензором поворота, сопровождающего деформацию.

Тензор $\mathbf{F}$ имеет девять независимых компонент, тензор $\mathbf{O}$ ---
три независимые компоненты, а каждый из тензоров $\mathbf{U}$ и $\mathbf{V}$ ---
по шесть независимых компонент.

Из единственности тензора поворота О в полярном разложении
следует, что тензоры искажений $\mathbf{U}$ и $\mathbf{V}$ связаны друг с другом
с помощью тензора $\mathbf{O}$ :

\[
\mathbf{V}=\mathbf{O} \cdot \mathbf{U} \cdot \mathbf{O}^{\mathsf{T}}, \quad \mathbf{U}=\mathbf{O}^{\mathsf{T}} \cdot \mathbf{V} \cdot \mathbf{O}
\]

Тензоры деформации Коши -- Грина и Альманзи могут быть выражены через тензоры
искажений $\mathbf{U}$ и $ \mathbf{V} $ следующим образом: 
\[
\begin{aligned}
\mathbf{C} & =\frac{1}{2}\left(\mathbf{U}^{2}-\mathbf{E}\right), &
\mathbf{A}=\frac{1}{2}\left(\mathbf{E}-\mathbf{V}^{-2}\right) \\
\boldsymbol{\Lambda} & =\frac{1}{2}\left(\mathbf{E}-\mathbf{U}^{-2}\right), &
\mathbf{J}=\frac{1}{2}\left(\mathbf{V}^{2}-\mathbf{E}\right)
\end{aligned}
\]

\paragraph{Собственные значения и собственные базисы.}
\begin{theorem*}
  Собственные значения тензоров $\mathbf{U}$ и $ \mathbf{V} $ совпадают:

\[
\lambda_{\alpha}=\mathring{\lambda}_{\alpha}, \quad \alpha=1,2,3
\]

а собственные векторы $\mathring{\mathbf{p}}_{\alpha}$ и $\mathbf{p}_{\alpha}$ связаны тензором поворота, сопровождающим деформацию:

\[
\mathbf{p}_{\alpha}=\mathbf{O} \cdot \mathring{\mathbf{p}}_{\alpha}
\]
\end{theorem*}
\begin{proof} 
\[
\mathbf{V}=\sum_{\alpha=1}^{3} \lambda_{\alpha} \mathbf{p}_{\alpha} \otimes
\mathbf{p}_{\alpha}=\mathbf{O} \cdot \mathbf{U} \cdot
\mathbf{O}^{\mathsf{T}}=\sum_{\alpha=1}^{3} \mathring{A}_{\alpha} \mathbf{O} \cdot \mathring{\mathbf{p}}_{\alpha} \otimes\left(\mathbf{O} \cdot \mathring{\mathbf{p}}_{\alpha}\right)=\sum_{\alpha=1}^{3} \dot{\lambda}_{\alpha} \mathbf{p}_{\alpha}^{\prime} \otimes \mathbf{p}_{\alpha}^{\prime},
\]
где $\mathbf{p}_{\alpha}^{\prime}=\mathbf{O} \cdot
\mathring{\mathbf{p}}_{\alpha}$. Согласно этому соотношению, мы получили
два различных собственных базиса тензора $\mathbf{V}$ и два набора собственных
значений, что невозможно, следовательно,
$\mathbf{p}_{\alpha}^{\prime}=\mathbf{O} \cdot
\mathring{\mathbf{p}}_{\alpha}=\mathbf{p}_{\alpha}$ и
$\lambda_{\alpha}=\mathring{A}_{\alpha}$, что и требовалось доказать.
\end{proof}

Оба собственных базиса ортогональны, поэтому взаимные векторы собственных
базисов не отличаются от $\mathbf{p}_{\alpha}$ и $\mathring{\mathbf{p}}_{\alpha}$:
\[
\mathbf{p}_{\alpha}=\mathbf{p}^{\alpha}, \quad \mathring{\mathbf{p}}_{\alpha}=\mathring{\mathbf{p}}^{\alpha}
\]

Важным для приложений является вопрос о вычислении $\lambda_{\alpha}, \mathbf{p}_{\alpha}$ и $\mathring{\mathbf{p}}_{\alpha}$ по заданному градиенту деформации $\mathbf{F}$, для этого применяют следующую процедуру.

\begin{enumerate}
  \item Образуем тензор $\mathbf{U}^{2}=\mathbf{F}^{\mathsf{T}} \cdot
\mathbf{F}$ (или $\mathbf{V}^{2}=\mathbf{F} \cdot \mathbf{F}^{\mathsf{T}}$ ) и
найдем его компоненты в каком-либо подходящем для рассматриваемой задачи базисе,
например, в декартовом $\overline{\mathbf{e}}_{i}$:
\[
\mathbf{U}^{2}=\left(\bar{U}^{2}\right)^{i}{ }_{j} \overline{\mathbf{e}}_{i} \otimes \overline{\mathbf{e}}^{j} \quad \text { и } \quad \mathbf{V}^{2}=\left(\bar{V}^{2}\right)^{i}{ }_{j} \overline{\mathbf{e}}_{i} \otimes \overline{\mathbf{e}}^{j} .
\]

  \item Найдем собственные значения матрицы $\left(\bar{U}^{2}\right)^{i}{
    }_{j}$.

\item Найдем собственные векторы $\mathring{\mathbf{p}}_{\alpha}$ тензора $\mathbf{U}$ и векторы $\mathbf{p}_{\alpha}$ тензора $\mathbf{V}$ из следующих уравнений:
\[
\mathbf{U}^{2} \cdot \mathring{\mathbf{p}}_{\alpha}=\lambda_{\alpha}^{2}
\mathring{\mathbf{p}}_{\alpha}, \quad \mathbf{V}^{2} \cdot
\mathbf{p}_{\alpha}=\lambda_{\alpha}^{2} \mathbf{p}_{\alpha}.
\]
записанных, например, в базисе $\overline{\mathbf{e}}_{i}$ :
\[
\left(\left(\bar{U}^{2}\right)^{i}{ }_{j}-\lambda_{\alpha}^{2}
\delta_{j}^{i}\right) \mathring{\widehat{Q}^{j}}{ }_{\alpha}=0,
\quad\left(\left(\bar{V}^{2}\right)^{i}{ }_{j}-\lambda_{\alpha}^{2}
\delta_{j}^{i}\right) \widehat{Q}^{j}{ }_{\alpha}=0, 
\]
где $\widehat{Q}^{j}_{\alpha}$ и $\mathring{\widehat{Q}}^{j}_{\alpha}$ --- якобиевы матрицы собственных векторов:
\[
  \mathring{\mathbf{p}}_{\alpha}=\mathring{\widehat{Q}}^{j}{ }_{\alpha}
  \overline{\mathbf{e}}_{j}, \quad \mathbf{p}_{\alpha}=\widehat{Q}^{j}{
  }_{\alpha} \overline{\mathbf{e}}_{j}.
\]

В
качестве дополнительных уравнений присоединяют условия нормировки
\[
\left|\mathbf{p}_{\alpha}\right|=1,
\quad\left|\mathring{\mathbf{p}}_{\alpha}\right|=1б
\]
которые эквивалентны следующим квадратным уравнениям: 
\[
  \tensor{\mathring{\widehat{Q}}}{^i_\alpha}
  \tensor{\mathring{\widehat{Q}}}{^j_\alpha} \delta_{ij} = 1, \quad
  \tensor{\widehat{Q}}{^i_\alpha} \tensor{\widehat{Q}}{^j_\alpha}\delta_{ij}=1.
\]

  \item Составляем диадные произведения и находим представления тензоров $
    \mathbf{U} $ и $ \mathbf{V} $ в собственных базисах, записанных, например, для декартова базиса $\overline{\mathbf{e}}_{i}$ :
\[
  \mathbf{U}=\sum_{\alpha=1}^{3} \lambda_{\alpha}
  \mathring{\widehat{Q}}^{i}_{\alpha} \mathring{\widehat{Q}}^{j}_{\alpha}
\overline{\mathbf{e}}_{i} \otimes \overline{\mathbf{e}}_{j}, \quad
\mathbf{V}=\sum_{\alpha=1}^{3} \lambda_{\alpha} \widehat{Q}^{i}{ }_{\alpha}
\widehat{Q}^{j}{ }_{\alpha} \overline{\mathbf{e}}_{i} \otimes
\overline{\mathbf{e}}_{j} . 
\]
\end{enumerate}

Заметим, что решение квадратных уравнений допускает
неединственность решения в смысле выбора знаков у компонент матриц
$\widehat{Q}^{i}{ }_{\alpha}$ и $\widehat{Q}^{i}{ }_{\alpha}$, которая
устраняется после привлечения еще одного дополнительного условия - совпадения
векторов $\mathring{\mathbf{p}}_{\alpha}$ и $\mathbf{p}_{\alpha}$ в предельном
переходе при $t \rightarrow 0_{+}:$

\[
t \rightarrow 0_{+} \Rightarrow \mathbf{p}_{\alpha}(t)=\mathring{\mathbf{p}}_{\alpha}(t), \quad \alpha=1,2,3
\]


  \que{Вектор перемещений,  соотношения между перемещениями и градиентом
деформаций, перемещениями и тензорами деформаций. Соотношения Коши в случае
малых деформаций.}
\begin{definition*}
  \emph{Вектором перемещения} точки $ \mathcal M $ называется вектор  
  \[
    \mathbf{u} = \mathbf{x} - \mathring{\mathbf{x}}.
  \]
\end{definition*}

\begin{theorem*}[соотношения между перемещениями и градиентом деформаций]
  \begin{align*}
  &\begin{aligned}
    F &= E + (\mathring{\nabla}\otimes \mathbf{u})^{\mathsf T}, & F^{-1} &= E -
    (\nabla \otimes \mathbf{u})^{\mathsf T},\\
    F^{\mathsf T} &= E + \mathring{\nabla}\otimes \mathbf{u}, & F^{-1\mathsf{T}}
                  &= E - \nabla \otimes \mathbf{u},
  \end{aligned}\\
  &C = \frac{1}{2} ( \mathring{\nabla}\otimes \mathbf{u} +
  \mathring{\nabla}\otimes \mathbf{u}^{\mathsf T} + \mathring{\nabla}\otimes
\mathbf{u} \cdot \mathring{\nabla}\otimes \mathbf{u}^{\mathsf T} ),\\
  &A = \frac{1}{2} (\nabla \otimes \mathbf{u} + \nabla \otimes
  \mathbf{u}^{\mathsf T} - \nabla \otimes \mathbf{u}\cdot \nabla\otimes
  \mathbf{u}^{\mathsf T}),\\
  &\Lambda = \frac{1}{2}(\nabla \otimes \mathbf{u} + (\nabla \otimes
  \mathbf{u})^{\mathsf T} - \nabla \otimes \mathbf{u}^{\mathsf T} \cdot \nabla
  \otimes \mathbf{u}),\\
  &J = \frac{1}{2} ( \mathring{\nabla} \otimes \mathbf{u} + \mathring{\nabla}
  \otimes \mathbf{u}^{\mathsf T} + \mathring{\nabla} \otimes \mathbf{u}^{\mathsf
  T}\cdot \mathring{\nabla}\otimes \mathbf{u}).
\end{align*}
\end{theorem*}
\begin{proof}
  Докажем для примера одну из цепочек. Для начала  
  \[
    F^{\mathsf T} = \mathring{\nabla}\otimes \mathbf{x} =
    \mathring{\nabla}\otimes (\mathring{\mathbf{x}} + \mathbf{u}) =
    \mathring{\mathbf{r}}^i \otimes \mathring{\mathbf{r}}_i +
    \mathring{\nabla}\otimes \mathbf{u} = E + \mathring{\nabla} \otimes
    \mathbf{u}.
  \]
  Тогда (почему-то считается, что $ (\mathring{\nabla}\otimes
  \mathbf{u})^{\mathsf T} = \mathring{\nabla}\otimes \mathbf{u}^{\mathsf T} $)
 \[
   C = \frac{1}{2} ( (E + \mathring{\nabla} \otimes \mathbf{u}) \cdot
   (E+\mathring{\nabla}\otimes \mathbf{u}^{\mathsf T}) - E) = \frac{1}{2}
   (\mathring{\nabla}\otimes \mathbf{u} + \mathring{\nabla}\otimes
   \mathbf{u}^{\mathsf T} + \mathring{\nabla}\otimes \mathbf{u}\cdot
   \mathring{\nabla}\otimes \mathbf{u}^{\mathsf T}).
 \]
\end{proof}

Вектор $ \mathbf{u} = \mathring{u}^i \mathring{\mathbf{r}}_i = u^i \mathbf{r}_i
$ можно разложить по любому из базисов, как и его производную. Тогда 
\begin{align*}
  \mathring{\nabla}\otimes \mathbf{u} &= \mathring{\mathbf{r}}^i \otimes
  \frac{\partial \mathbf{u}}{\partial X^i} = \mathring{\nabla}_i \mathring{u}^k
  \mathring{\mathbf{r}}^i \otimes \mathring{\mathbf{r}}_k = \mathring{\nabla}^i
  \mathring{u}^k \mathring{\mathbf{r}}_i \otimes \mathring{\mathbf{r}}_k,\\
  {\nabla}\otimes \mathbf{u} &= {\mathbf{r}}^i \otimes
  \frac{\partial \mathbf{u}}{\partial X^i} = {\nabla}_i {u}^k
  {\mathbf{r}}^i \otimes {\mathbf{r}}_k = {\nabla}^i
  {u}^k {\mathbf{r}}_i \otimes \mathbf{r}_k.
\end{align*}
Отсюда 
\begin{align*}
  F &= (\delta^k_i + \mathring{\nabla}_i \mathring{u}^k)\mathring{\mathbf{r}}_k
  \otimes \mathring{\mathbf{r}}^i = \tensor{\mathring{F}}{^k_i}
  \mathring{\mathbf{r}}_k\otimes \mathring{\mathbf{r}}^i,\\
  F^{\mathsf T} &= (\delta^k_i +
  \mathring{\nabla}^k\mathring{u}_i)\mathring{\mathbf{r}}_k\otimes
  \mathring{\mathbf{r}}^i = \tensor{\mathring{F}}{_i^k}\mathring{\mathbf{r}}_k\otimes
  \mathring{\mathbf{r}}^i,\\
  F^{-1} &= (\delta^k_i -
  {\nabla}_i {u}^k) {\mathbf{r}}_k\otimes
  {\mathbf{r}}^i = \tensor{{{F}^{-1}}}{^k_i} {\mathbf{r}}_k\otimes
  {\mathbf{r}}^i,\\
  F^{-1\mathsf T} &= (\delta^k_i +
  {\nabla}^k{u}_i){\mathbf{r}}_k\otimes
  {\mathbf{r}}^i = \tensor{{{F}^{-1}}}{_i^k}{\mathbf{r}}_k\otimes
  {\mathbf{r}}^i.
\end{align*}

Подставляя теперь это в выражения для $ C $ и $ A $, получаем  
\begin{align*}
  \varepsilon_{ij} &= \frac{1}{2}(\mathring{\nabla}_i \mathring{u}_j +
  \mathring{\nabla}_j\mathring{u}_i + \mathring{\nabla}_i\mathring{u}^k
  \mathring{\nabla}_j \mathring{u}_k),\\
  \varepsilon_{ij} &= \frac{1}{2}({\nabla}_i {u}_j +
  {\nabla}_j{u}_i + {\nabla}_i{u}^k
  {\nabla}_j {u}_k).
\end{align*}
А подставляя в выражения для $ \Lambda $ и $ J $, получим
\begin{align*}
  \varepsilon^{ij} &= \frac{1}{2}(\mathring{\nabla}^i \mathring{u}^j +
  \mathring{\nabla}^j\mathring{u}^i + \mathring{\nabla}^k\mathring{u}^i
  \mathring{\nabla}_k \mathring{u}_j),\\
  \varepsilon^{ij} &= \frac{1}{2}({\nabla}^i {u}^j +
  {\nabla}^j{u}^i + {\nabla}^k{u}^i
  {\nabla}_k {u}^j).
\end{align*}

Кроме того, 
\begin{align*}
  g_{ij} &= \mathring{g}_{ij} + \mathring{\nabla}_i \mathring{u}_j +
  \mathring{\nabla}_j\mathring{u}_i + \mathring{\nabla}_i\mathring{u}^k
  \mathring{\nabla}_j \mathring{u}_k = \mathring{g}_{ij} + {\nabla}_i {u}_j +
  {\nabla}_j{u}_i + {\nabla}_i{u}^k
  {\nabla}_j {u}_k,\\
  g^{ij} &= \mathring{g}^{ij}+\mathring{\nabla}^i \mathring{u}^j +
  \mathring{\nabla}^j\mathring{u}^i + \mathring{\nabla}^k\mathring{u}^i
  \mathring{\nabla}_k \mathring{u}_j = \mathring{g}^{ij} + {\nabla}^i {u}^j +
  {\nabla}^j{u}^i + {\nabla}^k{u}^i
  {\nabla}_k {u}^j.
\end{align*}



  \que{Вектор скорости, конвективная производная,  кинематическое  соотношение. Тензор скоростей деформации. }


  % Законы сохранения
  \que{Закон сохранения массы.  Уравнение неразрывности в переменных Лагранжа. Различные формы уравнения неразрывности.}

\paragraph{Закон сохранения массы.} 
\begin{axiom*}[закон сохранения массы]
	Для всякой сплошной среды $\mathcal{B}$ (тела) существует скалярная функция $M(\mathcal{B}, t)$, называемая \textbf{массой} тела и обладающая следующими свойствами:
	\begin{enumerate}
		\item положительностью: $M > 0$,
		
		\item аддитивностью: $M(\mathcal{B}_1 + \mathcal{B}_2, t) = M(\mathcal{B}_1, t) + M(\mathcal{B}_2, t), \, \forall \mathcal{B}_1 \text{ и } \mathcal{B}_2, \, \forall t \leqslant 0$,
		
		\item инвариантностью по отношению к любым преобразованиям координат и к любым движениям  
	\end{enumerate}
	
	Из последнего свойства следует, что масса в любой актуальной конфигурации не меняется:
	\begin{equation*}
		M(\mathcal{B}, t) = \mathrm{const}.
	\end{equation*}
\end{axiom*}

\begin{remark*}
	Закон можно записать иначе: 
	\begin{equation*}
		dM / dt = 0.
	\end{equation*}
	
	Из аддитивности массы следует, что $M$ можно представить следующим образом:
	\begin{equation*}
		M = \int \limits_{V} dm, 
	\end{equation*}
	где $dm$ --- масса элементарного объема $dV$, содержащего материальную точку $\mathcal{M}$ из рассматриваемой области $V$ сплошной среды.
\end{remark*}

\begin{definition*}
	Отношение 
	\begin{equation*}
		\rho = dm / dV
	\end{equation*}
	называется \textit{плотностью} вещества в точке $\mathcal{M}$. 
	
	В силу положительности массы $M$ и объема $dV$, масса и плотность также всегда положительны: $\rho > 0, \, dm > 0$. 
\end{definition*}

Теперь мы можем записать \textit{закон сохранения массы в интегральной форме}: 
\begin{equation*}
	\dv{}{t} \int\limits_{V} \rho \, dV = 0,
\end{equation*}
или, применяя это соотношение к элементарному объему, получим
\begin{equation*}
	\rho \, dV = \mathring{\rho} \, d\mathring{V} = \mathrm{const}.
\end{equation*}

Последнее соотношение называется \textit{законом сохранения массы в дифференциальной форме}. 

\paragraph{Уравнение неразрывности в переменных Лагранжа.} Рассмотрим в $\mathring{\mathcal{K}}$ элементарный объем $d\mathring{V}$, построенный на элементарных радиусах векторах, ориентированных по локальным векторам базиса $d\mathring{\mathbf{x}}_{\alpha} = \mathring{\mathbf{r}}_{\alpha} d X^{\alpha}$. В актуальной конфигурации $\mathcal{K}$ ему соответствует область $dV$, построенная на векторах $\mathbf{r}_{\alpha} dX^{\alpha}$. Объемы областей $d\mathring{V}$ и $dV$ в этом случае вычисляются с использованием формул:
\begin{align*}
	d\mathring{V} &= \mathring{\mathbf{r}}_1 \cdot \left(\mathring{\mathbf{r}}_2 \times \mathring{\mathbf{r}}_3\right) \, dX^1 dX^2 dX^3 = \sqrt{\mathring{g}} \, dX^1 dX^2 dX^3 = \abs{\frac{\partial \mathring{x}^k}{\partial X^i}} \, dX^1 dX^2 dX^3, \\
	dV &= \mathbf{r}_1 \cdot \left(\mathbf{r}_2 \times \mathbf{r}_3\right) \, dX^1 dX^2 dX^3 = \sqrt{g} \, dX^1 dX^2 dX^3 = \abs{\frac{\partial x^k}{\partial X^i}} \, dX^1 dX^2 dX^3.
\end{align*}

Подставляя эти выражения в закон сохранения массы в дифференциальной форме приходим к следующей теореме.

\begin{theorem*}
	Изменение плотности при переходе из конфигурации $\mathcal{K}$ в $\mathring{\mathcal{K}}$ определяется одним из следующих уравнений:
	\begin{equation*}
		\frac{\mathring{\rho}}{\rho} = \sqrt{\frac{g}{\mathring{g}}} = \frac{\abs{\partial x^k / \partial X^i}}{\partial \mathring{x}^j / \partial X^n} = \abs{\frac{\partial x^k}{\partial \mathring{x}^i}} = \det{\mathbf{F}}.
	\end{equation*}
	
	Эти уравнения называют \textbf{уравнениями неразрывности в переменных Лагранжа}.
	
	Часто используют соотношение элементарных объемов в $\mathcal{K}$ и $\mathring{\mathcal{K}}$:
	\begin{equation*}
		dV / d\mathring{V} = \sqrt{g / \mathring{g}}.
	\end{equation*}
\end{theorem*}

  \que{Дифференцирование интеграла по подвижному объему и уравнение неразрывности в пространственном описании.}

Рассмотрим некоторое переменное векторное поле $\mathbf{a}(x^i, t)$, являющееся непрерывно-дифференцируемой функцией $x^i$ и $t$ в области $V(t)$, $\forall t \geqslant 0$, где область $V(t)$ содержит одни и те же материальные точки (такую область обычно называют \textit{подвижным объемом}). Проинтегрируем поле $\mathbf{a}(x^i, t)$ по области $V(t)$ и вычислим производную от интеграла: $\frac{d}{dt} \int_{V(t)} \mathbf{a} \, dV$. 

Для этого сделаем замену переменных в интеграле: $x^i \to \mathring{x}^i$, где $x^i \in V, \mathring{x}^i \in \mathring{V}$, причем якобиан преобразований, согласно уравнению неразрывности в переменных Лагранжа, равен $\abs{\partial x^k / \partial \mathring{x}^i} = \sqrt{g / \mathring{g}}$: 
\begin{equation*}
	\frac{d}{dt} \int\limits_{V(t)} \mathbf{a} \, dV = \frac{d}{dt} \int\limits_{\mathring{V}} \sqrt{\frac{g}{\mathring{g}}} \mathbf{a} \, d\mathring{V}.
\end{equation*} 

Поскольку при такой замене переменных мыодновременно перешли из конфигурации $\mathring{\mathcal{K}}$ в конфигурацию $\mathring{\mathcal{K}}$, в кторой область $\mathring{V}$ не зависит от $t$, то производную по $t$ можно внести под знак интеграла:
\begin{equation*}
	\frac{d}{dt} \int\limits_{V(t)} \mathbf{a} \, dV = \int\limits_{\mathring{V}} \frac{d}{dt} \left(\frac{\sqrt{g}}{\sqrt{\mathring{g}}} \mathbf{a}\right) \, d\mathring{V} = \int\limits_{\mathring{V}} \frac{1}{\sqrt{\mathring{g}}} \left(\mathbf{a} \frac{d}{dt} \sqrt{g} + \sqrt{g} \frac{d\mathbf{a}}{dt}\right) \, d\mathring{V}.
\end{equation*}

Преобразуя окончательно получим:
\begin{align*}
	\frac{d}{dt} \int\limits_{V(t)} \mathbf{a} \, dV &= \int\limits_{\mathring{V}} \sqrt{g / \mathring{g}} \left(\mathbf{a} \nabla \cdot \mathbf{v} + \frac{\partial \mathbf{a}}{\partial t} + \mathbf{v} \cdot \nabla \otimes \mathbf{a}\right) \, d\mathring{V} = \\
	&= \int\limits_{\mathring{V}} \sqrt{\frac{g}{\mathring{g}}} \left(\frac{\partial \mathbf{a}}{\partial t} + \nabla \cdot \left(\mathbf{v} \otimes \mathbf{a}\right)\right) \, d\mathring{V} = \int\limits_{V} \left(\frac{\partial \mathbf{a}}{\partial t} + \nabla \cdot (\mathbf{v} \otimes \mathbf{a})\right) \, dV.
\end{align*}

Последнюю строку мы получили, совершив обратное преобразование координат $\mathring{x}^i \to x^i$.

Таким образом, доказана следующая теорема.
\begin{theorem}[правило дифференцирования интеграла по подвижному объему]
	Для произвольного переменного векторного поля $\mathbf{a}(x^i, t)$, заданного в $V(t) \forall t \geqslant 0$ и являющегося непрерывно-дифференцируемой функцией $x^i$ и $t$, имеет место следующее соотношение:
	\begin{equation*}
		\frac{d}{dt} \int\limits_{V(t)} \mathbf{a}(x^i, t) \, dV = \int\limits_{V} \left(\frac{\partial \mathbf{a}}{\partial t} + \nabla \cdot \mathbf{v} \otimes \mathbf{a}\right) \, dV.
	\end{equation*} 
	
	Выбирая в данной формуле в качестве векторного поля $\mathbf{a}(x^i, t) = \varphi(x^i, t) \bar{\mathbf{e}}_{\alpha}$, где $\bar{\mathbf{e}}_{\alpha}$ --- какой-либо из векторов декартова базиса, а $\varphi(x^i, t)$ --- переменное скалярное поле, и вынося $\bar{\mathbf{e}}_{\alpha}$ из-под знака интеграла в левой и правой частях, получаем формулу дифференцирования интеграла от скалярного поля: 
	\begin{equation*}
		\frac{d}{dt} \int\limits_{V(t)} \varphi(x^i, t) \, dV = \int\limits_{V} \left(\frac{\partial \varphi}{\partial t} + \nabla \cdot (\varphi \mathbf{v})\right) \, dV. 
	\end{equation*}
\end{theorem}
  \que{Закон изменения количества движения. Интегральная  форма   уравнения движения, Внутренние и внешние силы. Массовые и поверхностные силы. }

  \que{Вектор напряжений. Теоремы 1 и 2 Коши о свойствах вектора напряжений.}

\paragraph{Вектор напряжений.} Как уже было отмечено в ответе на предыдущий вопрос, для некоторого разбиение области $V$ на $V_1$ и $V_2$ поверхностью $\Sigma_0$ имеем, что на некоторой элементарной площадке $d\Sigma \in \Sigma_0$ действуют поверхностные силы $d \mathcal{F}_1$ и $d \mathcal{F}_2$ для соответствующих областей. Тогда плотности этих сил можно обозначить следующим образом: 
\begin{equation*}
	\mathbf{t}_{n} = d\mathcal{F}_1 / d\Sigma \quad \text{и} \quad t_{-n} = d\mathcal{F}_2 / d\Sigma.
\end{equation*} 

Векторы $\mathbf{t}_n$ и $\mathbf{t}_{-n}$ называют \textit{векторами напряжений}, они представляют собой плотности \textit{внутренних поверхностных сил} по отношению ко всей области $V$ сплошной среды (так как они определены для внутренних точек $\mathcal{M}$ этой области). 

\paragraph{Теорема Коши о свойствах вектора напряжений.} Очевидна относительность разбиения сил на внешние и внутренние: одни и те же силы могут быть внутренними или внешними по отношению к различным объемам сплошной среды. 

Запишем уравнения изменения количества движения для всей области $V$ и для отдельных его частей $V_1$ и $V_2$:
\begin{align*}
	\int\limits_{V_1} \rho \left(\mathbf{f}_1 - \frac{d \mathbf{v}}{\mathbf{t}}\right) \, dV + \int\limits_{\Sigma_1} \mathbf{s}_1 \, d\Sigma + \int\limits_{\Sigma_0} \mathbf{t}_n \, d\Sigma &= 0, \\
	\int\limits_{V_2} \rho \left(\mathbf{f}_2 - \frac{d\mathbf{v}}{dt}\right) \, dV + \int\limits_{\Sigma_2} \mathbf{s}_2 \, d\Sigma + \int\limits_{\Sigma_0} \mathbf{t}_{-n} \, d\Sigma &= 0,
\end{align*}
где $\mathbf{f}_i$ и $\mathbf{s}_i$ --- силы, действующие в областях $V_i$ и на поверхностях $\Sigma_i$, т.е. $\mathbf{f} = \mathbf{f}_i$ в $V_i$ и $\mathbf{s} = \mathbf{s}_i$ на $\Sigma_i$. В силу непрерывности всех функций $\mathbf{s}_i$ и $\mathbf{f}_i$, вычитая их уравнения изменения количества движения уравнения для отдельных его частей получаем, что
\begin{equation*}
	\int\limits_{\Sigma_0} \left(\mathbf{t}_n + \mathbf{t}_{-n}\right) \, d\Sigma = 0.
\end{equation*}

В силу произвольности поверхности $\Sigma_0$, заключаем, что $\mathbf{t}_n + \mathbf{t}_{-n} = 0$. Таким образом, мы доказали следующую теорему. 

\begin{theorem*}[первая теорема Коши --- о непрерывности вектора напряжений]
	Для одной и той же точки $\mathcal{M}$, являющейся внутренней точкой области $V$, вектор напряжений, определенный по отношению к площадкам $\mathbf{n} \, d\Sigma_0$ и $-(\mathbf{n} \, d\Sigma_0)$, различается только знаком:
	\begin{equation*}
		\mathbf{t}_{n} = - \mathbf{t}_{-n},
	\end{equation*}
	т.е. поле $\mathbf{t}_n(x)$ --- непрерывно в области $V$.
\end{theorem*}

\begin{theorem*}[вторая теорема Коши]
	Вектор напряжений $\mathbf{t}_n$ на произвольной площадке с нормалью $\mathbf{n}$ выражается через векторы напряжений $\mathbf{t}_{\alpha}$ на трех координатных площадках следующим образом:
	\begin{equation*}
		\mathbf{t}_{n} = \sum\limits_{\alpha = 1}^{3} \mathbf{n} \cdot \mathbf{r}_{\alpha} \abs{\mathbf{r}^{\alpha}} \mathbf{t}_{\alpha}.
	\end{equation*}
	
	Т.к. 
	\begin{equation*}
		\abs{\mathbf{r}^{\alpha}} = \left(\mathbf{r}^{\alpha} \cdot \mathbf{r}^{\alpha}\right)^{1/2} = \sqrt{g^{\alpha\alpha}},
	\end{equation*}
	то соотношение из формулировки теоремы можно переписать в виде:
	\begin{equation*}
		\mathbf{t}_{n} = \mathbf{n} \cdot \mathbf{T},
	\end{equation*}
	где $\mathbf{T}$ --- тензор второго ранга, называемый \textit{тензором истинных напряжений Коши}:
	\begin{gather*}
		\mathbf{T} = \sum\limits_{\alpha = 1}^{3} \mathbf{r}_{\alpha} \otimes \mathbf{t}^{\alpha} = \mathbf{r}_{i} \otimes \mathbf{t}^i, \\
		\mathbf{t}^{\alpha} \equiv \mathbf{t}_{\alpha} \sqrt{g^{\alpha\alpha}}. 
	\end{gather*}
	
	Можно переформулировать теорему следующим образом: <<Для непрерывного в $V \cup \Sigma$ поля вектора напряжений $\mathbf{s}(x) = \mathbf{t}_n(x)$, удовлетворяющих уравнению закона изменения количества движения в интегральной форме, всегда существует поле тензора $\mathbf{T}(\mathbf{x})$, удовлетворяющее соотношению $\mathbf{t}_n = \mathbf{n} \cdot \mathbf{T}$ в $V \cup \Sigma$>>. 
\end{theorem*}
  \que{Тензор напряжений Коши, физический  смысл компонент тензора  напряжений Коши. Тензор напряжений Пиолы-Кирхгофа, физический  смысл компонент тензора  напряжений Пиолы-Кирхгофа.}

  \que{Уравнение  движения  в  пространственном и материальном описании.}


  \que{Закон сохранения моментов количества движения. Обобщенная теорема Коши. Дифференциальная форма закона сохранения моментов количества движения. Полярные и неполярные среды. Симметрия тензора напряжений Коши.}

  \que{Первый закон термодинамики в пространственном и материальном описании. Интегральная и дифференциальная формулировки. Вектор потока тепла. }

Законы сохранения массы, изменения количества движения и момента количества движения описывают движение сплошной среды. Для учета тепловых эффектов в сплошных средах необходимо привлекать \textit{законы термодинамики}. Рассмотрим вначале неполярные среды.

\begin{definition*}
	\textit{Кинетической энергией} сплошной среды $V$ называют следующую скалярную функцию: 
	\begin{equation*}
		K = \int\limits_{V} \frac{\mathbf{v} \cdot \mathbf{v}}{2} \, dm = \int\limits_{V} \rho \frac{\abs{v}^2}{2} \, dV, \quad \abs{v}^2 = \mathbf{v} \cdot \mathbf{v}.
	\end{equation*}
\end{definition*}

\begin{definition*}
	\textit{Мощностью внешних массовых сил} $W_m$ и \textit{мощностью внешних поверхностных сил} $W_{\Sigma}$ называют следующие скалярные функции:
	\begin{equation*}
		W_{m} = \int\limits_{V} \mathbf{f} \cdot \mathbf{v} \, dm = \int\limits_{V} \rho \mathbf{f} \cdot \mathbf{v} \, dV, \quad W_{\Sigma} = \int\limits_{\Sigma} \mathbf{t}_n \cdot \mathbf{v} \, d\Sigma.
	\end{equation*}
\end{definition*}

\begin{axiom*}[первй закон термодинамики --- закон сохранения энергии]
	Для всякой сплошной среды $\mathcal{B}$ существуют две скалярные аддитивные функции: $U(\mathcal{B}, t)$ --- \textbf{внутренняя энергия} сплошной среды и $Q(\mathcal{B}, t)$ --- \textbf{скорость нагрева} сплошной среды, такие что $\forall t \geqslant 0$ выполняется уравнение
	\begin{equation*}
		\frac{dE}{dt} = W + Q,
	\end{equation*}
	где $E$ называют \textbf{полной энергией} сплошной среды, которая состоит из $U$ м $K$:
	\begin{equation*}
		E = U + K, \quad W = W_{m} + W_{\Sigma}.
	\end{equation*}
\end{axiom*}
\begin{remark*}
	Данная формулировка, в отличии от иных имеющихся в литературе, является универсальной, т.е. не зависит от типа сплошной среды. 
\end{remark*}
\begin{definition*}
	\textit{Плотностью внутренней энергии} называют функцию $e$, \textit{притоком тепла за счет массовых источников} --- функцию $q_m$, а \textit{притоком тепла за счет поверхностных источников} --- функию $q_{\Sigma}$, определенные в каждой точке сплошной среды $\mathcal{M}$ следующим образом:
	\begin{equation*}
		e = \frac{dU}{dm}, \quad q_m = \frac{dQ}{dm}, \quad q_{\Sigma} = \frac{dQ}{d\Sigma}.
	\end{equation*}
	В силы аддитивности функции $Q$ и $U$, для всего обхема сплошной среды получаем:
	\begin{gather*}
		Q = Q_m + Q_{\Sigma}, \\
		Q_m = \int\limits_{V} q_m \, dm = \int\limits_{V} \rho q_m \, dV, \quad Q_{\Sigma} = \int\limits_{\Sigma} q_{\Sigma} \, d\Sigma, \\
		U = \int\limits_{V} e \, dm = \int\limits_{V} \rho e \, dV.
	\end{gather*}
\end{definition*}

Подставляя выражения выше, определения кинетической энергии, мощности сил, внутреннюю энергию и ее плотность, получаем \textit{закон сохранения энергии} в интегральной форме:
\begin{equation*}
	\frac{d}{dt} \rho \left(e + \frac{\abs{v}^2}{2}\right) \, dV = \int\limits_{V} \rho \left(\mathbf{f} \cdot \mathbf{v} + q_m\right) \, dV + \int\limits_{\Sigma} \left(\mathbf{t}_n \cdot \mathbf{v} + q_{\Sigma}\right) \, d\Sigma.
\end{equation*}

Используя для левой части выражения правила дифференцирования для объемного интеграла:
\begin{equation*}
	\frac{d}{dt} \int\limits_{V} \rho \left(e + \frac{\abs{v}^2}{2}\right) \, dV = \int\limits_{V} \rho \left(\frac{de}{dt} + \mathbf{v} \cdot \frac{d\mathbf{v}}{dt}\right) \, dV,
\end{equation*}
получим следующую форму закона сохранения энергии:
\begin{equation*}
	\int\limits_{V} \rho \left(-\frac{d}{dt}\left(e + \frac{1}{2} \abs{v}^2\right) + \mathbf{f} \cdot \mathbf{v} + q_m\right) \, dV + \int\limits_{\Sigma} \left(\mathbf{t}_n \cdot \mathbf{v} + q_{\Sigma}\right) \, d\Sigma = 0.
\end{equation*}

\paragraph{Вектор потока тепла.} Уравнение закона сохранения энергии в интегральной форме имеет вид как в обобщенной теореме Коши, если положить:
\begin{equation*}
	A = e + \abs{v^2} / 2, \quad C = \mathbf{f} \cdot \mathbf{v} + q_m \quad \text{и} \quad B = \mathbf{t}_n \cdot \mathbf{v} + q_{\Sigma}.
\end{equation*}

Тогда к этом уравнению можно применить вторую теорему Коши, которая утверждает существование такого вектора $(-\mathbf{q})$, что в $V \cup \Sigma$ имеет место соотношение 
\begin{equation*}
	q_{\Sigma} = - \mathbf{n} \cdot \mathbf{q}.
\end{equation*}
Вектор $\mathbf{q}$ называют \textit{вектором потока тепла}. Для него справедливо:
\begin{equation*}
	\mathbf{q} = \sum\limits_{\alpha = 1}^{3} \mathbf{r}_{\alpha} \sqrt{g^{\alpha\alpha}} q_{\alpha} = \mathbf{r}_i q^i, \quad q^{\alpha} = q_{\alpha} \sqrt{g^{\alpha\alpha}}.
\end{equation*}

Запишем теперь закон сохранения энергии в материальном описании.

\begin{theorem*}
	В условиях теоремы о уравнении энергииЖ в каждой точке $\mathcal{M} \in \mathring{V}$ для всех рассматриваемых $t \geqslant 0$ имеет место \textbf{уравнение энергии в лагранжевом описании:}
	\begin{equation*}
		\mathring{\rho} \frac{d}{dt} \left(e + \frac{\abs{v}^2}{2}\right) = \mathring{\rho} \mathbf{f} \cdot \mathbf{v} + \mathring{\rho} q_m + \mathring{\nabla} \cdot \left(\mathbf{P} \cdot \mathbf{v}\right) - \mathring{\nabla} \cdot \mathring{\mathbf{q}}. 
	\end{equation*}
\end{theorem*}

  \que{Второй закон термодинамики в пространственном и материальном описании. Интегральная и дифференциальная формулировки. }

  \que{Понятие о коэффициенте полезного действия.}

  \que{Статические уравнения совместности деформаций. Четыре различные формулировки.}

  \que{Динамические уравнения совместности деформаций в материальном и пространственном описании.}
стр 140 второй димдим (нелинейная механика)



\textit{
\begin{itemize}
    \item (1.2.10) -- Теорема 1.4. Тензоры деформации и градиент деформации связаны с
вектором перемещений u следующими соотношениями...
\item (1.4.7) Полной производной по времени от переменного векторного поля a
\item (1.1.15) С помощью смешанного произведения трех различных векторов локальных базисов можно вычислить объемы
\item (1.20) теорема 2.4
\item (1.1.35), (1.2.10) теорема 1.4, просто формулы для F во всяких разных видах
\end{itemize}
}
 
  



\paragraph{Динамические уравнения совместности в лагранжевом описании.}

 Уравнения совместности деформаций можно записать еще в одном эквивалентном виде -- через вектор скорости.

Рассмотрим уравнение (1.2.10), связывающее $\mathbf{F}$ с $\stackrel{\circ}{\nabla} \otimes \mathbf{u}^{\mathrm{T}}$, и продифференцируем его по $t$ с учетом определения вектора скорости $\textbf{v}$:

\begin{equation}
\label{eq:71}
\frac{d}{d t} \stackrel{\circ}{\nabla} \otimes \mathbf{u} =\stackrel{\circ}{\mathbf{r}}^{i} \otimes \frac{\partial^{2} \mathbf{u}}{\partial X^{i} \partial t}=\stackrel{\circ}{\mathbf{r}}^{i} \otimes \frac{\partial \mathbf{v}}{\partial X^{i}}=\stackrel{\circ}{\boldsymbol{\nabla}} \otimes \mathbf{v}=\frac{d \mathbf{F}^{\mathrm{T}}}{d t} 
\end{equation}


В результате получим динамическое уравнение совместности в лагранжевом описании:

\begin{equation}
\label{eq:72}
\frac{d \mathbf{F}^{\mathrm{T}}}{d t}=\stackrel{\circ}{\nabla} \otimes \mathbf{v}
\end{equation}

\begin{theorem} Условия совместности деформаций выполнены тогда и только тогда, когда в $\mathcal{K}$ существует поле тензора $\mathbf{F}\left(X^{i}, t\right)$, удовлетворяющее следующим условиям:
\begin{enumerate}
    \item $\operatorname{det} \mathbf{F} \neq 0$ в каждой точке $X^{i}$,
    \item $\mathbf{F}\left(X^{i}, 0\right)=\mathbf{E} n p u t=0$,
    \item поле $\mathbf{F}$ обладает векторным потенциалом, т.е. для него существует такое поле вектора $\mathbf{v}$, что выполняется уравнение (7.2) $\forall t>0$ и $\forall X^{i} \in V$.
\end{enumerate}
\end{theorem}

 Пусть выполнены условия совместности деформаций, тогда, согласно определению 2.12 , существует вектор перемещений $\mathbf{u}\left(X^{i}, t\right)$ вместе со своим градиентом $\stackrel{\circ}{\nabla} \otimes \mathbf{u}$. Проделывая преобразования (7.1), убеждаемся в справедливости (7.2).

Покажем справедливость обратного утверждения. Пусть существует вектор-функция $\mathbf{v}$, удовлетворяющая (7.2). Рассмотрим функцию $\widetilde{\mathbf{u}}\left(X^{i}, t\right)=\int_{0}^{t} \mathbf{v}\left(X^{i}, \tau\right) d \tau$. Эта функция удовлетворяет уравнению

\begin{equation}
\label{eq:73}
\stackrel{\circ}{\nabla} \otimes \widetilde{\mathbf{u}}=\stackrel{\circ}{\boldsymbol{\nabla}} \otimes \int_{0}^{t} \mathbf{v} d \tau=\int_{0}^{t} \boldsymbol{\nabla} \otimes \mathbf{v} d \tau=\int_{0}^{t} \frac{d \mathbf{F}^{\mathrm{T}}}{d \tau} d \tau=\mathbf{F}^{\mathrm{T}}-\mathbf{E} 
\end{equation}

Тогда на основе этой функции можно построить радиус-вектор $\widetilde{\mathbf{x}}=\stackrel{\circ}{\mathbf{x}}+\widetilde{\mathbf{u}}$, с помощью которого тензор $\mathbf{F}$ будет представлен в виде:
\begin{equation}
\label{eq:74}
\mathbf{F}^{\mathrm{T}}=\mathbf{E}+\stackrel{\circ}{\nabla} \otimes \widetilde{\mathbf{u}}=\stackrel{\circ}{\mathbf{r}}^{i} \otimes \frac{\partial(\mathbf{\circ}+\widetilde{\mathbf{x}})}{\partial X^{i}}=\stackrel{\circ}{\mathbf{r}}^{i} \otimes \widetilde{\mathbf{r}}_{i}
\end{equation}
где $\widetilde{\mathbf{r}}_{i}=\partial \widetilde{\mathbf{x}} / \partial X^{i}$. Но это означает, что $\widetilde{\mathbf{u}}$ и есть искомый вектор перемещений $\mathbf{u}$, а $\mathbf{F}$ - искомый градиент деформации, поскольку они удовлетворяют
всем кинематическим соотношениям: (1.1.35), (1.2.10) и др. Таким образом, существует вектор перемещений u, а, следовательно, выполнены условия совместности деформаций.

 
\paragraph{Динамические уравнения совместности в пространственном описании.}
Докажем вначале вспомогательное утверждение.
\begin{theorem}
Пусть выполнено уравнение неразрывности (1.1.15), тогда градиент деформации удовлетворяет следующему уравнению:
\begin{equation}
\label{eq:75}
\nabla \cdot(\rho \mathbf{F})=0 
\end{equation}
\end{theorem}

 Представим градиент деформации в диадном базисе:

\begin{equation}
\label{eq:76}
\rho \mathbf{F}=\rho F^{i j} \mathbf{r}_{i} \otimes \mathbf{r}_{j} 
\end{equation}

Воспользуемся формулой для дивергенции любого тензора [12]:

\begin{equation}
\label{eq:77}
\nabla \cdot (\rho \mathbf{F}) = 
\frac{1}{\sqrt{g}}
\frac{\partial}{\partial X^i}
(\rho \sqrt{g} F^{i j} \mathbf{r}_{j} ) = 
\frac{\stackrel{\circ}{\rho}}{\sqrt{g}}
\frac{\partial}{\partial X^i}
( \sqrt{\stackrel{\circ} g} \stackrel{\circ} {\mathbf{r}_{j} }) 
\end{equation}

Здесь использовано уравнение неразрывности $\rho=\stackrel{\circ}{\rho} \sqrt{\stackrel{o}{g} / g}$, а также очевидные соотношения:
\begin{equation}
\label{eq:78}
F^{ik}\textbf{r}_k =
F^{jk}\textbf{r}^i \cdot\textbf{r}_j \otimes \textbf{r}_k 
=\textbf{r}^i \cdot F
=(\textbf{r}^i \cdot\textbf{r}_k)\otimes \stackrel{\circ} {\textbf{r}^k} =\stackrel{\circ} {\textbf{r}^i}
\end{equation}


Дифференцируя \eqref{eq:76} по частям, получаем

\begin{equation}
\label{eq:79}
\nabla \cdot(\rho \mathbf{F})
=\frac{\stackrel{\circ}{\rho}}{\sqrt{g}}\left(\frac{\partial \sqrt{\circ}}{\partial X^{i}} \stackrel{\circ}{r}^{i}+\sqrt{\stackrel{\circ}{g}} \frac{\partial \stackrel{\circ}{\mathbf{r}}^{i}}{\partial X^{i}}\right)
=\frac{\stackrel{\circ}{\rho}}{\sqrt{g}}\left(\sqrt{\stackrel{\circ}{g}} \stackrel{\circ}{\Gamma}_{i s}^{s} \stackrel{\circ}{\mathbf{r}} \stackrel{ }{i}-\sqrt{\stackrel{\circ}{g}} \stackrel{\circ}{\Gamma}_{i s}^{s} \stackrel{\circ}{\mathbf{r}}^{i}\right)
=0 
\end{equation}

Здесь использованы свойства символов Кристоффеля [12].

Преобразуем теперь уравнение \eqref{eq:72} с учетом (1.1.37):

\begin{equation}
\label{eq:710}
\frac{d \mathbf{F}^{\mathrm{T}}}{d t}=\mathbf{F}^{\mathrm{T}} \cdot \boldsymbol{\nabla} \otimes \mathbf{v} 
\end{equation}

и воспользуемся уравнением неразрывности (1.15), которое умножим на $\mathbf{F}^{\mathrm{T}}$ :

\begin{equation}
\label{eq:711}
\frac{\partial \rho}{\partial t} \mathbf{F}^{\mathrm{T}}+\mathbf{F}^{\mathrm{T}} \boldsymbol{\nabla} \cdot \rho \mathbf{v}=0 
\end{equation}

Умножим \eqref{eq:79} на $\rho$ и применим определение (1.4.7) полной производной по времени:

\begin{equation}
\label{eq:712}
\rho \frac{d \mathbf{F}^{\mathrm{T}}}{d t}=\rho \frac{\partial \mathbf{F}^{\mathrm{T}}}{\partial t}+\rho \mathbf{v} \cdot \boldsymbol{\nabla} \otimes \mathbf{F}^{\mathrm{T}}=\rho \mathbf{F}^{\mathrm{T}} \cdot \boldsymbol{\nabla} \otimes \mathbf{v} 
\end{equation}


Сложим теперь уравнения \eqref{eq:710} и \eqref{eq:711}, в результате получим:


\begin{equation}
\label{eq:713}
\frac{\partial \rho \mathbf{F}^{\mathrm{T}}}{\partial t}+\boldsymbol{\nabla} \cdot\left(\rho \mathbf{v} \otimes \mathbf{F}^{\mathrm{T}}\right)-\rho \mathbf{F}^{\mathrm{T}} \cdot \boldsymbol{\nabla} \otimes \mathbf{v}=0 
\end{equation}


Умножая уравнение \eqref{eq:74} тензорно на $-\mathbf{v}$ (т.е. $-\boldsymbol{\nabla} \cdot(\rho \mathbf{F}) \otimes \mathbf{v}=0$ ) и складывая полученное выражение с \eqref{eq:712}, приходим к динамическому уравнению совместности в пространственном описании:


\begin{equation}
\label{eq:714}
\frac{\partial \rho \mathbf{F}^{\mathrm{T}}}{\partial t}+\nabla \cdot\left(\rho \mathbf{v} \otimes \mathbf{F}^{\mathrm{T}}-\rho \mathbf{F} \otimes \mathbf{v}\right)=0 
\end{equation}


Выполнение этого уравнения, так же как и \eqref{eq:72}, необходимо и достаточно для соблюдения условий совместности деформаций в $\mathcal{K}$.

Преобразуя первое и второе слагаемое в \eqref{eq:713} по формуле (1.20), динамическое уравнение совместности можно записать также в виде


\begin{equation}
\label{eq:715}
\rho \frac{d \mathbf{F}^{\mathrm{T}}}{d t}=\nabla \cdot(\rho \mathbf{F} \otimes \mathbf{v})
\end{equation}

  \que{Полные системы законов сохранения в пространственном и материальном описании.}


  % Теория определяющих соотношений
  \que{Незамкнутость системы законов сохранений МСС. Понятие об определяющих соотношениях. Принципы построения определяющих соотношений. Основное термодинамическое тождество.}
\footnote{Димитриенко Ю.И. -- Нелинейная механика сплошных сред, стр. 149}

\paragraph{Незамкнутость системы законов сохранений МСС.}
Полная система законов сохранения в полных дифференциалах:
\begin{equation*}
  \begin{cases}
    \rho \dfrac{d \bar{A}_\alpha}{dt} = \nabla \cdot \bar{B}_\alpha + \rho C_\alpha, 
    \alpha = 1..6 \\

    \bar{A}_\alpha = \begin{pmatrix}
      1/\rho \\ \mathbf{v} \\ e + |v|^2 / 2 \\ \eta \\ \mathbf{u} \\ \mathbf{F}^T
    \end{pmatrix}, \quad

    \bar{B}_\alpha = \begin{pmatrix}
      \mathbf{v} \\ \mathbf{T} \\ \mathbf{T} \cdot \mathbf{v} - \mathbf{q} \\ -\mathbf{q} / \theta \\
      \mathbf{0} \\ \rho \mathbf{F} \otimes \mathbf{v}
    \end{pmatrix}, \quad

    C_\alpha = \begin{pmatrix}
      0 \\ \mathbf{f} \\ \mathbf{f} \cdot \mathbf{v} + q_m \\ 
      (q_m + q^*) / \theta \\ 
      \mathbf{v} \\ \mathbf{0}
    \end{pmatrix} 
  \end{cases}
\end{equation*}
данная система содержит 18 скалярных уравнений и 29 скалярных неизвестных
$\rho, \mathbf{v}, \mathbf{u}, \mathbf{T}, e, \eta, \theta, \mathbf{q}, \mathbf{F}, q^*$.
Эта система одинакова для любой сплошной среды.

\paragraph{Определяющие соотношения.}
Для замыкания системы уравнений необходимы дополнительные соотношения. Эти дополнительные
соотношения называют \emph{определяющими соотношениями}, поскольку именно они определяют, чем одна
сплошная среда отличается от другой (универсальные законы сохранения «не различают» типы
сплошных сред — они одинаковы для всех тел). Если заданы каким-либо образом определяющие
соотношения, то говорят, что задана \emph{модель сплошной среды}.

\paragraph{Принципы построения ОС.}
Вывод определяющих соотношений основан на привлечении некоторых дополнительных принципов,
т.е. физических допущений общего характера, которые, вообще говоря, не формулируются в виде
дифференциальных уравнений в частных производных. Основными такими принципами являются:
\begin{itemize}
  \item принцип термодинамически согласованного детерминизма,
  \item принцип локальности,
  \item принцип равноприсутствия,
  \item принцип материальной индифферентности (объективности),
  \item принцип материальной симметрии,
  \item принцип Онзагера.
\end{itemize}
Кроме того, для частных моделей сред формулируют дополнительные принципы.

\paragraph{Основное термодинамическое тождество.}
% TODO Нелинейный Димитриенко, стр 177.
Выпишем законы изменения энергии и закон притока тепла в полных дифференциалах:
\[
  \begin{cases}
    \rho \dfrac{de}{dt} =
    \mathbf{T} \cdot\cdot(\nabla \otimes \mathbf{v})^T +
    \rho q_m - \nabla\cdot\mathbf{q}, \\
    \rho\theta \dfrac{d\eta}{dt} =
    \rho q_m - \nabla \cdot \mathbf{q} + w^*,
  \end{cases}
\]
исключая $\nabla \cdot \mathbf{q}$ из этих уравнений, и вспоминая обозначение $w_{(i)} = \mathbf{T} \cdot\cdot (\nabla \otimes \mathbf{v})^T$, получим:
\[
  \rho \left( \dfrac{de}{dt} - \theta \dfrac{d\eta}{dt} \right) - w_{(i)} + w^* = 0.
\]

Сюда вместо $w_{(i)}$ можно подставить выражение через энергетические пары тензоров
(см. следующий вопрос), и тогда
это соотношение можно мыслить себе как некоторое соотношение, связывающее изменение трёх
основных величин: $e, \theta$ и $\stackrel{(n)}{\mathbf{C}}$
(или $e, \theta$ и $\stackrel{(n)}{\mathbf{G}}$,
или $e, \theta, \stackrel{(n)}{\mathbf{A}}$ и $\mathbf{O}^T$, 
или $e, \theta, \stackrel{(n)}{\mathbf{g}}$ и $\mathbf{O}^T$)
в локальной точке сплошной среды.

Различают также $e$- и $\psi$-формы ОТТ, то выражение, которое написано выше, называют $e$-формой,
потому что оно содержит энергию $e$. Если ввести т.н. \emph{свободную энергию Гельмгольца} 
$\psi = e - \theta \eta$, то получим выражение в $\psi-$ форме:
\[
  \rho \dfrac{d\psi}{dt} + \rho\eta \dfrac{d\theta}{dt} - w_{(i)} + w^* = 0.
\]

Вообще говоря, ОТТ можно записать ещё в 100 других видах, вводя всё новые и новые (и никому не
нужные) замены, но важно, что в ОТТ некоторые переменные входят через производные -- такие
будет называть \emph{реактивными} $\mathcal{R}$, а другие назовём \emph{активными} $\Lambda$.

  \que{Энергетические пары тензоров напряжений и деформаций. }
В законе изменения кинетической энергии использовалась величина $W_{(i)}$ -- мощность внутренних
поверхностных сил, которую мы определяли как
$W_{(i)} = - \int_V \mathbf{T} \cdot\cdot (\nabla \otimes \mathbf{v})^T \, dV$.
Обозначим $w_{(i)} = \mathbf{T} \cdot\cdot (\nabla \otimes \mathbf{v})^T$ --
\emph{мощность напряжений}, тогда $W_{(i)} = - \int_V w_{(i)} \, dV$.

\begin{definition}[\footnote{Димитриенко -- Нелинейная МСС, стр 150}]
  \emph{Энергетическими тензорами напряжений} $\stackrel{(n)}{\mathbf{T}}$,
  и \emph{энергетическими тензорами деформаций} $\stackrel{(n)}{\mathbf{C}}$
  называются такие пары тензоров, которыми наиболее удачно можно представить мощность напряжений
  в виде:
  \[
    w_{(i)} = \stackrel{(n)}{\mathbf{T}} \cdot\cdot \dfrac{d}{dt} \stackrel{(n)}{\mathbf{C}} + \mathbf{T}^K \cdot\cdot \mathbf{W},
  \]
  где $T^K = \dfrac{1}{2} (T - T^T)$ -- кососиметричная часть тензора напряжений Коши $T$, а 
  $\mathbf{W}$ -- тензор вихря.

  В случаях, когда $T = T^T$, второе слагаемое отсутствует.
\end{definition}

Выделяют 5 пар таких тензоров.

\begin{center}
  \begin{tabular}{|c|c|c|}
    \hline
    $n$ & $\stackrel{(n)}{\mathbf{T}}$ & $\stackrel{(n)}{\mathbf{C}}$ \\
    \hline
    I & $\mathbf{F}^T \cdot \mathbf{T}^S \cdot \mathbf{F}$ & $\boldsymbol{\Lambda} = \dfrac{1}{2} (\mathbf{E} - \mathbf{U}^{-2})$ \\
    II & $1/2 (\mathbf{F}^T \cdot \mathbf{T}^S \cdot \mathbf{O} + \mathbf{O}^T \cdot \mathbf{T}^S \cdot \mathbf{F})$ & $\mathbf{E} - \mathbf{U}^{-1}$ \\
    III & $\mathbf{O}^T \cdot \mathbf{T}^s \cdot \mathbf{O}$ & $\mathbf{B}$ \\
    IV & $1/2 (\mathbf{F}^{-1} \cdot \mathbf{T}^s \cdot \mathbf{O} + \mathbf{O}^T \cdot \mathbf{T}^S \cdot \mathbf{F}^{-1T})$ & $\mathbf{U} - \mathbf{E}$ \\
    V & $\mathbf{F}^{-1} \cdot \mathbf{T}^S \cdot \mathbf{F}^{-1T}$ & $\mathbf{C} = \dfrac{1}{2}(\mathbf{U}^2 - \mathbf{E})$ \\
    \hline
  \end{tabular}
\end{center}
где $T^S = \dfrac{1}{2} (T + T^T)$ - симметрическая часть тензора напряжений Коши;
$\mathbf{F}$ -- тензор градиента деформаций;
$\mathbf{O}, \mathbf{U}$ -- полярное разложение тензора градиента деформаций $\mathbf{F} = \mathbf{O} \cdot \mathbf{U}$.

% В случае третьей пары, энергетический тензор деформаций $\stackrel{(III)}{\mathbf{C}} = \mathbf{B}$
% неизвестен, но является решением следующего ДУ

% TODO проверить, что ещё можно написать.


  \que{Принцип термодинамически согласованного детерминизма, принцип локальности. }


  \que{Общий вид определяющих соотношений сплошных сред  (модели  An ).  Определение идеальных сред. Общий вид определяющих соотношений для идеальных сплошных сред.}

  \que{Н-преобразования отсчетной конфигурации. Понятие об H-индифферентных и Н-инвариантных тензорах, примеры.}

  \que{Группы симметрии сплошной среды. Принцип материальной симметрии. Определение жидких и твердых сред.}

\paragraph{Группы симметрии сплошной среды.}

\begin{definition*}
	Множество всех $H$-преобразований $\mathring{\mathcal{K}} \to \overset{\ast}{\mathcal{K}}$ отсчетной конфигурации $\mathring{\mathcal{K}}$ сплошной среды (или множество $\mathbf{H}$-тензоров) с определяющими соотношениями, для которого выполнены соотношения $\mathbf{H}$-преобразований, образует группу, называемую \textit{группой симметрии} $\mathring{G}_{s}$ сплошной среды. 
	
	Эту группу иногда называют \textit{группой эквивалентности.}
\end{definition*}

\begin{utv*}
	Можно показать, что множество всех $H$-преобразований является группой. Это делает Димитриенка на стр. 197.
\end{utv*}

\begin{axiom*}
	Для любой сплошной среды с определяющими соотношениями для движения $\mathring{\mathcal{K}} \to \mathcal{K}$ и с произвольной отсчетной конфигурацией $\mathring{\mathcal{K}}$ существует соответствующая группа симметрии $\mathring{G}_{s}$ --- группа $H$-преобразований отсчетной конфигурации $H : \mathring{\mathcal{K}} \to \overset{\ast}{\mathcal{K}}$, которые не изменяют определяющих соотношений, т.е. для любого движения $\overset{\ast}{\mathcal{K}} \to \mathcal{K}$ имеет место:
	\begin{equation*}
		\overset{\ast}{\Lambda} = \overset{\cup}{f}(\overset{\ast}{\mathcal{R}}).
	\end{equation*} 
\end{axiom*}

\paragraph{Определение жидких и твердых сред.} 

\begin{theorem*}
	Всякий $\mathbf{H}$-тензор, соответствующий $H$-преобразованию, является унимодулярным, т.е. удовлетворяет соотношению 
	\begin{equation*}
		\det{\mathbf{H}} = \pm 1.
	\end{equation*}
\end{theorem*}

\begin{theorem*}
	Группа симметрии $\mathring{G}_s$ любой сплошной среды является подгруппой полной унимодулярной группа:
	\begin{equation*}
		\mathring{G}_s \subset U.
	\end{equation*}
\end{theorem*}

\begin{definition*}
	Сплошную среду, которая для любой отсчетной конфигурации $\mathring{\mathcal{K}}$ имеет группу симметрии $\mathring{G}_s$, совпадающую с полной унимодулярной группой 
	\begin{equation*}
		\mathring{G}_{s} = U,
	\end{equation*}
	нызывают \textit{жидкостью} (жидкой средой).
\end{definition*}

\begin{definition*}
	Сплошную среду, для которой существует отсчетная конфигурация $\hat{\mathcal{K}}$, такая что ее группа симметрии $\hat{G}_s$ является подгруппой полной ортогональной группы
	\begin{equation*}
		\hat{G}_s \subset I,
	\end{equation*}
	называют \textit{твердой средой} (твердым телом). 
\end{definition*}

  \que{Основные группы симметрии твердых сред: группы изотропии, трансверсальной изотропии и ортотропии. Н-индифферентные функции относительно групп симметрии.}

  \que{Инварианты тензоров 2-го ранга относительно произвольной группы симметрии, функциональные базисы инвариантов. }

  \que{Инварианты тензоров 2-го ранга относительно  групп  изотропии, трансверсальной изотропии и ортотропии. Анизотропные среды, примеры. }

  \que{Представления определяющих соотношений для идеальных твердых сред c помощью инвариантов (случаи изотропии, трансверсальной изотропии, ортотропии).}


\end{document}
