\que{Общий вид определяющих соотношений сплошных сред  (модели  An ).  Определение идеальных сред. Общий вид определяющих соотношений для идеальных сплошных сред.}
\paragraph{Модель An.}
\[
  \begin{cases}
    \psi = \breve{\psi} ( \stackrel{(n)}{\mathbf{C}}, \theta ), \\
    \eta = \breve{\eta} ( \stackrel{(n)}{\mathbf{C}}, \theta ), \\
    \stackrel{(n)}{\mathbf{T}} = \breve{\mathcal{F}} ( \stackrel{(n)}{\mathbf{C}}, \theta ), \\
    w^* = \breve{w}^* ( \stackrel{(n)}{\mathbf{C}}, \theta ).
  \end{cases}
\]

\paragraph{Идеальные сплошные среды.}

\begin{definition}
  Сплошную среду называют \emph{идеальной}, если для нее
  соответствующие определяющие соотношения \eqref{thermodinamic-determinizm-principle} представляют собой
  обычные функции от активных переменных, т.е. имеют место соотношения
  \[
    \Lambda(t) = f(\mathcal{R}(t)),
  \]
  в которых значения $\Lambda(t)$ и $\mathcal{R}(t)$ рассматривают в одни и те
  моменты времени $t$.
\end{definition}

\paragraph{Определяющие соотношения для идеальных сплошных сред.}
