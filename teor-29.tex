\que{Принцип термодинамически согласованного детерминизма, принцип локальности.}

\paragraph{Аксиома принцип термодинамически согласованного детерминизма.}
Для любой сплошной среды активные переменные $\Lambda$ полностью определяются реактивными
переменными $\mathcal{R}$, иначе говоря, существует отображение обобщенного пространства
реактивных переменных $\chi_{\mathcal{R}}$ в пространсвто активных переменных $\chi_\Lambda$:
\begin{equation}
  \breve{f} : \chi_{\mathcal{R}} \to \chi_\Lambda,
\end{equation}
которое называют \emph{операторным соотношением (оператором)} и записывают в виде
\begin{equation}\label{thermodinamic-determinizm-principle}
  \Lambda = \breve{f} (\mathcal{R})
\end{equation}
, причём это соотношение <<согласовано с термодинамикой>>,
т.е. тождественно удовлетворяет ОТТ.

\begin{remark}
  Соотношение \eqref{thermodinamic-determinizm-principle} вообще говоря, просто
  устанавливает существование некоторой зависимости между этими наборами переменных, однако, это
  совсем не означает, что $\Lambda(t)$ зависит только от $\mathcal{R}(t)$.
\end{remark}

\paragraph{Принцип локальности.} Для всякой сплошной среды операторы
определяющих соотношений в любой из форм \eqref{thermodinamic-determinizm-principle} таковы, что
активные переменные в любой материальной точке зависят только от реактивных переменных в этой же
точке.
