\que{Основные группы симметрии твердых сред: группы изотропии, трансверсальной изотропии и ортотропии. Н-индифферентные функции относительно групп симметрии.}

\begin{definition*}
	Твердое тело называют \textit{изотропным}, если его группа симметрии $\hat{G}_s$ в неискаженной конфигурации $\hat{\mathcal{K}}$ совпадает с $I$:
	\begin{equation*}
		\hat{G}_s = I.
	\end{equation*}
\end{definition*}

Четыре наиболее широко используемые в МСС подгруппы:
\begin{itemize}
	\item триклинная группа $\hat{G}_s = E$ (точечная, состоит из 2 элементов), 
	
	\item группа ортотропии $\hat{G}_s = O$ (точечная, состоит из 8 элементов), 
	
	\item группа трансверсальной изотропии $\hat{G}_s = T_3$ (непрерывная),
	
	\item группа изотропии (полная ортогональная) $\hat{G}_s = I$ (непрерывная). 
\end{itemize}

\begin{definition*}
	Тензор $\Omega$, заданный в конфигурации $\mathring{\mathcal{K}}$, называют $H$-индифферентным относительно группы $\mathring{G}_s \subset I$, если для любого ортогонального тензора преобразования $\overset{\ast}{\mathbf{Q}} : \mathring{\mathcal{K}} \to \overset{\ast}{\mathcal{K}}$ из этой группы, то он удовлетворяет условию:
	\begin{equation*}
		\overset{\ast}{\Omega} = \overset{\ast}{\mathbf{Q}}^{T} \cdot \Omega \cdot \overset{\ast}{\mathbf{Q}}, \quad \forall \overset{\ast}{\mathbf{Q}} \in \mathring{G}_s.
	\end{equation*}
	
	ИНаче говоря, существуют тензоры, $H$-индефферентные относительно только определенных подгрупп ортогональной группы; тензоры же, которые $H$-индифферентны относительно любых $H$-преобразований будем называть \textit{абсолютно} $H$-\textit{индифферентными}.
\end{definition*}