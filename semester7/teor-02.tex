\que{Диадные  базисы, алгебраическое определение  тензора, геометрическое представление тензора }

\begin{definition}[алгебраическое, \textbf{из википедии}]
	Тензором типа $(^s _r)$ называется \textit{полилинейная функция} (полилинейная форма) $$T \colon  (V^*)^s \times V^r\to R,$$ то есть числовая функция от $s+r$ аргументов следующего вида $$T(f_1,f_2,...,f_s,x_1,x_2,...,x_r),$$ где $f_i$ -- линейные функционалы на $V$, а $x_j$ -- векторы пространства $V$.
	
	Координатами тензора в некотором базисе будут значения полилинейной функции на различных комбинациях базисных векторов: $T^{i_1i_2...i_s}_{j_1j_2...j_r}=T(e^{i_1},e^{i_2},...,e^{i_s},e_{j_1},e_{j_2},...,e_{j_r})$
\end{definition}
{\small
\begin{remark}[Определение полилинейной функции]
	% \textbf{Просто для понимания записал, чтобы в определении выше не раскидываться терминами. Думаю, что в ответ это не стоит писать.}
	
  Основным объектом полилинейной алгебры является полилинейное ($n$-линейное)
  отображение:
	 $$f : V_1 \times \dots \times V_n \rightarrow W,$$
  где $V_1, \dots, V_n$ и $W$ -- векторные пространства над определённым полем
  $K$. Условие $n$ -- линейности означает, строго говоря, что для каждого $i =
  1, \dots, n$ семейство отображений
	$$(\pi_if)_{\{x_k | k \ne i\}} : V_k \rightarrow W; \quad 
	(\pi_if)_{\{x_k | k \ne i\}}(x_i) = f(x_1, \dots, x_n),$$
  зависящее от $n - 1$ переменных $\{x_k | k \ne i\}$ как от параметров, состоит
  из линейных отображений. Можно также определить $n$-линейное отображение
  рекурсивно (по индукции), как <<линейное>> отображение из $V_n$ в векторное
  пространство $(n - 1)$-линейных отображений.
\end{remark}
}

%\begin{figure}[ht!]
%	\centering
%	\includesvg[scale = 0.8]{que2}
%	\caption{Геометрическое представление тензора}
%	\label{fig:que2}
%\end{figure}
\begin{definition}[геометрическое]
	Пусть
	\begin{enumerate}
		\item $\mathcal{L}_n$ --- линейное (векторное) пространство --- \emph{порождающее пространство}.
		Выберем две системы векторов в $\mathcal{L}_n$: $\mathbf{a}_i, \mathbf{b}^{[i]}, i = \overline{1, n}$.
		(квадратные скобки --- просто обозначение).
		
		\item Построим формальный векторный набор из $\mathbf{a}_i,
		\mathbf{b}^{[j]}$ длины $2n$.
		Векторы из $\mathbf{a}_i$ называем левыми, а из $\mathbf{b}^{[i]}$
		правыми. Таким образом\footnote{Далее скобки и запятые будут опускаться.}, 
		$A \equiv ((\mathbf{a}_1, \mathbf{b}^{[1]}), (\mathbf{a}_2,
		\mathbf{b}^{[2]}), \dots (\mathbf{a}_n, \mathbf{b}^{[n]}))
		\equiv ((\mathbf{a}_i, \mathbf{b}^{[i]}))$.
		Здесь у нас $ n $ пар векторов (и в каждой паре $ i $ есть левый $\mathbf{a}_i$ и правый вектор
		$\mathbf{b}^{[i]}$).
		
		\item Введем теперь операции с векторными наборами.
		\begin{enumerate}
			\item \textsc{Сложение} однотипных векторных наборов. \emph{Однотипными} будем называть такие наборы, у 
			которых совпадают либо все левые, либо все правые векторы:
			\[
			A_1 \equiv \mathbf{a}_i \mathbf{b}^{[i]} \leftrightarrow A_2 = \mathbf{a}_i \mathbf{c}^{[i]}; 
			\quad
			A_1 \leftrightarrow A_3 = \mathbf{d}_i \mathbf{b}^{[i]}.
			\]
			Тогда
			\[
			A_1+A_2 = \mathbf{a}_i (\mathbf{b}^{[i]} + \mathbf{c}^{[i]}); \quad
			A_1+A_3 = (\mathbf{a}_i + \mathbf{d}_i) \mathbf{b}^{[i]}.
			\]
      Это \emph{частичная операция}, то есть такая, которая определена только на
			подмножестве элементов (векторов). К тому же, как легко видеть, указанное
      отношение рефлексивно,
			симметрично, но не транзитивно, поэтому не может быть названа
			эквивалентностью.
			
			\item \textsc{Умножение} на число $s \in \mathbb{R}$.
			\[
			sA = (s \mathbf{a}_i) \mathbf{b}^{[i]} = \mathbf{a}_i (s \mathbf{b}^{[i]}).
			\]
			Последнее выражение показывает, что мы называем равными не только
			полностью совпадающие наборы.
			
			\item \textsc{Эквивалентность} векторных наборов.
			Векторные наборы $A$ и $B$ называются \emph{эквивалентными}, если выполняется хотя бы одно 
			из следующих условий:
			\begin{enumerate}
				\item Векторные наборы $A$ и $B$ состоят из одних и тех же пар, но в
          совокупности упорядоченных
				произвольным образом.
				
				Например,
				$A = \mathbf{a}_1 \mathbf{b}^{[1]} \mathbf{a}_2 \mathbf{b}^{[2]},
				\, B = \mathbf{a}_2 \mathbf{b}^{[2]} \mathbf{a}_1 \mathbf{b}^{[1]}$, откуда $A \sim B$.
				
				\item Набор $A$ может быть получен из другого набора с помощью согласованной операции
				умножения левых и правых векторов:
				\[
				A = \mathbf{a}_i \mathbf{b}^{[i]} \sim B = (s \mathbf{a}_i)
				(s^{-1}\mathbf{b}^{[i]})
				\quad \forall s \in \mathbb{R},\ s \neq 0.
				\]
				
				\item Если в $A$ и $B$ все векторы $\mathbf{a}_i$ и $\mathbf{b}^{[i]}$ совпадают, кроме тех пар,
				у которых хотя бы один вектор нулевой.
				Например,
				$A=\mathbf{a}_1 \mathbf{b}^{[1]} (\mathbf{a}_i \mathbf{0}) \sim B = \mathbf{a}_1 \mathbf{b}^{[1]} (\mathbf{0} \mathbf{b}^{[2]}) \sim \mathbf{a}_1 \mathbf{b}^{[1]} (\mathbf{c}_{2} \mathbf{0})$.
			\end{enumerate}
			
			\item Пусть теперь есть некоторый векторный набор $A = \mathbf{a}_i
			\mathbf{b}^{[i]}$. Введём 
			множество всех векторных наборов $B = \mathbf{c}_i \mathbf{d}^{[i]}$, эквивалентных $A$ и
			обозначим его $T = [A] = [\mathbf{a}_i \mathbf{b}^{[i]}]$. \end{enumerate}
		Таким
		образом определён
		\emph{тензор второго ранга}.
		
		\item \textsc{Диады и базисные диады}. Пусть $A = \mathbf{a}_i
		\mathbf{b}^{[i]}$, где
		существует не более чем одна пара ненулевых векторов $\mathbf{a}_i
		\mathbf{b}^{[i]}$. Положим по определению
		\[
		[\mathbf{a}_1 \mathbf{b}^{[1]} \mathbf{0} \mathbf{0} \dots \mathbf{0} \mathbf{0}] = \mathbf{a}_1 \otimes \mathbf{b}^{[1]}; \quad
		[\mathbf{0} \mathbf{0} \mathbf{a}_2 \mathbf{b}^{[2]} \dots \mathbf{0} \mathbf{0}] = \mathbf{a}_2 \otimes \mathbf{b}^{[2]}; \quad
		[\mathbf{0} \mathbf{0} \mathbf{0} \mathbf{0} \dots \mathbf{a}_1 \mathbf{b}^{[1]}] = \mathbf{a}_n \otimes \mathbf{b}^{[n]};
		\]
		
		Иными словами, мы научились по любой паре векторов конструировать диаду. Пусть $\mathbf{e}_i$ и
		$\mathbf{h}_j$ -- базисы в $\mathcal{L}_n$. Первый выберем в качестве левых векторов, а
		второй --- правых. Набор $[\mathbf{e}_1 \mathbf{0} \mathbf{e}_2 \mathbf{0}
		\dots \mathbf{e}_i \mathbf{h}_j \dots \mathbf{e}_n \mathbf{0}] =
		\mathbf{e}_i \otimes \mathbf{h}_j$ --- базисная диада. (Вместо всех
		$\mathbf{e}_k, k\neq i$ можно было поставить нули.) 
		
		\item \textsc{Диадный базис.}
		\begin{equation*}
			\mathbf{h}_j = \mathbf{e}_j \longrightarrow \mathbf{e}_i \otimes
			\mathbf{e}_j.
		\end{equation*}
		
		\begin{theorem}
			Любой тензор второго ранга $ T = [\mathbf{a}_i \mathbf{b}^{[i]}]$ можно представить в виде линейной комбинации базисных диад
			\begin{equation}\label{lec_2:eq:tensor_basis}
				T = T^{ij} \mathbf{e}_i \otimes \mathbf{e}_j,
			\end{equation}
			причём
			\[
			T = (T^{ij} \mathbf{e}_i) \otimes \mathbf{e}_j
			= [\mathbf{a}^{[j]} \mathbf{e}_j]
			= [\mathbf{e}_i \mathbf{b}^{[i]}].
			\]
		\end{theorem}
		\begin{corollary*}
			Множество всех тензоров (второго ранга) образует линейное пространство,
			где сложение векторов подчиняется правилам
			\begin{gather*}
				T_1 = [\mathbf{a}_i \mathbf{b}^{[i]}], \quad
				T_2 = [\mathbf{c}_j \mathbf{d}^{[j]}]; \\
				\mathbf{a}_i = a^j_i \mathbf{e}_j, \quad
				\mathbf{b}^{[i]} = b^{ik} \mathbf{e}_k,\quad
				\mathbf{c}_i = c^{j}_{i} \mathbf{e}_j, \quad
				\mathbf{d}^{[i]} = d^{ik} \mathbf{e}_k,  \\
				T_1 = [(a^i \mathbf{e}_i) (b^{[ik]} \mathbf{e}_k)] = a^j_i, \quad
				T_1 + T_2 = (a^j_i b^{ik} + c^{j}_i d^{jk}_i) \mathbf{e}_j \otimes \mathbf{e}_k
				= (T_1^{ij} + T_2^{ij}) \mathbf{e}_i \otimes \mathbf{e}_j,
			\end{gather*}
			а базисные диады образуют базис в тензорном пространстве.
		\end{corollary*}
	\end{enumerate}
\end{definition}

