\que{Лагранжево и эйлерово описания движения сплошных сред. Актуальная  и отсчетные конфигурации. Локальные базисы и метрические матрицы   в конфигурациях.}
\paragraph{Лагранжево и эйлерово описание.} Пусть дана сплошная среда $ \mathcal
P$ и некоторая её материальная точка $ \mathcal
M$ с радиусом-вектором $ \mathbf{x} $. 

Движение этой материальной точки может
рассматриваться в обычном декартовом базисе $ O\mathbf{\bar{e}}_i $, то есть как
вектор-функция $ \mathbf{x}(t) =
x^i(t)\mathbf{\bar{e}}_i  $. Такой подход к описанию движения называют
\emph{Эйлеровым}.

Другой подход состоит в том, чтобы связать с телом $ \mathcal{P} $ систему
криволинейных координат $ X^i $, которая будет <<двигаться>> вместе с телом
так, что координаты $ X^i $ точки $ \mathcal M $ в любой момент времени будут
одни и те же. Эти криволинейные координаты принадлежат телу только в случае,
если принадлежат некоторой области изменения, $ X^k \in V_X $. Этот подход
называется \emph{Лагранжевым}.

От криволинейных координат требуется регулярность, поэтому существуют следующие
соотношения: 
\[
  \mathbf{x} = \mathbf{x}(X^k, t), \qquad X^k = X^k(x^i, t).
\]
Здесь под $ X^k $ подразумевается полный набор криволинейных координат (в
трёхмерном пространстве их количество варьируется от одной до трёх).
Естественно, эти функции определены, только если $ X^k \in V_X $, $ \mathbf x \in
\mathcal P(t) $.

Соответственно, скалярные, векторные и тензорные поля (в том числе переменные)
тоже можно раскрывать как функции
криволинейных координат и времени либо же декартовых координат и времени. При
этом для твёрдых тел чаще используют лагранжево описание (следят, как меняется
фиксированная точка тела), а для жидкостей и газов --- Эйлерово (следят, как
материальные точки тела <<проходят>> через фиксированную точку пространства).

\paragraph{Актуальная и отсчётная конфигурации.} Для сплошной среды $ \mathcal P
$ определены отображения в точечно-евклидово пространство (типа строим мат. модель) $
V = V(\mathcal P, t)$, причём $ V $ является замкнутым множеством без
изолированных точек (т. н. \emph{совершенное} тело). 

Собственно, \emph{отсчётной} конфигурацией называется множество $ \mathring V =
V(\mathcal P,
0)$, а \emph{актуальной} конфигурацией --- множество $ V = V(\mathcal P, t_0) $ для
некоторого интересующего нас момента $ t_0 > 0 $.

\paragraph{Локальные базисы и метрические матрицы в конфигурациях.}\label{Локальные базисы и метрические матрицы в конфигурациях} С
криволинейными координатами $ X^i $ (точнее, с обратным отображением $ \mathbf
x(X^i) $) связана \emph{матрица Якоби} $
\tensor{Q}{^i_j} = \left( \frac{\partial x^i}{\partial X^j}\right)  $. Причём
предполагается, что Якобиан ($ \det Q $) отличен от нуля (условия регулярности),
то есть имеет обратную матрицу, которую назовём $ \tensor{P}{^i}{_j} $.
Заметим, что, вообще говоря\footnote{Если криволинейные координаты не являются
прямолинейными, то есть преобразование не линейное.}, эта матрица зависит от точки тела $ \mathcal
P$ и от времени (поскольку сами отображения $ \mathbf{x}(X^i, t) $ зависят в том
числе от времени).

Для фиксированного момента времени и точки $ \mathcal M \in \mathcal P $ назовём
столбцы этой матрицы \emph{локальным базисом} в точке $ \mathcal M $, обозн. $
\mathring{\mathbf{r}}_k $ для нулевого момента времени и $ \mathbf{r}_k $ иначе. 

Каждому локальному базису соответствует \emph{метрический тензор} $ g_{ij} =
\mathbf{r}_i \cdot \mathbf{r}_j $ (аналогично для $ \mathring g_{ij} $, далее не
подчёркивается) и
\emph{обратный метрический тензор} $ g^{ij} = g^{-1}_{ij} $. После последнего
определения нам стало доступно <<жонглирование>> индексами, в том числе
локального базиса. Именно, определим \emph{взаимный локальный базис $
\mathbf{r}^i $} по формуле $ \mathbf{r}^i = g^{ij}\mathbf{r}_j $.

Частный случай, когда векторы $ \mathbf{r}_i $ ортогональны, то есть матрица $
g_{ij} $ диагональна, дополняется определением \emph{коэффициентов Ламэ} $
H_\alpha $ как $ H_\alpha = \sqrt{g_{\alpha\alpha}} $, то есть длины
соответствующего базисного вектора.
