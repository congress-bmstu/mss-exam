\que{Физический  смысл компонент тензора  деформаций.  Преобразование
ориентированной площадки при деформации сплошной среды. Геометрическая картина
преобразования малой окрестности.}
\paragraph{Физический смысл.} Имеем  
\[
  \varepsilon_{\alpha\beta} =
  \frac{1}{2}(|\mathbf{r}_\alpha||\mathbf{r}_\beta|\cos\psi_{\alpha\beta} -
  |\mathring{\mathbf{r}}_\alpha||
  \mathring{\mathbf{r}}_\beta|\cos\mathring{\psi}_{\alpha\beta}).
\]
Далее нестандартный анализ. Введём длины элементарных радиусов-векторов: $ ds^2
= d\mathbf{x}\cdot d\mathbf{x}$, $ d\mathring{s}^2 = d\mathring{\mathbf{x}}\cdot
d\mathring{\mathbf{x}}$.

При этом $ d\mathring{\mathbf{x}} $ будем выбирать вдоль одного из векторов
$\mathring{\mathbf{r}}_\alpha$. Тогда $ d\mathbf{x} $ тоже будет направлен вдоль
$ \mathbf{r}_\alpha $. В этом случае  
\begin{align*}
  |d\mathring{\mathbf{x}}| &= d\mathring{s}_\alpha = \left| \frac{\partial
  \mathring{\mathbf{x}}}{\partial X^\alpha} dX^\alpha \right| =
  |\mathring{\mathbf{r}}_\alpha | dX^\alpha,\\
    |d\mathbf{x}| &= ds_\alpha = \left| \frac{\partial x}{\partial
  X^\alpha}dX^\alpha \right| = |\mathbf{r}_\alpha | dX^\alpha.
\end{align*}
Отсюда находим  
\[
  ds_\alpha/d\mathring{s}_\alpha =
  |\mathbf{r}_\alpha|/|\mathring{\mathbf{r}}_\alpha| = \delta_\alpha + 1,
\]
где $ \delta_\alpha $ называют \emph{относительным удлинением}. Получаем  
\[
  |\mathring{\mathbf{r}}_\alpha| = |\mathring{\mathbf{r}}_\alpha|(1 +
  \delta_\alpha)
\]
и  
\begin{align*}
  \varepsilon_{\alpha\beta} &= \frac{1}{2}
  |\mathring{\mathbf{r}}_\alpha||\mathring{\mathbf{r}}_\beta|
  ((1+\delta_\alpha)(1+\delta_\beta)\cos\psi_{\alpha\beta} -
  \cos\mathring{\psi}_{\alpha\beta}), \\
  \varepsilon_{\alpha\alpha} &=
  \frac{1}{2}|\mathring{\mathbf{r}}_\alpha|^2((1+\delta_\alpha)^2 - 1) =
  \frac{\mathring{g}_{\alpha\alpha}}{2}((1+\delta_\alpha)^2 - 1).
\end{align*}

Если $ X^i $ есть декартовы координаты, то $ \mathring{g}_{\alpha\beta} =
\delta_{\alpha\beta} $, и при $ \delta_\alpha \ll 1 $ имеем $
\varepsilon_{\alpha\alpha} \approx \delta_\alpha $. При этом  
\[
  \varepsilon_{\alpha\beta} =
  \frac{1}{2}(1+\delta_\alpha)(1+\delta_\beta)\sin\chi_{\alpha\beta},
\]
где $ \chi_{\alpha\beta} = \mathring{\psi}_{\alpha\beta} - \psi_{\alpha\beta} =
(\pi/2) - \psi_{\alpha\beta}$ --- изменение угла между базисными векторами $
\mathbf{r}_\alpha $ и $ \mathbf{r}_\beta $. При $ \delta_\alpha \ll 1 $, $
\chi_{\alpha\beta} \ll 1 $ имеем $ \varepsilon_{\alpha\beta} \approx
\chi_{\alpha\beta}/2 $.

\paragraph{Преобразование ориентированной площадки.} Далее какой-то бред.
Рассмотрим некоторую гладкую поверхность $ \Sigma $, которой принадлежат
какие-либо две из координатных линий $ X^\alpha $ и $ X^\beta $.

Тогда можно ввести вектор нормали  
\[
  \mathbf{n} = \frac{1}{\sqrt{\tilde{g}}} \mathbf{r}_\alpha \times
  \mathbf{r}_\beta,
\]
где $ \tilde g $ --- определитель метрической матрицы, составленной только из
выбранных векторов $ \mathbf{r}_\alpha $, $ \mathbf{r}_\beta $. Эта нормаль
является единичной.

Рассмотрим элементарную площадку $ d\Sigma $, построенную на элементарных
радиусах-векторах $ d\mathbf{x}_\alpha $, направленных по векторам локального
базиса, то есть $ d\mathbf{x}_\alpha = \mathbf{r}_\alpha dX^\alpha $. Назовём
величину  
\[
  d\Sigma = \sqrt{\tilde g} dX^\alpha dX^\beta
\]
\emph{площадью элементарной площадки} $ d\Sigma $ (обозначение супер),
построенной на векторах $ d\mathbf{x}_\alpha $, $ d\mathbf{x}_\beta $. Тогда  
\[
  \mathbf{n} d\Sigma = \mathbf{r}_\alpha dX^\alpha \times \mathbf{r}_\beta
  dX^\beta = d\mathbf{x}_\alpha \times d\mathbf{x}_\beta.
\]
Величину $ \mathbf{n}d\Sigma $ называют \emph{ориентированной площадкой}.

Площадке $ d\Sigma $ соответствует площадка $ d\mathring{\Sigma} $, построенная
на $ d\mathring{\mathbf{x}}_\alpha $, $ d\mathring{\mathbf{x}}_\beta$, 
\[
  \mathring{\mathbf{n}}d\mathring{\Sigma} = \mathring{\mathbf{r}}_\alpha
  dX^\alpha \times \mathring{\mathbf{r}}_\beta dX^\beta =
  \mathring{\mathbf{r}}_\alpha \times \mathring{\mathbf{r}}_\beta dX^\alpha
  dX^\beta.
\]

Так как $ \mathbf{r}^\gamma = F^{-1\mathsf T}\cdot \mathring{\mathbf{r}}^\gamma
$, то  
\begin{equation}\label{eq:govno}
  \mathbf{n}d\Sigma = \sqrt{g}\epsilon_{\alpha\beta\gamma}F^{-1\mathsf T} \cdot
  \mathring{\mathbf{r}}^\gamma dX^\alpha dX^\beta =
  \sqrt{g/\mathring{g}} F^{-1 \mathsf T}\cdot
  \mathring{\mathbf{r}}_\alpha\times\mathring{\mathbf{r}}_\beta dX^\alpha
  dX^\beta = 
  \sqrt{g/\mathring{g}} F^{-1\mathsf
  T}\cdot \mathring{\mathbf{n}}d\mathring{\Sigma}.
\end{equation}
Умножим это уравнение скалярно само на себя и получим  
\[
  d\Sigma^2 = \frac{g}{\mathring{g}}(\mathring{\mathbf{n}}\cdot F^{-1}\cdot
  F^{-1\mathsf T}\cdot \mathring{\mathbf{n}})d\mathring{\Sigma}^2 =
  \frac{g}{\mathring g}(\mathring{\mathbf{n}}\cdot G^{-1}\cdot
  \mathring{\mathbf{n}})d\mathring{\Sigma}^2,
\]
то есть 
\[
  d\Sigma/d\mathring{\Sigma} = \sqrt{g/\mathring{g}}(\mathring{\mathbf{n}}\cdot
  G^{-1}\cdot \mathring{\mathbf{n}})^{1/2}.
\]

С другой стороны, из той же формулы \eqref{eq:govno} можно сначала выразить $
\mathring{\mathbf{n}} $, а затем получившееся соотношение умножить скалярно само
на себя. Тогда  
\[
  d\mathring{\Sigma}^2 = \frac{\mathring{g}}{g}(\mathbf{n}\cdot F\cdot
  F^{\mathsf T}\cdot \mathbf{n})d\Sigma^2.
\]
Отсюда  
\[
  d\mathring{\Sigma}/d\Sigma = \sqrt{\mathring g/g}(\mathbf{n}\cdot g^{-1}\cdot
  \mathbf{n})^{1/2}.
\]
Иными словами,  
\[
  (\mathbf{n}\cdot g^{-1}\cdot \mathbf{n})^{1/2} = (\mathring{\mathbf{n}}\cdot
  G^{-1}\cdot \mathring{n})^{-1/2}.
\]
После подстановки полученных выражений в \eqref{eq:govno} получаем  
\[
    (\mathring{\mathbf{n}}\cdot
    G^{-1}\cdot \mathring{n})^{1/2} \mathbf{n} = F^{-1\mathsf T}\cdot
    \mathring{\mathbf{n}}, \qquad (\mathbf{n}\cdot g^{-1}\cdot \mathbf{n})^{1/2}
    \mathring{\mathbf{n}} = F^{\mathsf T} \cdot \mathbf{n}.
\]

\paragraph{Геометрическая картина преобразования малой окрестности.} 
Известно соотношение
\[ 
  d \mathbf{x}=\mathbf{F} \cdot d\mathring{\mathbf{x}}.
\]

Запишем это соотношение в декартовых координатах
\[ d x^{i}=\bar{F}_{m}^{i} d \mathring{x}^{m}, \]
где $\bar{F}_{m}^{i}$ - компоненты градиента деформации в декартовом базисе:
\[ \bar{F}_{m}^{i}=\left(\partial x^{i} / \partial \mathring{x}^m\right). \]
Фактически, это аффинное преобразование.
Из общих свойств аффинных преобразований следует, параллелограммы при аффинном
преобразовании перейдут в параллелограммы, а сферы перейдут в эллипсоиды.

Отношение длин $d s_{\alpha} / d\mathring{s}$
произвольного отрезка (элементарного радиуса-вектора $d \mathbf{x}$ в
$\mathcal{K}$ и $\mathring{\mathcal{K}}$ ) не зависит от начальной
длины $d \mathring{s}$ этого отрезка (так как относительное удлинение
$\delta_{\alpha}$ не зависит от $d s_{\alpha}$ ).

Согласно полярному разложению, указанное преобразование из $\mathring{\mathcal{K}}$ в
$\mathcal{K}$ всегда можно представить суперпозицией двух преобразований: 
\[
  d\mathbf{x} = O \cdot d\mathring{\mathbf{x}}', \qquad d\mathring{\mathbf{x}}'
  = U \cdot d\mathring{\mathbf{x}},
\]
или

\[ d \mathbf{x}=\mathbf{V} \cdot d \mathbf{x}^{\prime}, \quad d
\mathbf{x}^{\prime}=\mathbf{O} \cdot d \mathbf{x} \]

Тензор искажений $\mathbf{U}$, обладающий тремя собственными направлениями
$\mathring{\mathbf{p}}_{\alpha}$, изменяет малую окрестность точки
$\mathcal{M}$, сжимая или растягивая ее вдоль этих трех направлений
$\mathring{\mathbf{p}}_{\alpha}$. Тензор поворота $ O $ поворачивает
деформированную вдоль $\mathring{\mathbf{p}}_{\alpha}$ окрестность
<<жестким образом>>, переводя направление $\mathring{\mathbf{p}}_{\alpha}$
в $\mathbf{p}_{\alpha}$. Если используем левый тензор искажений $\mathbf{V}$, то
вначале осуществляется поворот осей $\mathring{\mathbf{p}}_{\alpha}$ в
$\mathring{\mathcal{K}}$ до их совпадения с $\mathbf{p}_{\alpha}$ (с
точностью до параллельного переноса), а затем сжатие/растяжение малой
окрестности вдоль направления $\mathbf{p}_{\alpha}$. Результат, очевидно, будет
одинаковым.

Если точка $\mathcal{M}_{\alpha}$ связана с $\mathcal{M}$ радиусом-вектором $d
\mathbf{x}_{\alpha}$, ориентированным по собственному направлению
$\mathring{\mathbf{p}}_{\alpha}$ (заметим, заранее до деформации
неизвестному), то в $\mathcal{K}$ эта точка $\mathcal{M}_{\alpha}$ будет связана
с $\mathcal{M}$ радиусом-вектором $d \mathbf{x}_{\alpha}$, ориентированным вдоль
соответствующего собственного направления $\mathbf{p}_{\alpha}$.

Если малую окрестность точки $\mathcal{M}$ в $\mathring{\mathcal{K}}$
взять в виде сферы, то в $\mathcal{K}$ эта сфера перейдет в
эллипсоид, главные оси которого направлены по собственным направлениям
$\mathbf{p}_{\alpha}$.

Таким образом, преобразование малой окрестности каждой точки $\mathcal{M}$
сплошной среды при деформации всегда можно представить в виде растяжения/сжатия
вдоль собственных направлений и поворота как жесткого целого, а также
перемещения как жесткого целого.
