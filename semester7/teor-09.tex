\que{Градиент деформаций. Тензоры деформации Альманзи и Коши-Грина. }
\paragraph{Градиент деформаций.}
\begin{definition*}
  \emph{Градиентом деформации} называется тензор второго ранга 
  \begin{equation}\label{eq:F2}
    F = \mathbf{r}_k \otimes \mathring{\mathbf{r}}^k.
  \end{equation}
\end{definition*}
Дальше нестандартный анализ. Пусть $ \mathbf{x} $ --- радиус-вектор точки $
\mathcal M $ тела $ \mathcal P $. Градиент деформации отвечает за то, как
изменяется элементарный
вектор $ d\mathbf{x} $, соединяющий бесконечно близкие точки $ \mathcal M $ и $
\mathcal M'$: 
\begin{equation}\label{eq:F}
  d\mathbf{x} = F d\mathbf{\mathring{x}}.
\end{equation}
Действительно, $ d\mathbf{x} = \mathbf{r}_k dX^k $. Тогда $ \mathbf{r}^m\cdot
d\mathbf{x} = dX^m = \mathring{\mathbf{r}}^m\cdot d\mathbf{\mathring{x}} $ и,
наконец, $ d\mathbf{x} = \mathbf{r}_k (\mathbf{r}^k\cdot d\mathring{\mathbf{x}})
= \mathbf{r}_k
\otimes\mathbf{\mathring{r}}^k \cdot
d\mathbf{\mathring{x}}$ (типа умножение на скаляр --- то же самое, что и
тензорное умножение на скаляр). В каком порядке производить тензорное и
скалярное умножение неважно, поэтому приходим к изначальной формуле
\eqref{eq:F}. Из формулы \eqref{eq:F2} же непосредственно вытекает, что  
\[
  F \cdot \mathring{\mathbf{r}}_i = \mathbf{r}_k \delta^k_i = \mathbf{r}_i.
\]

В мсс часто используются следующие вариации градента деформации: 
\begin{gather*}
  F^{\mathsf T} = \mathring{\nabla}\otimes \mathbf x, \qquad F^{-1} =
  \mathring{\mathbf{r}}_k \otimes \mathbf{r}^k, \\
  F^{-1\mathsf T} = \nabla \otimes \mathring{\mathbf{x}}.
\end{gather*}
$ F^{\mathsf T} $ связывает уже градиенты произвольного вектора в
разных конфигурациях: 
\[
  \mathring{\nabla}\otimes \mathbf{a} = F^{\mathsf T}\cdot \nabla \otimes
  \mathbf{a}.
\]
Это правда, 
\[
  \nabla \otimes \mathbf{a} = \mathbf{r}^i \otimes \frac{\partial
  \mathbf{a}}{\partial X^i} = \mathbf{r}^j \delta^{i}_{j} \otimes \frac{\partial
\mathbf{a}}{\partial X^i} = \mathbf{r}^j \otimes \mathring{\mathbf{r}}_j \cdot
\mathring{\mathbf{r}}^i \otimes \frac{\partial \mathbf{a}}{\partial X^i} =
F^{-1\mathsf T} \cdot \mathring{\nabla}\otimes \mathbf{a}.
\]


\paragraph{Тензоры деформации.}
\begin{definition*}
 \begin{align*}
   C &= \varepsilon_{ij} \mathring{\mathbf{r}}^i \otimes
   \mathring{\mathbf{r}}^j = \frac{1}{2}(F^{\mathsf T}\cdot F - E) & (\emph{правый тензор деформации Коши -- Грина}),\\
   A &= \varepsilon_{ij}\mathbf{r}^i \otimes \mathbf{r}^j =
   \frac{1}{2}(E-F^{-1\mathsf T}\cdot F^{-1}) & (\emph{левый
   тензор деформации Альманзи}),\\
   \Lambda &= \varepsilon^{ij}\mathring{\mathbf{r}}_i \otimes
   \mathring{\mathbf{r}}_j = \frac{1}{2} (E-F^{-1}\cdot F^{-1\mathsf T})& (\emph{правый тензор тензор деформации Альманзи}),\\
   J &= \varepsilon^{ij} \mathbf{r}_i \otimes \mathbf{r}_j = \frac{1}{2}(F\cdot
   F^{\mathsf T} - E)& (\emph{левый
   тензор деформации Коши -- Грина}).
 \end{align*}
\end{definition*}
Здесь  
\[
  \varepsilon_{ij} = \frac{1}{2}(g_{ij} - \mathring{g}_{ij}), \quad
  \varepsilon^{ij} = \frac{1}{2}(\mathring{g}^{ij} - g^{ij}).
\]
\begin{remark*}
  Заметим, что $ \varepsilon^{kl} \neq \varepsilon_{ij}g^{ik}g^{jl} $ и
  наоборот. Эти компоненты определены независимо и не должны подвергаться
  жонглированию.
\end{remark*}
\begin{proof}
  Докажем, например, первое соотношение. 
  \[
    C = \varepsilon_{ij} \mathring{\mathbf{r}}^i \otimes \mathring{\mathbf{r}}^j
    = \frac{1}{2}(g_{ij} - \mathring{g}_{ij}) \mathring{\mathbf{r}}^i \otimes
    \mathring{\mathbf{r}}^j = \frac{1}{2}(\mathbf{r}_{i}\cdot\mathbf{r}_j
    \otimes \mathring{\mathbf{r}}^i \otimes \mathring{\mathbf{r}}^j - E) =
    \frac{1}{2}(\mathring{\mathbf{r}}^i \otimes \mathbf{r}_i \cdot \mathbf{r}_j
    \otimes \mathring{\mathbf{r}}^j - E).
  \]
\end{proof}
