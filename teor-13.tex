\que{Вектор скорости, конвективная производная,  кинематическое  соотношение.
Тензор скоростей деформации. }
\paragraph{Вектор скорости.}
\begin{definition*}
  \emph{Вектором скорости} называется
\begin{equation*}
\mathbf{v}\left(X^{i}, t\right)=\left.\frac{\partial \mathbf{x}}{\partial
t}\left(X^{i}, t\right)\right|_{X^{i}} \quad \text{(типа зафиксировали точку
тела)}.
\end{equation*}
\end{definition*}

При этом
\begin{equation*}
\mathbf{v}=\bar{v}^{i} \overline{\mathbf{e}}_{i}=\frac{\partial x^{i}}{\partial
t} \overline{\mathbf{e}}_{i}, \quad \bar{v}^{i}=\frac{\partial x^{i}}{\partial
t}\left(X^{j}, t\right).
\end{equation*}


\paragraph{Полная производная тензора по времени.} Напомним, что
произвольное векторное (а также скалярное и тензорное) поле может быть
представлено
\begin{equation*}
\mathbf{a}(\mathbf{x}, t)=\mathbf{a}\left(\mathbf{x}\left(X^{j}, t\right), t\right)
\end{equation*}
в двух описаниях.

\begin{definition*}
  Полной производной по времени от переменного векторного поля $ \mathbf{a} $
  называют частную производную по $t$ при фиксированных значениях координат
  $X^{i}$:
\begin{equation*}
\dot{\mathbf{a}} \equiv \frac{d \mathbf{a}}{d t}=\left.\frac{\partial
  \mathbf{a}}{\partial t}\right|_{X^{i}} = \left.\frac{\partial
    \mathbf{a}}{\partial t}\right|_{x^{i}}+\left.\frac{\partial
  \mathbf{a}}{\partial x^{j}} \frac{\partial x^{j}}{\partial t}\right|_{X^{i}}.
\end{equation*}
\end{definition*}

Заметим, что
\begin{equation*}
\frac{\partial \mathbf{a}}{\partial x^{j}} \frac{\partial x^{j}}{\partial
t}=\bar{v}^{j} P_{j}^{k} \frac{\partial \mathbf{a}}{\partial X^{k}}=\bar{v}^{i}
\overline{\mathbf{e}}_{i} \cdot \overline{\mathbf{e}}^{j} P_{j}^{k} \otimes
\frac{\partial \mathbf{a}}{\partial X^{k}}=\mathbf{v} \cdot \mathbf{r}^{k}
\otimes \frac{\partial \mathbf{a}}{\partial X^{k}}=\mathbf{v} \cdot \nabla
\otimes \mathbf{a},
\end{equation*}
откуда
\begin{equation*}
\frac{d \mathbf{a}}{d t}=\frac{\partial \mathbf{a}}{\partial t}+\mathbf{v} \cdot
\boldsymbol{\nabla} \otimes \mathbf{a}.
\end{equation*}

\begin{definition*}
Выражение $\mathbf{v} \cdot \boldsymbol{\nabla} \otimes \mathbf{a}$ называют
\emph{конвективной производной}.
\end{definition*}

Конвективная производная характеризует изменение поля
за счет перемещения материальной частицы $\mathcal{M}$ из точки $\mathbf{x}$ в
точку $\mathbf{x}+\mathbf{v} d t$ пространства.

Например,
\begin{equation*}
\mathbf{v}=\frac{d \mathbf{x}}{d t}=\frac{d \mathbf{u}}{d t}=\frac{\partial \mathbf{u}}{\partial t}+\mathbf{v} \cdot \nabla \otimes \mathbf{u} 
\end{equation*}

\paragraph{Дифференциал тензора.}
\begin{definition*}
Дифференциалом переменного тензорного поля (дифференциалом тензора) ${ }^{n}
\Omega\left(x^{i}, t\right)$ называют следующий объект:
\begin{equation*}
d^{n} \boldsymbol{\Omega}=\frac{d^{n} \Omega}{d t} d t.
\end{equation*}
\end{definition*}

Используя формулу для полной производной тензора по времени, получаем, что дифференциал тензора можно представить в виде
\begin{equation*}
d^{n} \boldsymbol{\Omega}\left(x^{i}, t\right)=\left(\frac{\partial^{n} \boldsymbol{\Omega}}{\partial t}+\mathbf{v} \cdot \boldsymbol{\nabla} \otimes{ }^{n} \boldsymbol{\Omega}\right) d t 
\end{equation*}

Перепишем это выражение в виде
\begin{equation*}
d^{n} \boldsymbol{\Omega}=\frac{\partial^{n} \boldsymbol{\Omega}}{\partial t} d t+d \mathbf{x} \cdot \boldsymbol{\nabla} \otimes{ }^{n} \boldsymbol{\Omega}
\end{equation*}

В случае стационарных тензорных полей, т.е. когда $\partial^{n} \Omega / \partial t=0$, дифференциал тензорного поля имеет следующий вид:
\begin{equation*}
\widehat{d}^{n} \boldsymbol{\Omega}=d \mathbf{x} \cdot \boldsymbol{\nabla} \otimes{ }^{n} \boldsymbol{\Omega}
\end{equation*}

Для дифференциала вектора имеем 
\begin{equation*}
d \mathbf{a}\left(X^{i}, t\right)=\frac{d \mathbf{a}}{d t} d t=\left(\frac{\partial \mathbf{a}}{\partial t}+\mathbf{v} \cdot \boldsymbol{\nabla} \otimes \mathbf{a}\right) d t
\end{equation*}

Кроме того, по определению полагаем (случай стационарного векторного поля)
\begin{equation*}
d \hat{\mathbf{a}}=(\mathbf{v} \cdot \boldsymbol{\nabla} \otimes \mathbf{a}) d t=(\boldsymbol{\nabla} \otimes \mathbf{a})^{\mathrm{T}} \cdot d \mathbf{x} \text {. }
\end{equation*}


В частности, если $\mathbf{a}=\mathring{\mathbf{x}}$, то имеем
\begin{equation*}
d \widehat{\mathring{\mathbf{x}}}=(\boldsymbol{\nabla} \otimes
\stackrel{\circ}{\mathbf{x}})^{\mathrm{T}} \cdot d \mathbf{x}=\mathbf{F}^{-1}
\cdot d \mathbf{x},
\end{equation*}
то есть получили тот же дифференциал, что использовали ранее.

\paragraph{Градиент скорости и тензор скоростей деформации.} Рассмотрим дифференциал вектора скорости $\widehat{d}
\mathbf{v}$: 
\[
  \hat{d}\mathbf{v} = \frac{\partial}{\partial t} d\mathbf{x} = \frac{\partial^2
  \mathbf{x}}{\partial X^i \partial t} dX^i = \frac{\partial^2
\mathbf{x}}{\partial X^i \partial t}\otimes \mathring{\mathbf{r}}^i\cdot
d\mathring{\mathbf{x}} = \left( \mathring{\mathbf{r}}^i\otimes \frac{\partial
\mathbf{v}}{\partial X^i} \right)^{\mathsf T} \cdot d\mathring{\mathbf{x}} =
(\mathring{\nabla}\otimes\mathbf{v})^{\mathsf T} \cdot d\mathring{\mathbf{x}}.
\]

Аналогично, используя уравнение $d X^{i}=\mathbf{r}^{i} \cdot
d \mathbf{x}$, получим еще одно представление для вектора $\widehat{d \mathbf{v}}$ :
\begin{equation*}
\widehat{d} \mathbf{v}=(\boldsymbol{\nabla} \otimes \mathbf{v})^{\mathrm{T}} \cdot d \mathbf{x} 
\end{equation*}

\begin{definition*}
Тензор второго ранга $(\boldsymbol{\nabla} \otimes \mathbf{v})^{\mathrm{T}}$
называют \emph{градиентом скорости}.
\end{definition*}

Тензор $\mathbf{L}$, как и всякий тензор второго ранга, можно представить в виде суммы симметричного тензора $\mathbf{D}$ и кососимметричного $\mathbf{W}$:
\begin{equation*}
\mathbf{L}=\mathbf{D}+\mathbf{W} 
\end{equation*}

\begin{definition*}
  Симметричный \emph{тензор скоростей деформации} $\mathbf{D}$ определяют
  следующим образом:
\begin{equation*}
\mathbf{D}=\frac{1}{2}\left(\boldsymbol{\nabla} \otimes \mathbf{v}+\boldsymbol{\nabla} \otimes \mathbf{v}^{\mathrm{T}}\right) 
\end{equation*}
\end{definition*}
Этот тензор имеет 6 независимых компонент.
