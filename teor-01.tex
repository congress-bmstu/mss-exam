\que{Локальные  векторные базисы, метрические матрицы, преобразования  координат.}
\paragraph{ Координаты и локальные векторы базиса. Локальные векторные базисы, метрические матрицы, преобразования координат.}
Введём в трёхмерном пространстве прямоугольную декартову систему координат
$Ox^i$, $і= 1, 2, 3$ с началом в точке $ O $. Тогда каждой точке пространства
$M$ будет взаимооднозначно соответствовать радиус-вектор $х$ с началом в точке
$O$ и концом в точке $M$. Выберем такой ортонормированный базис
$\bar{\textbf{e}}_i$, у которого линии действия векторов совпадают с осями
$Ox^i$, а начало с точкой $О$ декартовой системы координат. Такой базис будем
называть \emph{декартовым}. Радиус-вектор $x$ всегда можно разложить по базису
$\bar{\textbf{e}}_i$: $\label{eq:11}
\textbf{x} = x^i\bar{\textbf{e}}_i$ где $x^i$ --- координаты точки $M$ в
декартовой системе координат. Трёхмерное пространство, в котором существует
единая для всех точек прямоугольная декартова система координат, называют
трехмерным евклидовым.
Ввёдем криволинейные координаты $X^i$, которые связаны с $x^i$ функциями вида
\begin{equation}
\label{eq:12}
    \textbf{x} = \textbf{x}(X^j).
\end{equation}
Тогда радиус-вектор точки $M$ может быть представлен как функция координат 
$X^j$:
\begin{equation}
\label{eq:13}
    x^i = x^i(X^j).
\end{equation}
Будем далее предполагать, что функции $x^i$ непрерывно дифференцируемы и взаимнооднозначны, тогда их можно обратить:
\begin{equation}
\label{eq:14}
    X^j = X^j(x_i).
\end{equation}
Ввиду дифференцируемости функций $\textbf{x}$ можно ввести производные
\begin{equation}
\label{eq:15}
\textbf{R}_j = \frac{\partial\textbf{x}}{\partial X^j},
\end{equation}
которые образуют векторы, назваемые \emph{локальными векторами базиса}. Эти
векторы в $\textbf{R}_j$ образуют \textit{локальные векторы базиса}, они
направлены по касательным к координатным линиям $X_j\equiv\mathrm{const}$ в точке $M$.
В отличие от $\bar{\mathbf{e}}_i$, векторы базиса $\mathbf{R}_j$  меняются при
переходе от одной точки $M$ к другой, $M'$.

% Заметим, что хотя формально $\bar{\mathbf{e}}_i$, и $\mathbf{R}_j$ определялись
% в разных точках, их можно привести к одной точке, так как векторы
% $\bar{\mathbf{e}}_i$, являются свободными.


\paragraph {Якобиевы матрицы.} Можно связать векторы базисов
$\bar{\textbf{e}}_i$, и $\textbf{R}_j$  Из \eqref{eq:11}  и \eqref{eq:15}
находим выражение:
\begin{equation}
\label{eq:16}
    \textbf{R}_j = \frac{\partial\textbf{x}}{\partial X^j} = \frac{\partial
    x^i}{\partial X^j} \bar{\textbf{e}}_i = Q^i_j \bar{\textbf{e}}_i,
\end{equation}
где введен объект $\bar{Q}^j_i$ с двумя индексами
\begin{equation}\label{jacobi-matrix}
    \bar{Q}^j_i= \frac{\partial x^j}{\partial X^i}.
\end{equation}
Такие объекты всегда можно записать в виде матрицы. Матрицу
\eqref{jacobi-matrix} (где верхний индекс указывает на номер строки) называют
\emph{матрицей преобразования} или
\emph{якобиевой матрицей}.

В силу взамной однозначности функций \eqref{eq:14}, в любой точке $X^i$
детерминант якобиевой матрицы всегда отличен от нуля.

Обратную матрицу будем также иногда обозначать как 
\begin{equation}
\label{eq:19}
    \bar{P}^i_j\bar{Q}^j_k = \bar{Q}^i_j\bar{P}^j_k  = \delta^i_k,
\end{equation}
где $\delta^i_k$ --- смешанный символ Кронекера
\begin{equation}
\label{eq:110}
    \delta^i_k = \begin{cases}
      1, & i = k, \\
      0, & i \ne k.
    \end{cases}
\end{equation}
Также обозначают
\begin{equation}
\bar{P}^i_j= (\bar{Q}^{-1})^i_j   = (\bar{Q}^i_j)^{-1} 
\end{equation}
Введём также ковариантный символ Кронекера $\delta_{ik}$  и контравариантный
символ Кронекера $\delta^{ik}$, значения которых совладают с $\delta^i_k$.
% \begin{equation}
% \label{eq:110}
% 	\textbf{????}
% \end{equation}

Для якобиевой матрицы $\bar{Q}^j_i$ обратная матрица имеет вид
\begin{equation}
\label{eq:111}
    \bar{P}^j_i= \frac{\partial X^j}{\partial x^i}.
\end{equation}
С помощью $\bar{P}^j_i$ можно выразить векторы декартова базиса
$\bar{\mathbf{e}}_k$ через $\textbf{R}_i$ . Получаем
\begin{equation}
\label{eq:112}
\bar{\mathbf{e}}_k = \bar{P}^i_k \textbf{R}_i.
\end{equation}


\paragraph{Метрические матрицы.}
В силу ортонормированности векторов базиса, их скалярное произведение можно
записать с помощью символов Кронекера:
\begin{equation}
\label{eq:113}
    \bar{\textbf{e}}_k \cdot \bar{\textbf{e}}_j = \delta_{kj}.
\end{equation}
Тогда скалярное произведение векторов $\textbf{R}_i, \textbf{R}_j$ с помощью
уравнений \eqref{eq:16} и \eqref{eq:113}  можно представить в виде
\begin{equation}
\label{eq:114}
    \textbf{R}_i\cdot \textbf{R}_j = 
    \bar{Q}^k_i \bar{Q}^l_j \bar{\textbf{e}}_k \bar{\textbf{e}}_l =
    \bar{Q}^k_i \bar{Q}^l_j \delta_{kl} =:
    g_{ij}.
\end{equation}
 Матрица $g_{ij}$, введенная по формуле \eqref{eq:114}, называется
 \emph{метрической} или \emph{фундаментальной}.

Определитель метрической матрицы обозначим следующим образом:
\begin{equation}
\label{eq:115}
    g := \det(g_{ij}) = (\det(\bar{Q}^k_i))^2.
\end{equation}
Метрическая матрица, очевидно, является симметричной по индексам $і$, $j$.

В силу \eqref{eq:115} всегда $g \ne 0$, поэтому для $g_{ij}$ всегда существует
обратная метрическая матрица $g^{kl}$.
Непосредственной подстановкой выражения $g_{ij} = \bar{Q}^k_i \bar{Q}^l_j
\delta_{kl}$  можно убедиться в том, что обратная метрическая
матрица $g^{ij}$ выражается через обратную матрицу Якоби:
\begin{equation}
\label{eq:118}
    g^{ij} = \bar{P}^i_m \bar{P}^j_n \delta^{mn}.
\end{equation}

\paragraph{Векторы взаимного базиса.}
 С помощью $g^{ij}$ определяем векторы локального взаимного базиса $\textbf{R}_i
 =g^{ik}\textbf{R}_k$:
\begin{equation}
\label{eq:119}
    \textbf{R}_i = \delta^k_i\textbf{R}_k =
    g^{kl}g_{li}\textbf{R}_k = g_{li}\textbf{R}^l.
\end{equation}
