\que{Локальные  векторные базисы, метрические матрицы, преобразования  координат.}
димдим тензорное исчисление 17

\subsection{ Координаты и локальные векторы базиса}
  Локальные векторные базисы, метрические матрицы, преобразования координат.
Введем втрехмерном пространстве прямоугольную декартову систему координат$х^i, і= 1, 2, 3$ с началом вточке 0, тогда каждой точке пространства $М$ будет взаимооднозначно соответствовать радиус-вектор $х$ с началом в точке $O$ и концом в точке $М$ (рис.1.1). Выберем такой ортонормированный базис $\bar{\textbf{e}}_i$, укоторого линии действия векторов совпадают с осями $Ox^i$, а начало - с точкой $О$ декартовой системы координат. Такой базис будем называть декартовым. Радиус-вектор $x$ всегда можно разложить по базису $\bar{\textbf{e}}_i$:

\begin{equation}
    \textbf{x} = x^i\bar{\textbf{e}}_i
\end{equation}
где $x^i$ - координаты точки $М$ в декартовой системе координат. Трехмерное пространство, в котором существует единая для всех точек прямоугольная декартова система координат, называют трехмерным евклидовым.
Введем криволинейные координаты $X^i$, которые связаны с $x^i$ функциями вида:
\begin{equation}
    \textbf{x} = \textbf{x}(X^j)
\end{equation}
Тогда радиус-вектор точки $М$ может быть представлен как функция координат 
$X^j$:
\begin{equation}
    x^i = x^i(X^j)
\end{equation}
Будем далее предполагать, что функции $x^i$ непрерывно дифференцируемы и взаимнооднозначны, тогда их можно обратить:
\begin{equation}
    X^j = X^j(x_i)
\end{equation}
Ввиду дифференцируемости функций $\textbf{x}$ можно ввести производные:

\begin{equation}
    \textbf{R}_j = \partial\textbf{x}/\partial X^j
\end{equation}
которые образуют векторы, назваемые локальными векторами базиса. Эти векторы в $\textbf{R}_j$ образуют \textit{локальные вектора базиса}, они направлены по касательным к координатнымлиниям $X_j=const$ вточке $М$.\par
Вотличие от $\bar{\textbf{e}}_i$, векторы базиса $\textbf{R}_j$  меняются при переходе от одной точки $М$ к другой $М'$.\par
Заметим, что хотя формально $\bar{\textbf{e}}_i$, и $\textbf{R}_j$ определялись вразных точках, их можно привести к одной точке, так как векторы $\bar{\textbf{e}}_i$, являются свободными.\par


\subsection {Якобиевы матрицы}
Можно связать векторы базисов $\bar{\textbf{e}}_i$, и $\textbf{R}_j$  Из (1.5) и (1.1) находим выражение:
\begin{equation}
    \textbf{R}_j = \frac{\partial\textbf{x}}{\partial X^j} = \frac{\partial x^i}{\partial X^j} \bar{\textbf{e}}_i = Q^i_j \bar{\textbf{e}}_i
\end{equation}
где введен объект $\bar{Q}^j_i$ с двумя индексами
\begin{equation}
    \bar{Q}^j_i= \frac{\partial x^j}{\partial X^i}
\end{equation}
Такие объекты всегда можно записать в виде матрицы. Матрицу (1.7) называют матрицей преобразования или якобиевой матрицей.\par

В силу взамной однозначности функций (1.4), в любой точке $X^i$ детерминант якобиевой матрицы всегда отличен от нуля:

Обратную матрицу будем также иногда обозначать как 
\begin{equation}
    \bar{P}^i_j\bar{Q}^j_k = \bar{Q}^j_i\bar{P}^j_k  = \delta^i_k
\end{equation}
где $\delta^i_k$ смешанный символ Кронекера
\begin{equation}
    \delta^i_k = \begin{cases}
    1, i = k\\
    0, i \ne k
    \end{cases}
\end{equation}
Также обозначают
\begin{equation}
    \bar{P}^i_j= (\bar{Q^{-1}})^i_j   = (\bar{Q}^i_j)^{-1} 
\end{equation}
Введем также ковариантный символ Кронекера $\delta_{ik}$  и контравариантный символ Кронекера $\delta^{ik}$ , значения которых совладают с $\delta^i_k$ :

Для якобиевой матрицы $\bar{Q}^j_i$ обратная акобиева матрица имеет вид:
\begin{equation}
    \bar{P}^j_i= \frac{\partial X^j}{\partial x^i}
\end{equation}
Спомощью $\bar{P}^j_i$, можно выразить векторы декартова базиса $e_k$ через $\textbf{R}_i$ . Домножая (1.6) на $\bar{P}^i_k$, получаем
\begin{equation}
    e_k = \bar{P}^i_k \textbf{R}_i
\end{equation}
\subsection{Метрические матрицы}
В силу ортонормированности векторов е;, их скалярное произведение можно записать с помощью символов Кронекера:
\begin{equation}
    \bar{\textbf{e}}_k \cdot \bar{\textbf{e}}_j = \delta_{kj}
\end{equation}
Тогда скалярное произведение векторов $\textbf{R}_i, \textbf{R}_j$ с помощью уравнений (1.6) и (1.13) можно представить в виде:
\begin{equation}
    \textbf{R}_i\cdot \textbf{R}_j = 
    \bar{Q}^k_i \bar{Q}^l_j \bar{\textbf{e}}_k \bar{\textbf{e}}_l =
    \bar{Q}^k_i \bar{Q}^l_j \delta_{kl} =
    g_{ij}
\end{equation}
 Матрица $g_{ij}$, введенная по формуле (1.14), называется
метрической или фундаментальной.

Определитель метрической матрицы обозначим следующим образом:
\begin{equation}
    g = \det(g_{ij}) = (\det(\bar{Q}^k_i))^2
\end{equation}
Метрическая матрица, очевидно, является симметричной по индексам і,j;

В силу (1.8) всегда $g \ne 0$, поэтому для $g_{ij}$; всегда существует обратная метрическая матрица $g^{kl}$
Непосредственной подстановкой выражения $g_{ij} = \bar{Q}^k_i \bar{Q}^l_j \delta_{kl}$ в (1.17) можно убедиться в том, что обратная метрическая матрица $g^{ij}$ выражается через обратную матрицу Якоби:
\begin{equation}
    g^{ij} = \bar{P}^i_k \bar{P}^j_l \delta^{kl}
\end{equation}
\subsection{Векторы взаимного базиса}
 помощью $g^{ij}$ определяем векторы локального взаимного базиса $\textbf{R}_i =g^{ik}\textbf{R}_k$
(1.19)
\begin{equation}
    \textbf{R}_i = \delta^k_i\textbf{R}_k =
    g^{kl}g_{li}\textbf{R}_k = g_{li}\textbf{R}^l
\end{equation}