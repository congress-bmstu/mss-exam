\que{Динамические уравнения совместности деформаций в материальном и пространственном описании.}
стр 140 второй димдим (нелинейная механика)



\textit{
\begin{itemize}
    \item (1.2.10) -- Теорема 1.4. Тензоры деформации и градиент деформации связаны с
вектором перемещений u следующими соотношениями...
\item (1.4.7) Полной производной по времени от переменного векторного поля a
\item (1.1.15) С помощью смешанного произведения трех различных векторов локальных базисов можно вычислить объемы
\item (1.20) теорема 2.4
\item (1.1.35), (1.2.10) теорема 1.4, просто формулы для F во всяких разных видах
\end{itemize}
}
 
  



\paragraph{Динамические уравнения совместности в лагранжевом описании.}

 Уравнения совместности деформаций можно записать еще в одном эквивалентном виде -- через вектор скорости.

Рассмотрим уравнение (1.2.10), связывающее $\mathbf{F}$ с $\stackrel{\circ}{\nabla} \otimes \mathbf{u}^{\mathrm{T}}$, и продифференцируем его по $t$ с учетом определения вектора скорости $\textbf{v}$:

\begin{equation}
\label{eq:71}
\frac{d}{d t} \stackrel{\circ}{\nabla} \otimes \mathbf{u} =\stackrel{\circ}{\mathbf{r}}^{i} \otimes \frac{\partial^{2} \mathbf{u}}{\partial X^{i} \partial t}=\stackrel{\circ}{\mathbf{r}}^{i} \otimes \frac{\partial \mathbf{v}}{\partial X^{i}}=\stackrel{\circ}{\boldsymbol{\nabla}} \otimes \mathbf{v}=\frac{d \mathbf{F}^{\mathrm{T}}}{d t} 
\end{equation}


В результате получим динамическое уравнение совместности в лагранжевом описании:

\begin{equation}
\label{eq:72}
\frac{d \mathbf{F}^{\mathrm{T}}}{d t}=\stackrel{\circ}{\nabla} \otimes \mathbf{v}
\end{equation}

\begin{theorem} Условия совместности деформаций выполнены тогда и только тогда, когда в $\mathcal{K}$ существует поле тензора $\mathbf{F}\left(X^{i}, t\right)$, удовлетворяющее следующим условиям:
\begin{enumerate}
    \item $\operatorname{det} \mathbf{F} \neq 0$ в каждой точке $X^{i}$,
    \item $\mathbf{F}\left(X^{i}, 0\right)=\mathbf{E} n p u t=0$,
    \item поле $\mathbf{F}$ обладает векторным потенциалом, т.е. для него существует такое поле вектора $\mathbf{v}$, что выполняется уравнение (7.2) $\forall t>0$ и $\forall X^{i} \in V$.
\end{enumerate}
\end{theorem}

 Пусть выполнены условия совместности деформаций, тогда, согласно определению 2.12 , существует вектор перемещений $\mathbf{u}\left(X^{i}, t\right)$ вместе со своим градиентом $\stackrel{\circ}{\nabla} \otimes \mathbf{u}$. Проделывая преобразования (7.1), убеждаемся в справедливости (7.2).

Покажем справедливость обратного утверждения. Пусть существует вектор-функция $\mathbf{v}$, удовлетворяющая (7.2). Рассмотрим функцию $\widetilde{\mathbf{u}}\left(X^{i}, t\right)=\int_{0}^{t} \mathbf{v}\left(X^{i}, \tau\right) d \tau$. Эта функция удовлетворяет уравнению

\begin{equation}
\label{eq:73}
\stackrel{\circ}{\nabla} \otimes \widetilde{\mathbf{u}}=\stackrel{\circ}{\boldsymbol{\nabla}} \otimes \int_{0}^{t} \mathbf{v} d \tau=\int_{0}^{t} \boldsymbol{\nabla} \otimes \mathbf{v} d \tau=\int_{0}^{t} \frac{d \mathbf{F}^{\mathrm{T}}}{d \tau} d \tau=\mathbf{F}^{\mathrm{T}}-\mathbf{E} 
\end{equation}

Тогда на основе этой функции можно построить радиус-вектор $\widetilde{\mathbf{x}}=\stackrel{\circ}{\mathbf{x}}+\widetilde{\mathbf{u}}$, с помощью которого тензор $\mathbf{F}$ будет представлен в виде:
\begin{equation}
\label{eq:74}
\mathbf{F}^{\mathrm{T}}=\mathbf{E}+\stackrel{\circ}{\nabla} \otimes \widetilde{\mathbf{u}}=\stackrel{\circ}{\mathbf{r}}^{i} \otimes \frac{\partial(\mathbf{\circ}+\widetilde{\mathbf{x}})}{\partial X^{i}}=\stackrel{\circ}{\mathbf{r}}^{i} \otimes \widetilde{\mathbf{r}}_{i}
\end{equation}
где $\widetilde{\mathbf{r}}_{i}=\partial \widetilde{\mathbf{x}} / \partial X^{i}$. Но это означает, что $\widetilde{\mathbf{u}}$ и есть искомый вектор перемещений $\mathbf{u}$, а $\mathbf{F}$ - искомый градиент деформации, поскольку они удовлетворяют
всем кинематическим соотношениям: (1.1.35), (1.2.10) и др. Таким образом, существует вектор перемещений u, а, следовательно, выполнены условия совместности деформаций.

 
\paragraph{Динамические уравнения совместности в пространственном описании.}
Докажем вначале вспомогательное утверждение.
\begin{theorem}
Пусть выполнено уравнение неразрывности (1.1.15), тогда градиент деформации удовлетворяет следующему уравнению:
\begin{equation}
\label{eq:75}
\nabla \cdot(\rho \mathbf{F})=0 
\end{equation}
\end{theorem}

 Представим градиент деформации в диадном базисе:

\begin{equation}
\label{eq:76}
\rho \mathbf{F}=\rho F^{i j} \mathbf{r}_{i} \otimes \mathbf{r}_{j} 
\end{equation}

Воспользуемся формулой для дивергенции любого тензора [12]:

\begin{equation}
\label{eq:77}
\nabla \cdot (\rho \mathbf{F}) = 
\frac{1}{\sqrt{g}}
\frac{\partial}{\partial X^i}
(\rho \sqrt{g} F^{i j} \mathbf{r}_{j} ) = 
\frac{\stackrel{\circ}{\rho}}{\sqrt{g}}
\frac{\partial}{\partial X^i}
( \sqrt{\stackrel{\circ} g} \stackrel{\circ} {\mathbf{r}_{j} }) 
\end{equation}

Здесь использовано уравнение неразрывности $\rho=\stackrel{\circ}{\rho} \sqrt{\stackrel{o}{g} / g}$, а также очевидные соотношения:
\begin{equation}
\label{eq:78}
F^{ik}\textbf{r}_k =
F^{jk}\textbf{r}^i \cdot\textbf{r}_j \otimes \textbf{r}_k 
=\textbf{r}^i \cdot F
=(\textbf{r}^i \cdot\textbf{r}_k)\otimes \stackrel{\circ} {\textbf{r}^k} =\stackrel{\circ} {\textbf{r}^i}
\end{equation}


Дифференцируя \eqref{eq:76} по частям, получаем

\begin{equation}
\label{eq:79}
\nabla \cdot(\rho \mathbf{F})
=\frac{\stackrel{\circ}{\rho}}{\sqrt{g}}\left(\frac{\partial \sqrt{\circ}}{\partial X^{i}} \stackrel{\circ}{r}^{i}+\sqrt{\stackrel{\circ}{g}} \frac{\partial \stackrel{\circ}{\mathbf{r}}^{i}}{\partial X^{i}}\right)
=\frac{\stackrel{\circ}{\rho}}{\sqrt{g}}\left(\sqrt{\stackrel{\circ}{g}} \stackrel{\circ}{\Gamma}_{i s}^{s} \stackrel{\circ}{\mathbf{r}} \stackrel{ }{i}-\sqrt{\stackrel{\circ}{g}} \stackrel{\circ}{\Gamma}_{i s}^{s} \stackrel{\circ}{\mathbf{r}}^{i}\right)
=0 
\end{equation}

Здесь использованы свойства символов Кристоффеля [12].

Преобразуем теперь уравнение \eqref{eq:72} с учетом (1.1.37):

\begin{equation}
\label{eq:710}
\frac{d \mathbf{F}^{\mathrm{T}}}{d t}=\mathbf{F}^{\mathrm{T}} \cdot \boldsymbol{\nabla} \otimes \mathbf{v} 
\end{equation}

и воспользуемся уравнением неразрывности (1.15), которое умножим на $\mathbf{F}^{\mathrm{T}}$ :

\begin{equation}
\label{eq:711}
\frac{\partial \rho}{\partial t} \mathbf{F}^{\mathrm{T}}+\mathbf{F}^{\mathrm{T}} \boldsymbol{\nabla} \cdot \rho \mathbf{v}=0 
\end{equation}

Умножим \eqref{eq:79} на $\rho$ и применим определение (1.4.7) полной производной по времени:

\begin{equation}
\label{eq:712}
\rho \frac{d \mathbf{F}^{\mathrm{T}}}{d t}=\rho \frac{\partial \mathbf{F}^{\mathrm{T}}}{\partial t}+\rho \mathbf{v} \cdot \boldsymbol{\nabla} \otimes \mathbf{F}^{\mathrm{T}}=\rho \mathbf{F}^{\mathrm{T}} \cdot \boldsymbol{\nabla} \otimes \mathbf{v} 
\end{equation}


Сложим теперь уравнения \eqref{eq:710} и \eqref{eq:711}, в результате получим:


\begin{equation}
\label{eq:713}
\frac{\partial \rho \mathbf{F}^{\mathrm{T}}}{\partial t}+\boldsymbol{\nabla} \cdot\left(\rho \mathbf{v} \otimes \mathbf{F}^{\mathrm{T}}\right)-\rho \mathbf{F}^{\mathrm{T}} \cdot \boldsymbol{\nabla} \otimes \mathbf{v}=0 
\end{equation}


Умножая уравнение \eqref{eq:74} тензорно на $-\mathbf{v}$ (т.е. $-\boldsymbol{\nabla} \cdot(\rho \mathbf{F}) \otimes \mathbf{v}=0$ ) и складывая полученное выражение с \eqref{eq:712}, приходим к динамическому уравнению совместности в пространственном описании:


\begin{equation}
\label{eq:714}
\frac{\partial \rho \mathbf{F}^{\mathrm{T}}}{\partial t}+\nabla \cdot\left(\rho \mathbf{v} \otimes \mathbf{F}^{\mathrm{T}}-\rho \mathbf{F} \otimes \mathbf{v}\right)=0 
\end{equation}


Выполнение этого уравнения, так же как и \eqref{eq:72}, необходимо и достаточно для соблюдения условий совместности деформаций в $\mathcal{K}$.

Преобразуя первое и второе слагаемое в \eqref{eq:713} по формуле (1.20), динамическое уравнение совместности можно записать также в виде


\begin{equation}
\label{eq:715}
\rho \frac{d \mathbf{F}^{\mathrm{T}}}{d t}=\nabla \cdot(\rho \mathbf{F} \otimes \mathbf{v})
\end{equation}
